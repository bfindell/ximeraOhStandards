\documentclass[nooutcomes]{ximera}
%\documentclass[space,handout,nooutcomes]{ximera}

\usepackage{epsfig}

\graphicspath{
  {./}
  {figures/}
  {../laode}
  {../laode/figures}
}

\usepackage{epstopdf}
\epstopdfsetup{outdir=./}

\usepackage{morewrites}
\makeatletter
\newcommand\subfile[1]{%
\renewcommand{\input}[1]{}%
\begingroup\skip@preamble\otherinput{#1}\endgroup\par\vspace{\topsep}
\let\input\otherinput}
\makeatother

\newcommand{\EXER}{}
\newcommand{\includeexercises}{\EXER\directlua{dofile(kpse.find_file("exercises","lua"))}}

\newenvironment{computerExercise}{\begin{exercise}}{\end{exercise}}

%\newcounter{ccounter}
%\setcounter{ccounter}{1}
%\newcommand{\Chapter}[1]{\setcounter{chapter}{\arabic{ccounter}}\chapter{#1}\addtocounter{ccounter}{1}}

%\newcommand{\section}[1]{\section{#1}\setcounter{thm}{0}\setcounter{equation}{0}}

%\renewcommand{\theequation}{\arabic{chapter}.\arabic{section}.\arabic{equation}}
%\renewcommand{\thefigure}{\arabic{chapter}.\arabic{figure}}
%\renewcommand{\thetable}{\arabic{chapter}.\arabic{table}}

%\newcommand{\Sec}[2]{\section{#1}\markright{\arabic{ccounter}.\arabic{section}.#2}\setcounter{equation}{0}\setcounter{thm}{0}\setcounter{figure}{0}}
  
\newcommand{\Sec}[2]{\section{#1}}

\setcounter{secnumdepth}{2}
%\setcounter{secnumdepth}{1} 

%\newcounter{THM}
%\renewcommand{\theTHM}{\arabic{chapter}.\arabic{section}}

\newcommand{\trademark}{{R\!\!\!\!\!\bigcirc}}
%\newtheorem{exercise}{}

\newcommand{\dfield}{{\sf dfield9}}
\newcommand{\pplane}{{\sf pplane9}}
\newcommand{\PPLANE}{{\sf PPLANE9}}

% BADBAD: \newcommand{\Bbb}{\bf}

\newcommand{\R}{\mbox{$\Bbb{R}$}}
\newcommand{\C}{\mbox{$\Bbb{C}$}}
\newcommand{\Z}{\mbox{$\Bbb{Z}$}}
\newcommand{\N}{\mbox{$\Bbb{N}$}}
\newcommand{\D}{\mbox{{\bf D}}}
\usepackage{amssymb}
%\newcommand{\qed}{\hfill\mbox{\raggedright$\square$} \vspace{1ex}}
%\newcommand{\proof}{\noindent {\bf Proof:} \hspace{0.1in}}

\newcommand{\setmin}{\;\mbox{--}\;}
\newcommand{\Matlab}{{M\small{AT\-LAB}} }
\newcommand{\Matlabp}{{M\small{AT\-LAB}}}
\newcommand{\computer}{\Matlab Instructions}
\newcommand{\half}{\mbox{$\frac{1}{2}$}}
\newcommand{\compose}{\raisebox{.15ex}{\mbox{{\scriptsize$\circ$}}}}
\newcommand{\AND}{\quad\mbox{and}\quad}
\newcommand{\vect}[2]{\left(\begin{array}{c} #1_1 \\ \vdots \\
 #1_{#2}\end{array}\right)}
\newcommand{\mattwo}[4]{\left(\begin{array}{rr} #1 & #2\\ #3
&#4\end{array}\right)}
\newcommand{\mattwoc}[4]{\left(\begin{array}{cc} #1 & #2\\ #3
&#4\end{array}\right)}
\newcommand{\vectwo}[2]{\left(\begin{array}{r} #1 \\ #2\end{array}\right)}
\newcommand{\vectwoc}[2]{\left(\begin{array}{c} #1 \\ #2\end{array}\right)}

\newcommand{\ignore}[1]{}


\newcommand{\inv}{^{-1}}
\newcommand{\CC}{{\cal C}}
\newcommand{\CCone}{\CC^1}
\newcommand{\Span}{{\rm span}}
\newcommand{\rank}{{\rm rank}}
\newcommand{\trace}{{\rm tr}}
\newcommand{\RE}{{\rm Re}}
\newcommand{\IM}{{\rm Im}}
\newcommand{\nulls}{{\rm null\;space}}

\newcommand{\dps}{\displaystyle}
\newcommand{\arraystart}{\renewcommand{\arraystretch}{1.8}}
\newcommand{\arrayfinish}{\renewcommand{\arraystretch}{1.2}}
\newcommand{\Start}[1]{\vspace{0.08in}\noindent {\bf Section~\ref{#1}}}
\newcommand{\exer}[1]{\noindent {\bf \ref{#1}}}
\newcommand{\ans}{\textbf{Answer:} }
\newcommand{\matthree}[9]{\left(\begin{array}{rrr} #1 & #2 & #3 \\ #4 & #5 & #6
\\ #7 & #8 & #9\end{array}\right)}
\newcommand{\cvectwo}[2]{\left(\begin{array}{c} #1 \\ #2\end{array}\right)}
\newcommand{\cmatthree}[9]{\left(\begin{array}{ccc} #1 & #2 & #3 \\ #4 & #5 &
#6 \\ #7 & #8 & #9\end{array}\right)}
\newcommand{\vecthree}[3]{\left(\begin{array}{r} #1 \\ #2 \\
#3\end{array}\right)}
\newcommand{\cvecthree}[3]{\left(\begin{array}{c} #1 \\ #2 \\
#3\end{array}\right)}
\newcommand{\cmattwo}[4]{\left(\begin{array}{cc} #1 & #2\\ #3
&#4\end{array}\right)}

\newcommand{\Matrix}[1]{\ensuremath{\left(\begin{array}{rrrrrrrrrrrrrrrrrr} #1 \end{array}\right)}}

\newcommand{\Matrixc}[1]{\ensuremath{\left(\begin{array}{cccccccccccc} #1 \end{array}\right)}}



\renewcommand{\labelenumi}{\theenumi}
\newenvironment{enumeratea}%
{\begingroup
 \renewcommand{\theenumi}{\alph{enumi}}
 \renewcommand{\labelenumi}{(\theenumi)}
 \begin{enumerate}}
 {\end{enumerate}\endgroup}

\newcounter{help}
\renewcommand{\thehelp}{\thesection.\arabic{equation}}

%\newenvironment{equation*}%
%{\renewcommand\endequation{\eqno (\theequation)* $$}%
%   \begin{equation}}%
%   {\end{equation}\renewcommand\endequation{\eqno \@eqnnum
%$$\global\@ignoretrue}}

%\input{psfig.tex}

\author{Martin Golubitsky and Michael Dellnitz}

%\newenvironment{matlabEquation}%
%{\renewcommand\endequation{\eqno (\theequation*) $$}%
%   \begin{equation}}%
%   {\end{equation}\renewcommand\endequation{\eqno \@eqnnum
% $$\global\@ignoretrue}}

\newcommand{\soln}{\textbf{Solution:} }
\newcommand{\exercap}[1]{\centerline{Figure~\ref{#1}}}
\newcommand{\exercaptwo}[1]{\centerline{Figure~\ref{#1}a\hspace{2.1in}
Figure~\ref{#1}b}}
\newcommand{\exercapthree}[1]{\centerline{Figure~\ref{#1}a\hspace{1.2in}
Figure~\ref{#1}b\hspace{1.2in}Figure~\ref{#1}c}}
\newcommand{\para}{\hspace{0.4in}}

\usepackage{ifluatex}
\ifluatex
\ifcsname displaysolutions\endcsname%
\else
\renewenvironment{solution}{\suppress}{\endsuppress}
\fi
\else
\renewenvironment{solution}{}{}
\fi

%\ifxake
%\newenvironment{matlabEquation}{\begin{equation}}{\end{equation}}
%\else
\newenvironment{matlabEquation}%
{\let\oldtheequation\theequation\renewcommand{\theequation}{\oldtheequation*}\begin{equation}}%
  {\end{equation}\let\theequation\oldtheequation}
%\fi

\makeatother



\title{Vertical Angles}
\author{Brad Findell}
\begin{document}
\begin{abstract}
Proofs updated. 
\end{abstract}
\maketitle

\fixnote{Below are three different proofs.  Please consider them separately.  And in each proof,  which of the details should be included, and which should be omitted?}

\begin{problem}
Point P is the intersection of lines $m$ and $n$.  Prove that $\angle 1\cong \angle 3$.  

\begin{image}
\definecolor{qqqqff}{rgb}{0.,0.,1.}
\begin{tikzpicture}[line cap=round,line join=round,>=triangle 45,x=1.0cm,y=1.0cm]
\clip(-2.5,-1.3) rectangle (7.4,3.4);
\draw [line width=0.8pt,domain=-2.4:7.3] plot(\x,{(-0.-1.*\x)/-2.});
\draw [line width=0.8pt,domain=-2.4:7.3] plot(\x,{(--5.-1.*\x)/3.});
%\draw [line width=0.8pt,dash pattern=on 2pt off 2pt,domain=-2.4:7.3] plot(\x,{(--2.066-0.997*\x)/0.071});
\draw (1.2,1.2) node[anchor=north west] {$1$};
\draw (1.8,1.6) node[anchor=north west] {$2$};
\draw (2.45,1.3) node[anchor=north west] {$3$};
%\draw (2.,0.9) node[anchor=north west] {$4$};
%\begin{scriptsize}
\draw [fill=qqqqff] (2.,1.) circle (1.5pt);
\draw[color=qqqqff] (1.85,0.65) node {$P$};
%\draw [fill=qqqqff] (0.,0.) circle (1.5pt);
\draw[color=black] (-2.,-0.7) node {$m$};
%\draw [fill=qqqqff] (5.,0.) circle (1.5pt);
\draw[color=black] (-1.8,2.5) node {$n$};
%\draw[color=black] (2.1,3.) node {$k$};
%\draw [fill=qqqqff] (4.,2.) circle (1.5pt);
%\draw [fill=qqqqff] (-1.,2.) circle (1.5pt);
%\end{scriptsize}
\end{tikzpicture}
\end{image}

\fixnote{When students write equations about linear pairs, they often write two equations for non-overlapping linear pairs---which doesn't help.  The figure above is intended to help avoid that dead end, but it would be worthwhile to discuss that dead end anyway.}

\begin{enumerate}

\item $\angle 1\cong \angle 3$ because they are both \wordChoice{\choice{complementary}\choice[correct]{supplementary}\choice{opposite}\choice{congruent}} to $\angle 2$.

\detail{First write down equations about linear pairs of angles: 
\[
m\angle 1 + m\angle 2 = 180^\circ
\]
\[
m\angle 3 + m\angle 2 = 180^\circ
\]
By comparing the two equations, some students will see clearly that $m\angle 1=m\angle 3$.   
A more formal approach would be to do some algebra.  
}

\item A rotation of \wordChoice{\choice{$90^\circ$}\choice[correct]{$180^\circ$}\choice{$360^\circ$}} about $P$ maps $m$ onto itself, maps $n$ onto itself, and 
swaps $\angle 1$ and $\angle 3$.  Because rotations preserve angle measures, 
it must be that $\angle 1\cong \angle 3$.  

\detail{Line $m$ is the union of two opposite rays with endpoint $P$.  Check that the $180^\circ$ rotation about $P$ swaps these opposite rays.  The same idea holds for line $n$ so that together the sides of $\angle 1$ become the sides of $\angle 3$ and vice versa.}

\item Reflecting about the \wordChoice{\choice[correct]{bisector}\choice{supplement}\choice{opposite}} of $\angle 2$ swaps $\angle 1$ and $\angle 3$.  Because reflections preserve angle measures, it follows that $\angle 1\cong \angle 3$. 

\detail{The reflection swaps the two rays that are the sides of $\angle 2$. Because reflections take lines to lines, that reflection must swap not just the rays but lines $m$ and $n$.}

\end{enumerate}

\begin{image}
\definecolor{qqqqff}{rgb}{0.,0.,1.}
\begin{tikzpicture}[line cap=round,line join=round,>=triangle 45,x=1.0cm,y=1.0cm]
\clip(-2.5,-1.3) rectangle (7.4,3.4);
\draw [line width=0.8pt,domain=-2.4:7.3] plot(\x,{(-0.-1.*\x)/-2.});
\draw [line width=0.8pt,domain=-2.4:7.3] plot(\x,{(--5.-1.*\x)/3.});
\draw [line width=0.8pt,dash pattern=on 2pt off 2pt,domain=-2.4:7.3] plot(\x,{(--2.066-0.997*\x)/0.071});
\draw (1.2,1.2) node[anchor=north west] {$1$};
\draw (1.8,1.6) node[anchor=north west] {$2$};
\draw (2.45,1.3) node[anchor=north west] {$3$};
%\draw (2.,0.9) node[anchor=north west] {$4$};
%\begin{scriptsize}
\draw [fill=qqqqff] (2.,1.) circle (1.5pt);
\draw[color=qqqqff] (1.85,0.65) node {$P$};
%\draw [fill=qqqqff] (0.,0.) circle (1.5pt);
\draw[color=black] (-2.,-0.7) node {$m$};
%\draw [fill=qqqqff] (5.,0.) circle (1.5pt);
\draw[color=black] (-1.8,2.5) node {$n$};
\draw[color=black] (2.1,3.) node {$k$};
%\draw [fill=qqqqff] (4.,2.) circle (1.5pt);
%\draw [fill=qqqqff] (-1.,2.) circle (1.5pt);
%\end{scriptsize}
\end{tikzpicture}
\end{image}


\end{problem}


\end{document}
