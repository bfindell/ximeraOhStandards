\documentclass[nooutcomes]{ximera}
%\documentclass[space,handout,nooutcomes]{ximera}

% For preamble materials

\usepackage{pgf,tikz}
\usepackage{mathrsfs}
\usetikzlibrary{shapes,arrows}
\usepackage{framed}
\pgfplotsset{compat=1.16}

\def\fixnote#1{\begin{framed}{\textcolor{red}{Fix note: #1}}\end{framed}}  % Allows insertion of red notes about needed edits
%\def\fixnote#1{}

\def\detail#1{{\textcolor{blue}{Detail: #1}}}   

\graphicspath{
  {./}
  {proofs/}
}

%\pdfOnly{\renewcommand{\answer}[1][[yy]{\fbox{\hspace{1in}\rule[-.3\baselineskip]{0pt}{15pt}}}}


\newcommand{\N}{\mathbb N}
\newcommand{\W}{\mathbb W}
\newcommand{\C}{\mathbb C}
\newcommand{\Z}{\mathbb Z}
\newcommand{\Q}{\mathbb Q}
\newcommand{\R}{\mathbb R}




\title{Quadrilaterals}
\author{Brad Findell}
\begin{document}
\begin{abstract}
Proof. 
\end{abstract}
\maketitle


\begin{problem}
Adapted from Ohio's 2017 Geometry released item 13. 

Two pairs of parallel lines intersect to form a parallelogram as shown.  
\begin{image}
\includegraphics{Q13.png}
\end{image}
Complete the following proof that opposite angles of a parallelogram are congruent: 

\begin{enumerate}
\item $\angle 1 \cong \angle 2$ as \wordChoice{\choice{opposite angles}\choice[correct]{alternate interior angles}\choice{corresponding angles}}
for parallel lines \wordChoice{\choice[correct]{$m$ and $n$}\choice{$k$ and $l$}}.
\item $\angle 3 \cong \angle 2$ as \wordChoice{\choice{opposite angles}\choice{alternate interior angles}\choice[correct]{corresponding angles}}for parallel lines \wordChoice{\choice{$m$ and $n$}\choice[correct]{$k$ and $l$}}.
\item Then $\angle 1 \cong \angle 3$ because they are both congruent 
to $\angle 2$. 
\end{enumerate}
\end{problem}

\begin{problem}
Adapted from Ohio's 2018 Geometry released item 21. 

Given the parallelogram $WXYZ$, prove that $\overline{WX}\cong\overline{YZ}$. 

\begin{image}
\definecolor{qqqqff}{rgb}{0.,0.,1.}
\begin{tikzpicture}[line width=0.8pt,line cap=round,line join=round,x=1.0cm,y=1.0cm] % ,>=triangle 45
\clip(-0.5,-0.6) rectangle (5.5,2.5);
\draw (0.,0.)-- (4.,0.);
\draw (4.,0.)-- (5.,2.);
\draw (5.,2.)-- (1.,2.);
\draw (1.,2.)-- (0.,0.);
\draw (1.,2.)-- (4.,0.);
\begin{scriptsize}
\draw [fill=qqqqff] (0.,0.) circle (1.2pt);
\draw[color=qqqqff] (-0.28,-0.03) node {$Z$};
\draw [fill=qqqqff] (1.,2.) circle (1.2pt);
\draw[color=qqqqff] (0.86,2.29) node {$W$};
\draw [fill=qqqqff] (4.,0.) circle (1.2pt);
\draw[color=qqqqff] (4.22,0.07) node {$Y$};
\draw [fill=qqqqff] (5.,2.) circle (1.2pt);
\draw[color=qqqqff] (5.14,2.29) node {$X$};
\end{scriptsize}
\end{tikzpicture}
\end{image}

\fixnote{It really would help to have an online environment that allows students to mark diagrams.}

Complete the proof below: 
\begin{enumerate}
\item $\angle ZWY \cong \angle XYW$ as \wordChoice{\choice[correct]{alternate interior angles}\choice{corresponding angles}\choice{opposite angles}} for parallel segments \wordChoice{\choice[correct]{$\overline{WZ}$ and $\overline{XY}$}\choice{$\overline{WX}$ and $\overline{YZ}$}}.
\item $\angle ZYW \cong \angle XWY$ for the same reason, this time for parallel segments \wordChoice{\choice{$\overline{WZ}$ and $\overline{XY}$}\choice[correct]{$\overline{WX}$ and $\overline{YZ}$}}.
\item $\overline{WY}\cong\overline{YW}$ because a segment is congruent to itself. 
\item $\triangle WYZ \cong \triangle YWX$ by \wordChoice{\choice{SAS}\choice[correct]{ASA}\choice{SSS}}.  
\item Then $\overline{YZ}\cong\overline{WX}$ as corresponding parts of congruent triangles. 
\end{enumerate}
\fixnote{Maybe number the angles.}

\end{problem}


\begin{problem}

Quadrilateral $ABCD$ is a kite as marked.  Prove that $\overleftrightarrow{BD}$ is the perpendicular bisector of $\overline{AC}$. 

\begin{image}
\definecolor{qqqqff}{rgb}{0.,0.,1.}
\begin{tikzpicture}[line cap=round,line width=0.8pt,line join=round,>=triangle 45,x=1.0cm,y=1.0cm]
\clip(-5,-6) rectangle (5,4);
\draw (-3.,0.)-- (0.,-5.);
\draw (-1.440,-2.424) -- (-1.595,-2.516);
\draw (-1.405,-2.484) -- (-1.559,-2.576);
\draw (0.,-5.)-- (3.,0.);
\draw (1.405,-2.484) -- (1.559,-2.576);
\draw (1.440,-2.424) -- (1.595,-2.516);
\draw (3.,0.)-- (0.,2.);
\draw (1.450,0.925) -- (1.550,1.075);
\draw (0.,2.)-- (-3.,0.);
\draw (-1.450,0.925) -- (-1.550,1.075);
\draw (0.,2.)-- (0.,-5.);
\draw (-3.,0.)-- (3.,0.);
\draw (0.,2.)-- (3.,0.);
%\begin{scriptsize}
\draw [fill=qqqqff] (-3.,0.) circle (1.2pt);
\draw[color=qqqqff] (-3.24,0.31) node {$A$};
\draw [fill=qqqqff] (3.,0.) circle (1.2pt);
\draw[color=qqqqff] (3.22,0.21) node {$C$};
\draw [fill=qqqqff] (0.,2.) circle (1.2pt);
\draw[color=qqqqff] (0.14,2.29) node {$B$};
\draw [fill=qqqqff] (0.,-5.) circle (1.2pt);
\draw[color=qqqqff] (-0.26,-5.09) node {$D$};
%\draw [fill=qqqqff] (0.,0) circle (1.2pt);
%\draw[color=qqqqff] (-0.26,-.26) node {$X$};
%\end{scriptsize}
\end{tikzpicture}
\end{image}

Key theorem:  The points on a perpendicular bisector are exactly those that are equidistant from the endpoints of a segment.  

Proof:  Because $B$ and $D$ are each $\answer[format=string]{equidistant}$ from $A$ and $C$, they each must lie on the perpendicular bisector of segment $\answer{AC}$, which implies that 
$\overleftrightarrow{BD}$ is its perpendicular bisector.  

\end{problem}

\begin{problem}
Quadrilateral $ABCD$ is a kite as marked.  Prove that $\overleftrightarrow{BD}$ is the perpendicular bisector of $\overline{AC}$. 

\begin{image}
\definecolor{qqqqff}{rgb}{0.,0.,1.}
\begin{tikzpicture}[line cap=round,line width=0.8pt,line join=round,>=triangle 45,x=1.0cm,y=1.0cm]
\clip(-5,-6) rectangle (5,4);
\draw (-3.,0.)-- (0.,-5.);
\draw (-1.440,-2.424) -- (-1.595,-2.516);
\draw (-1.405,-2.484) -- (-1.559,-2.576);
\draw (0.,-5.)-- (3.,0.);
\draw (1.405,-2.484) -- (1.559,-2.576);
\draw (1.440,-2.424) -- (1.595,-2.516);
\draw (3.,0.)-- (0.,2.);
\draw (1.450,0.925) -- (1.550,1.075);
\draw (0.,2.)-- (-3.,0.);
\draw (-1.450,0.925) -- (-1.550,1.075);
\draw (0.,2.)-- (0.,-5.);
\draw (-3.,0.)-- (3.,0.);
\draw (0.,2.)-- (3.,0.);
%\begin{scriptsize}
\draw [fill=qqqqff] (-3.,0.) circle (1.2pt);
\draw[color=qqqqff] (-3.24,0.31) node {$A$};
\draw [fill=qqqqff] (3.,0.) circle (1.2pt);
\draw[color=qqqqff] (3.22,0.21) node {$C$};
\draw [fill=qqqqff] (0.,2.) circle (1.2pt);
\draw[color=qqqqff] (0.14,2.29) node {$B$};
\draw [fill=qqqqff] (0.,-5.) circle (1.2pt);
\draw[color=qqqqff] (-0.26,-5.09) node {$D$};
\draw [fill=qqqqff] (0.,0) circle (1.2pt);
\draw[color=qqqqff] (-0.26,-.26) node {$X$};
%\end{scriptsize}
\end{tikzpicture}
\end{image}

A proof that makes use of triangle congruence:

\begin{image}
\tikzstyle{block} = [rectangle, draw, fill=blue!20, 
    text width=8em, text centered, rounded corners, minimum height=2em]
\tikzstyle{Block} = [rectangle, draw, fill=green!20, 
    text width=10em, text centered, rounded corners, minimum height=3em]
\tikzstyle{implies} = [draw, -latex']

\begin{tikzpicture}[node distance = 1.5cm, auto]
    % Place nodes
    \node [Block] (a) {$\triangle BAD \cong \triangle BCD$};
    \node [block, left of=a, node distance = 4cm] (init) {$AD=CD$};
    \node [block, right of=a, node distance = 4cm] (init2) {$BD=BD$};
    \node [block, below of=init2] (init3) {$AB=CB$};
    \node [block, below of=a] (b) {$\angle ABD \cong \angle CBD$};
    \node [block, left of=b, node distance = 4cm]  (c) {$BX=BX$};
    \node [Block, below of=b] (d) {$\triangle ABX \cong \triangle CBX$};
    \node [block, below of=d] (e) {$AX=CX$};
    \node [block, below of=e] (f) {$X$ is midpoint of $\overline{AC}$};
    \node [block, right of=e, node distance = 4cm] (g) {$\angle BXA \cong \angle BXC$};    
    \node [block, right of=g, text width=10em, node distance = 4cm] (h) {$\angle BXA$ and $\angle BXC$ form a linear pair};
    \node [block, below of=g] (i) {$\overline{BX}\perp\overline{AC}$};

    \node [Block, text width=12em, below of=f] (m) 
       {Summary: $\overleftrightarrow{BD}$ is the $\perp$ 
       bisector of $\overline{AC}$.};    
      % Draw edges
    \path [implies] (init) -- (a);
    \path [implies] (init2) -- (a);
    \path [implies] (init3) -- (a);
    \path [implies] (init3) -- (d);
    \path [implies] (a) -- (b);
    \path [implies] (b) -- (d);
    \path [implies] (c) -- (d);
    \path [implies] (d) -- (e);
    \path [implies] (e) -- (f);
    \path [implies] (d) -- (g);
    \path [implies] (g) -- (i);
    \path [implies] (h) -- (i);
    \path [implies] (f) -- (m);
    \path [implies] (i) -- (m);
\end{tikzpicture}
\end{image}

\fixnote{Do we need a step about $\overleftrightarrow{BX}$ and $\overleftrightarrow{BD}$ being the same line?}

In the proof above, $\triangle BAD \cong \triangle BCD$ by $\answer[format=string]{SSS}$, and 
$\triangle ABX \cong \triangle CBX$ by $\answer[format=string]{SAS}$.  

%\begin{feedback}[correct]
%Paragraph proof:
%\\ $\overline{BD}\cong \overline{BD}$, so that $\triangle BAD \cong \triangle BCD$ by SSS.  
%\\ $\angle ABD \cong \angle CBD$ by CPCTC. 
%\\ $\overline{BX}\cong \overline{BX}$, so that $\triangle ABX \cong \triangle CBX$ by SAS.  
%\\ $\angle BXA \cong \angle BXC$ by CPCTC, and they are a linear pair, 
%so $\overline{BX}\perp\overline{AC}$. 
%\\ $\overline{AX}\cong \overline{CX}$ by CPCTC, so $X$ is the midpoint of $\overline{AC}$.
%\\ Thus, $\overleftrightarrow{BD}$ is the perpendicular bisector of $\overline{AC}$.
%\end{feedback}
\end{problem}

\end{document}
