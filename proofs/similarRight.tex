\documentclass[nooutcomes]{ximera}
%\documentclass[space,handout,nooutcomes]{ximera}

% For preamble materials

\usepackage{pgf,tikz}
\usepackage{mathrsfs}
\usetikzlibrary{shapes,arrows}
\usepackage{framed}
\pgfplotsset{compat=1.16}

\def\fixnote#1{\begin{framed}{\textcolor{red}{Fix note: #1}}\end{framed}}  % Allows insertion of red notes about needed edits
%\def\fixnote#1{}

\def\detail#1{{\textcolor{blue}{Detail: #1}}}   

\graphicspath{
  {./}
  {proofs/}
}

%\pdfOnly{\renewcommand{\answer}[1][[yy]{\fbox{\hspace{1in}\rule[-.3\baselineskip]{0pt}{15pt}}}}


\newcommand{\N}{\mathbb N}
\newcommand{\W}{\mathbb W}
\newcommand{\C}{\mathbb C}
\newcommand{\Z}{\mathbb Z}
\newcommand{\Q}{\mathbb Q}
\newcommand{\R}{\mathbb R}




\title{Similar Right Triangles}
\author{Brad Findell}
\begin{document}
\begin{abstract}
Proofs. 
\end{abstract}
\maketitle

\begin{problem}
Adapted from Ohio's 2017 Geometry released item 17.

\begin{image}
\definecolor{qqwuqq}{rgb}{0.,0.39215,0.}
\definecolor{uuuuuu}{rgb}{0.2667,0.2667,0.2667}
\definecolor{qqqqff}{rgb}{0.,0.,1.}
\begin{tikzpicture}[line cap=round,line join=round,>=triangle 45,x=1.0cm,y=1.0cm]
%\clip(-0.75,-1) rectangle (2.9,3.5);
\clip(-3,-.1) rectangle (5,3.3);
\draw[line width=0.8pt,color=qqwuqq,fill=qqwuqq,fill opacity=0.1] (0.,0.) -- (0.,0.25) -- (0.25,0.25) -- (0.25,0.) -- cycle;
\draw[line width=0.8pt,color=qqwuqq,fill=qqwuqq,fill opacity=0.1] (1.2457,1.1314) -- (1.0373,0.9925) -- (1.1762,0.7842) -- (1.3846,0.9231) -- cycle;
\draw[line width=0.8pt,color=qqwuqq,fill=qqwuqq,fill opacity=0.1] (1.1762,0.7842) -- (1.3152,0.5758) -- (1.5235,0.7147) -- (1.3846,0.9231) -- cycle;
\draw [line width=0.8pt] (0.,0.)-- (2.,0.);
\draw [line width=0.8pt] (0.,3.)-- (0.,0.);
\draw [line width=0.8pt] (0.,3.)-- (2.,0.);
\draw [line width=0.8pt,dash pattern=on 3pt off 3pt] (0.,0.)-- (1.3846,0.9231);
\begin{scriptsize}
\draw [fill=qqqqff] (0.,0.) circle (1.2pt);
\draw[color=qqqqff] (-0.25,0.15) node {$A$};
\draw [fill=qqqqff] (2.,0.) circle (1.2pt);
\draw[color=qqqqff] (2.20,0.15) node {$B$};
\draw [fill=qqqqff] (0.,3.) circle (1.2pt);
\draw[color=qqqqff] (0.22,3.20) node {$C$};
\draw [fill=uuuuuu] (1.3846,0.9231) circle (1.2pt);
\draw[color=uuuuuu] (1.58,1.10) node {$D$};
\end{scriptsize}
\end{tikzpicture}
\end{image}
 
%\begin{image}
%\includegraphics{Q17.png}
%\end{image}
Complete the following proof that $\triangle DAC$ is similar to $\triangle DBA$:
% The 2017 EOC item prompts for ``AA postulate,'' but AA is not a postulate.
 
\begin{enumerate}
\item $\triangle DBA\sim \triangle \answer[format=string]{ABC}$ by \wordChoice{\choice[correct]{AA similarity}\choice{CPCTC}\choice{right triangle similarity}} because they share $\angle B$ and they each have a right angle.
 
\item $\triangle DAC\sim \triangle \answer[format=string]{ABC}$ for the same reason because they share \wordChoice{\choice{$\angle A$}\choice{$\angle B$}\choice[correct]{$\angle C$}} and they each have a right angle.
 
\item $\triangle DAC\sim \triangle DBA$ because 
\wordChoice{\choice{CPCTC}\choice{right triangle similarity}\choice[correct]{they are both similar to$\triangle ABC$}}.
\end{enumerate}
 
\end{problem}


\begin{problem}
A different proof, also adapted from Ohio's 2017 Geometry released item 17. 

\begin{image}
\definecolor{qqwuqq}{rgb}{0.,0.39215,0.}
\definecolor{uuuuuu}{rgb}{0.2667,0.2667,0.2667}
\definecolor{qqqqff}{rgb}{0.,0.,1.}
\begin{tikzpicture}[line cap=round,line join=round,>=triangle 45,x=1.0cm,y=1.0cm,scale=0.8]
%\clip(-0.75,-1) rectangle (2.9,3.5);
%\clip(-2,-1) rectangle (4,3.5);
\draw[line width=0.8pt,color=qqwuqq,fill=qqwuqq,fill opacity=0.1] (0.,0.) -- (0.,0.25) -- (0.25,0.25) -- (0.25,0.) -- cycle; 
\draw[line width=0.8pt,color=qqwuqq,fill=qqwuqq,fill opacity=0.1] (1.2457,1.1314) -- (1.0373,0.9925) -- (1.1762,0.7842) -- (1.3846,0.9231) -- cycle; 
\draw[line width=0.8pt,color=qqwuqq,fill=qqwuqq,fill opacity=0.1] (1.1762,0.7842) -- (1.3152,0.5758) -- (1.5235,0.7147) -- (1.3846,0.9231) -- cycle; 
\draw [line width=0.8pt] (0.,0.)-- (2.,0.);
\draw [line width=0.8pt] (0.,3.)-- (0.,0.);
\draw [line width=0.8pt] (0.,3.)-- (2.,0.);
\draw [line width=0.8pt,dash pattern=on 3pt off 3pt] (0.,0.)-- (1.3846,0.9231);
\begin{scriptsize}
\draw [fill=qqqqff] (0.,0.) circle (1.2pt);
\draw[color=qqqqff] (-0.25,0.15) node {$A$};
\draw [fill=qqqqff] (2.,0.) circle (1.2pt);
\draw[color=qqqqff] (2.20,0.15) node {$B$};
\draw [fill=qqqqff] (0.,3.) circle (1.2pt);
\draw[color=qqqqff] (0.22,3.20) node {$C$};
\draw [fill=uuuuuu] (1.3846,0.9231) circle (1.2pt);
\draw[color=uuuuuu] (1.58,1.10) node {$D$};
\end{scriptsize}
\end{tikzpicture}
\end{image}

%\begin{image}
%\includegraphics{Q17.png}
%\end{image}
Complete the following proof that $\triangle DAC$ is similar to $\triangle DBA$: 
% The 2017 EOC item prompts for ``AA postulate,'' but AA is not a postulate.

\begin{enumerate}
\item $\angle B$ and $\angle BAD$ are $\answer[format=string]{complementary}$ because they are acute angles in a right triangle. 

\item $\angle DAC$ and $\angle BAD$ are complementary because they are adjacent angles that form $\angle BAC$, which is \wordChoice{\choice[correct]{right}\choice{acute}\choice{obtuse}}.  

\item $\angle B \cong \angle DAC$ because they are both complementary to $\angle BAD$.  

\item $\triangle DAC\sim \triangle DBA$ by \wordChoice{\choice[correct]{AA similarity}\choice{CPCTC}\choice{right triangle similarity}} because $\angle B \cong \angle DAC$ and they each have a right angle.
\end{enumerate}

\end{problem}


\end{document}
