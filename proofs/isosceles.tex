\documentclass[nooutcomes]{ximera}
%\documentclass[space,handout,nooutcomes]{ximera}

\usepackage{epsfig}

\graphicspath{
  {./}
  {figures/}
  {../laode}
  {../laode/figures}
}

\usepackage{epstopdf}
\epstopdfsetup{outdir=./}

\usepackage{morewrites}
\makeatletter
\newcommand\subfile[1]{%
\renewcommand{\input}[1]{}%
\begingroup\skip@preamble\otherinput{#1}\endgroup\par\vspace{\topsep}
\let\input\otherinput}
\makeatother

\newcommand{\EXER}{}
\newcommand{\includeexercises}{\EXER\directlua{dofile(kpse.find_file("exercises","lua"))}}

\newenvironment{computerExercise}{\begin{exercise}}{\end{exercise}}

%\newcounter{ccounter}
%\setcounter{ccounter}{1}
%\newcommand{\Chapter}[1]{\setcounter{chapter}{\arabic{ccounter}}\chapter{#1}\addtocounter{ccounter}{1}}

%\newcommand{\section}[1]{\section{#1}\setcounter{thm}{0}\setcounter{equation}{0}}

%\renewcommand{\theequation}{\arabic{chapter}.\arabic{section}.\arabic{equation}}
%\renewcommand{\thefigure}{\arabic{chapter}.\arabic{figure}}
%\renewcommand{\thetable}{\arabic{chapter}.\arabic{table}}

%\newcommand{\Sec}[2]{\section{#1}\markright{\arabic{ccounter}.\arabic{section}.#2}\setcounter{equation}{0}\setcounter{thm}{0}\setcounter{figure}{0}}
  
\newcommand{\Sec}[2]{\section{#1}}

\setcounter{secnumdepth}{2}
%\setcounter{secnumdepth}{1} 

%\newcounter{THM}
%\renewcommand{\theTHM}{\arabic{chapter}.\arabic{section}}

\newcommand{\trademark}{{R\!\!\!\!\!\bigcirc}}
%\newtheorem{exercise}{}

\newcommand{\dfield}{{\sf dfield9}}
\newcommand{\pplane}{{\sf pplane9}}
\newcommand{\PPLANE}{{\sf PPLANE9}}

% BADBAD: \newcommand{\Bbb}{\bf}

\newcommand{\R}{\mbox{$\Bbb{R}$}}
\newcommand{\C}{\mbox{$\Bbb{C}$}}
\newcommand{\Z}{\mbox{$\Bbb{Z}$}}
\newcommand{\N}{\mbox{$\Bbb{N}$}}
\newcommand{\D}{\mbox{{\bf D}}}
\usepackage{amssymb}
%\newcommand{\qed}{\hfill\mbox{\raggedright$\square$} \vspace{1ex}}
%\newcommand{\proof}{\noindent {\bf Proof:} \hspace{0.1in}}

\newcommand{\setmin}{\;\mbox{--}\;}
\newcommand{\Matlab}{{M\small{AT\-LAB}} }
\newcommand{\Matlabp}{{M\small{AT\-LAB}}}
\newcommand{\computer}{\Matlab Instructions}
\newcommand{\half}{\mbox{$\frac{1}{2}$}}
\newcommand{\compose}{\raisebox{.15ex}{\mbox{{\scriptsize$\circ$}}}}
\newcommand{\AND}{\quad\mbox{and}\quad}
\newcommand{\vect}[2]{\left(\begin{array}{c} #1_1 \\ \vdots \\
 #1_{#2}\end{array}\right)}
\newcommand{\mattwo}[4]{\left(\begin{array}{rr} #1 & #2\\ #3
&#4\end{array}\right)}
\newcommand{\mattwoc}[4]{\left(\begin{array}{cc} #1 & #2\\ #3
&#4\end{array}\right)}
\newcommand{\vectwo}[2]{\left(\begin{array}{r} #1 \\ #2\end{array}\right)}
\newcommand{\vectwoc}[2]{\left(\begin{array}{c} #1 \\ #2\end{array}\right)}

\newcommand{\ignore}[1]{}


\newcommand{\inv}{^{-1}}
\newcommand{\CC}{{\cal C}}
\newcommand{\CCone}{\CC^1}
\newcommand{\Span}{{\rm span}}
\newcommand{\rank}{{\rm rank}}
\newcommand{\trace}{{\rm tr}}
\newcommand{\RE}{{\rm Re}}
\newcommand{\IM}{{\rm Im}}
\newcommand{\nulls}{{\rm null\;space}}

\newcommand{\dps}{\displaystyle}
\newcommand{\arraystart}{\renewcommand{\arraystretch}{1.8}}
\newcommand{\arrayfinish}{\renewcommand{\arraystretch}{1.2}}
\newcommand{\Start}[1]{\vspace{0.08in}\noindent {\bf Section~\ref{#1}}}
\newcommand{\exer}[1]{\noindent {\bf \ref{#1}}}
\newcommand{\ans}{\textbf{Answer:} }
\newcommand{\matthree}[9]{\left(\begin{array}{rrr} #1 & #2 & #3 \\ #4 & #5 & #6
\\ #7 & #8 & #9\end{array}\right)}
\newcommand{\cvectwo}[2]{\left(\begin{array}{c} #1 \\ #2\end{array}\right)}
\newcommand{\cmatthree}[9]{\left(\begin{array}{ccc} #1 & #2 & #3 \\ #4 & #5 &
#6 \\ #7 & #8 & #9\end{array}\right)}
\newcommand{\vecthree}[3]{\left(\begin{array}{r} #1 \\ #2 \\
#3\end{array}\right)}
\newcommand{\cvecthree}[3]{\left(\begin{array}{c} #1 \\ #2 \\
#3\end{array}\right)}
\newcommand{\cmattwo}[4]{\left(\begin{array}{cc} #1 & #2\\ #3
&#4\end{array}\right)}

\newcommand{\Matrix}[1]{\ensuremath{\left(\begin{array}{rrrrrrrrrrrrrrrrrr} #1 \end{array}\right)}}

\newcommand{\Matrixc}[1]{\ensuremath{\left(\begin{array}{cccccccccccc} #1 \end{array}\right)}}



\renewcommand{\labelenumi}{\theenumi}
\newenvironment{enumeratea}%
{\begingroup
 \renewcommand{\theenumi}{\alph{enumi}}
 \renewcommand{\labelenumi}{(\theenumi)}
 \begin{enumerate}}
 {\end{enumerate}\endgroup}

\newcounter{help}
\renewcommand{\thehelp}{\thesection.\arabic{equation}}

%\newenvironment{equation*}%
%{\renewcommand\endequation{\eqno (\theequation)* $$}%
%   \begin{equation}}%
%   {\end{equation}\renewcommand\endequation{\eqno \@eqnnum
%$$\global\@ignoretrue}}

%\input{psfig.tex}

\author{Martin Golubitsky and Michael Dellnitz}

%\newenvironment{matlabEquation}%
%{\renewcommand\endequation{\eqno (\theequation*) $$}%
%   \begin{equation}}%
%   {\end{equation}\renewcommand\endequation{\eqno \@eqnnum
% $$\global\@ignoretrue}}

\newcommand{\soln}{\textbf{Solution:} }
\newcommand{\exercap}[1]{\centerline{Figure~\ref{#1}}}
\newcommand{\exercaptwo}[1]{\centerline{Figure~\ref{#1}a\hspace{2.1in}
Figure~\ref{#1}b}}
\newcommand{\exercapthree}[1]{\centerline{Figure~\ref{#1}a\hspace{1.2in}
Figure~\ref{#1}b\hspace{1.2in}Figure~\ref{#1}c}}
\newcommand{\para}{\hspace{0.4in}}

\usepackage{ifluatex}
\ifluatex
\ifcsname displaysolutions\endcsname%
\else
\renewenvironment{solution}{\suppress}{\endsuppress}
\fi
\else
\renewenvironment{solution}{}{}
\fi

%\ifxake
%\newenvironment{matlabEquation}{\begin{equation}}{\end{equation}}
%\else
\newenvironment{matlabEquation}%
{\let\oldtheequation\theequation\renewcommand{\theequation}{\oldtheequation*}\begin{equation}}%
  {\end{equation}\let\theequation\oldtheequation}
%\fi

\makeatother



\title{Isosceles Triangle Theorem}
\author{Brad Findell}
\begin{document}
\begin{abstract}
Proofs updated. 
\end{abstract}
\maketitle

% How to translate part of a TikZ image
%
%\begin{scope}[shift={(2,0)}]
%  ... insert graphic here
%\end{scope}
%
% This figure shows an $\answer[format=string]{angle bisector}$ 


%\begin{image}
%% Isosceles triangle ABC, marked
%\definecolor{qqqqff}{rgb}{0.,0.,1.}
%\definecolor{qqwuqq}{rgb}{0.,0.39215,0.}
%\begin{tikzpicture}[line width=0.8pt,line cap=round,line join=round,>=triangle 45,x=1.0cm,y=1.0cm]
%%\clip(-0.4,-0.4) rectangle (2.4,2.6);
%\clip(-1.4,-0.4) rectangle (3.4,2.6);
%%\draw [shift={(1.,2.2)},color=qqwuqq,fill=qqwuqq,fill opacity=0.1] (0,0) -- (-114.444:0.404) arc (-114.444:-65.556:0.404) -- cycle;  % Mark angle C
%%\draw [shift={(1.,2.2)},color=qqwuqq,fill=qqwuqq,fill opacity=0.1] (0,0) -- (-114.444:0.55) arc (-114.444:-90.:0.55) -- cycle;  % Mark angle DCA
%%\draw [shift={(1.,2.2)},color=qqwuqq,fill=qqwuqq,fill opacity=0.1] (0,0) -- (-90.:0.48) arc (-90.:-65.556:0.48) -- cycle;  % Mark angle DCB
%%\draw[line width=0.8pt,color=qqwuqq,fill=qqwuqq,fill opacity=0.1] (1.,0.2) -- (0.8,0.2) -- (0.8,0.) -- (1.,0.) -- cycle; 
%%\draw[line width=0.8pt,color=qqwuqq,fill=qqwuqq,fill opacity=0.1] (1.2,0.) -- (1.2,0.2) -- (1.,0.2) -- (1.,0.) -- cycle; 
%\draw (0.,0.)-- (1.,2.2);
%\draw (0.4171,1.1376) -- (0.5828,1.0623);
%\draw (1.,2.2)-- (2.,0.);
%\draw (1.5828,1.1376) -- (1.4171,1.0623);
%\draw (0.,0.)-- (2.,0.);
%%\draw (1.,2.2)-- (1.,0.); % segment CD
%%\draw (1.0910,1.1354) -- (0.9090,1.1354);  %  Mark on CD
%%\draw (1.0910,1.0646) -- (0.9090,1.0646);  %  Mark on CD
%%\draw (0.4292,0.0910) -- (0.4292,-0.0910); % Mark on AD
%%\draw (0.5,0.0910) -- (0.5,-0.0910);       % Mark on AD
%%\draw (0.5708,0.0910) -- (0.5708,-0.0910); % Mark on AD
%%\draw (1.4292,0.0910) -- (1.4292,-0.0910); % Mark on DB
%%\draw (1.5,0.0910) -- (1.5,-0.0910);       % Mark on DB
%%\draw (1.5708,0.0910) -- (1.5708,-0.0910); % Mark on DB
%\begin{scriptsize}
%\draw [fill=qqqqff] (0.,0.) circle (1.2pt);
%\draw[color=qqqqff] (-0.18,-0.13) node {$A$};
%\draw [fill=qqqqff] (2.,0.) circle (1.2pt);
%\draw[color=qqqqff] (2.18,-0.13) node {$B$};
%\draw [fill=qqqqff] (1.,2.2) circle (1.2pt);
%\draw[color=qqqqff] (1.14,2.45) node {$C$};
%%\draw [fill=qqqqff] (1.,0.) circle (1.2pt);
%%\draw[color=qqqqff] (0.97,-0.2) node {$D$};
%\end{scriptsize}
%\end{tikzpicture}
%\end{image}

\fixnote{Below are several different proofs, along with one that is not a proof.  Please consider them separately.\\Any (or all) of the proofs might be extended to conclude that, in the case of an isosceles triangle, the perpendicular bisector, angle bisector, median, and altitude all lie on the same line.}

\begin{problem}
Prove that the base angles of an isosceles triangle are congruent.   

\begin{image}
\definecolor{qqqqff}{rgb}{0.,0.,1.}
\definecolor{qqwuqq}{rgb}{0.,0.39215,0.}
\begin{tikzpicture}[line width=0.8pt,line cap=round,line join=round,>=triangle 45,x=1.0cm,y=1.0cm]
%\clip(-0.4,-0.4) rectangle (2.4,2.6);
\clip(-0.4,-0.4) rectangle (5.4,2.6);
%\draw [shift={(1.,2.2)},color=qqwuqq,fill=qqwuqq,fill opacity=0.1] (0,0) -- (-114.444:0.404) arc (-114.444:-65.556:0.404) -- cycle;  % Mark angle C
%\draw [shift={(1.,2.2)},color=qqwuqq,fill=qqwuqq,fill opacity=0.1] (0,0) -- (-114.444:0.55) arc (-114.444:-90.:0.55) -- cycle;  % Mark angle DCA
%\draw [shift={(1.,2.2)},color=qqwuqq,fill=qqwuqq,fill opacity=0.1] (0,0) -- (-90.:0.48) arc (-90.:-65.556:0.48) -- cycle;  % Mark angle DCB
%\draw[line width=0.8pt,color=qqwuqq,fill=qqwuqq,fill opacity=0.1] (1.,0.2) -- (0.8,0.2) -- (0.8,0.) -- (1.,0.) -- cycle; 
%\draw[line width=0.8pt,color=qqwuqq,fill=qqwuqq,fill opacity=0.1] (1.2,0.) -- (1.2,0.2) -- (1.,0.2) -- (1.,0.) -- cycle; 
\draw (0.,0.)-- (1.,2.2);
\draw (0.4171,1.1376) -- (0.5828,1.0623);
\draw (1.,2.2)-- (2.,0.);
\draw (1.5828,1.1376) -- (1.4171,1.0623);
\draw (0.,0.)-- (2.,0.);
%\draw (1.,2.2)-- (1.,0.); % segment CD
%\draw (1.0910,1.1354) -- (0.9090,1.1354);  %  Mark on CD
%\draw (1.0910,1.0646) -- (0.9090,1.0646);  %  Mark on CD
%\draw (0.4292,0.0910) -- (0.4292,-0.0910); % Mark on AD
%\draw (0.5,0.0910) -- (0.5,-0.0910);       % Mark on AD
%\draw (0.5708,0.0910) -- (0.5708,-0.0910); % Mark on AD
%\draw (1.4292,0.0910) -- (1.4292,-0.0910); % Mark on DB
%\draw (1.5,0.0910) -- (1.5,-0.0910);       % Mark on DB
%\draw (1.5708,0.0910) -- (1.5708,-0.0910); % Mark on DB
\begin{scriptsize}
\draw [fill=qqqqff] (0.,0.) circle (1.2pt);
\draw[color=qqqqff] (-0.18,-0.13) node {$A$};
\draw [fill=qqqqff] (2.,0.) circle (1.2pt);
\draw[color=qqqqff] (2.18,-0.13) node {$B$};
\draw [fill=qqqqff] (1.,2.2) circle (1.2pt);
\draw[color=qqqqff] (1.14,2.45) node {$C$};
%\draw [fill=qqqqff] (1.,0.) circle (1.2pt);
%\draw[color=qqqqff] (0.97,-0.2) node {$D$};
\end{scriptsize}
%\end{tikzpicture}

\begin{scope}[shift={(3,0)}]
% Isosceles Triangle with Angle Bisector
%\definecolor{qqqqff}{rgb}{0.,0.,1.}
%\definecolor{qqwuqq}{rgb}{0.,0.39215,0.}
%\begin{tikzpicture}[line width=0.8pt,line cap=round,line join=round,>=triangle 45,x=1.0cm,y=1.0cm]
%\clip(-0.4,-0.4) rectangle (2.4,2.6);
%\draw [shift={(1.,2.2)},color=qqwuqq,fill=qqwuqq,fill opacity=0.1] (0,0) -- (-114.444:0.404) arc (-114.444:-65.556:0.404) -- cycle;  % Mark angle C
\draw [shift={(1.,2.2)},color=qqwuqq,fill=qqwuqq,fill opacity=0.1] (0,0) -- (-114.444:0.55) arc (-114.444:-90.:0.55) -- cycle;  % Mark angle DCA
\draw [shift={(1.,2.2)},color=qqwuqq,fill=qqwuqq,fill opacity=0.1] (0,0) -- (-90.:0.48) arc (-90.:-65.556:0.48) -- cycle;  % Mark angle DCB
%\draw[line width=0.8pt,color=qqwuqq,fill=qqwuqq,fill opacity=0.1] (1.,0.2) -- (0.8,0.2) -- (0.8,0.) -- (1.,0.) -- cycle; 
%\draw[line width=0.8pt,color=qqwuqq,fill=qqwuqq,fill opacity=0.1] (1.2,0.) -- (1.2,0.2) -- (1.,0.2) -- (1.,0.) -- cycle; 
\draw (0.,0.)-- (1.,2.2);
\draw (0.4171,1.1376) -- (0.5828,1.0623);
\draw (1.,2.2)-- (2.,0.);
\draw (1.5828,1.1376) -- (1.4171,1.0623);
\draw (0.,0.)-- (2.,0.);
\draw (1.,2.2)-- (1.,0.); % segment CD
\draw (1.0910,1.1354) -- (0.9090,1.1354);  %  Mark on CD
\draw (1.0910,1.0646) -- (0.9090,1.0646);  %  Mark on CD
%\draw (0.4292,0.0910) -- (0.4292,-0.0910); % Mark on AD
%\draw (0.5,0.0910) -- (0.5,-0.0910);       % Mark on AD
%\draw (0.5708,0.0910) -- (0.5708,-0.0910); % Mark on AD
%\draw (1.4292,0.0910) -- (1.4292,-0.0910); % Mark on DB
%\draw (1.5,0.0910) -- (1.5,-0.0910);       % Mark on DB
%\draw (1.5708,0.0910) -- (1.5708,-0.0910); % Mark on DB
\begin{scriptsize}
\draw [fill=qqqqff] (0.,0.) circle (1.2pt);
\draw[color=qqqqff] (-0.18,-0.13) node {$A$};
\draw [fill=qqqqff] (2.,0.) circle (1.2pt);
\draw[color=qqqqff] (2.18,-0.13) node {$B$};
\draw [fill=qqqqff] (1.,2.2) circle (1.2pt);
\draw[color=qqqqff] (1.14,2.45) node {$C$};
\draw [fill=qqqqff] (1.,0.) circle (1.2pt);
\draw[color=qqqqff] (0.97,-0.2) node {$D$};
\end{scriptsize}
\end{scope}
\end{tikzpicture}
\end{image}

\begin{enumerate}
\item Beginning with the given figure on the left, Morgan draws $\overline{CD}$ and marks the figure intending that this new segment is a(n) \wordChoice{\choice{median}\choice[correct]{angle bisector}\choice{perpendicular bisector}\choice{altitude}}.

\item Based on the marked figure, Morgan claims that the $\triangle ACD\cong \triangle\answer[format=string]{BCD}$ by \wordChoice{\choice[correct]{SAS}\choice{SSS}\choice{SSA}\choice{ASA}\choice{HL}}. 

\item Finally, Morgan concludes that $\angle A\cong \angle\answer[format=string]{B}$, as they are corresponding parts of congruent triangles. 
\end{enumerate}

\end{problem}

\begin{problem}
Prove that the base angles of an isosceles triangle are congruent.   

\begin{image}
\definecolor{qqqqff}{rgb}{0.,0.,1.}
\definecolor{qqwuqq}{rgb}{0.,0.3921,0.}
\begin{tikzpicture}[line width=0.8pt,line cap=round,line join=round,>=triangle 45,x=1.0cm,y=1.0cm]
%\clip(-0.4,-0.4) rectangle (2.4,2.6);
\clip(-0.4,-0.4) rectangle (5.4,2.6);
%\draw [shift={(1.,2.2)},color=qqwuqq,fill=qqwuqq,fill opacity=0.1] (0,0) -- (-114.444:0.404) arc (-114.444:-65.556:0.404) -- cycle;  % Mark angle C
%\draw [shift={(1.,2.2)},color=qqwuqq,fill=qqwuqq,fill opacity=0.1] (0,0) -- (-114.444:0.55) arc (-114.444:-90.:0.55) -- cycle;  % Mark angle DCA
%\draw [shift={(1.,2.2)},color=qqwuqq,fill=qqwuqq,fill opacity=0.1] (0,0) -- (-90.:0.48) arc (-90.:-65.556:0.48) -- cycle;  % Mark angle DCB
%\draw[line width=0.8pt,color=qqwuqq,fill=qqwuqq,fill opacity=0.1] (1.,0.2) -- (0.8,0.2) -- (0.8,0.) -- (1.,0.) -- cycle; 
%\draw[line width=0.8pt,color=qqwuqq,fill=qqwuqq,fill opacity=0.1] (1.2,0.) -- (1.2,0.2) -- (1.,0.2) -- (1.,0.) -- cycle; 
\draw (0.,0.)-- (1.,2.2);
\draw (0.4171,1.1376) -- (0.5828,1.0623);
\draw (1.,2.2)-- (2.,0.);
\draw (1.5828,1.1376) -- (1.4171,1.0623);
\draw (0.,0.)-- (2.,0.);
%\draw (1.,2.2)-- (1.,0.); % segment CD
%\draw (1.0910,1.1354) -- (0.9090,1.1354);  %  Mark on CD
%\draw (1.0910,1.0646) -- (0.9090,1.0646);  %  Mark on CD
%\draw (0.4292,0.0910) -- (0.4292,-0.0910); % Mark on AD
%\draw (0.5,0.0910) -- (0.5,-0.0910);       % Mark on AD
%\draw (0.5708,0.0910) -- (0.5708,-0.0910); % Mark on AD
%\draw (1.4292,0.0910) -- (1.4292,-0.0910); % Mark on DB
%\draw (1.5,0.0910) -- (1.5,-0.0910);       % Mark on DB
%\draw (1.5708,0.0910) -- (1.5708,-0.0910); % Mark on DB
\begin{scriptsize}
\draw [fill=qqqqff] (0.,0.) circle (1.2pt);
\draw[color=qqqqff] (-0.18,-0.13) node {$A$};
\draw [fill=qqqqff] (2.,0.) circle (1.2pt);
\draw[color=qqqqff] (2.18,-0.13) node {$B$};
\draw [fill=qqqqff] (1.,2.2) circle (1.2pt);
\draw[color=qqqqff] (1.14,2.45) node {$C$};
%\draw [fill=qqqqff] (1.,0.) circle (1.2pt);
%\draw[color=qqqqff] (0.97,-0.2) node {$D$};
\end{scriptsize}
%\end{tikzpicture}

\begin{scope}[shift={(3,0)}]
% Isosceles triangle with median
%\definecolor{qqqqff}{rgb}{0.,0.,1.}
%\definecolor{qqwuqq}{rgb}{0.,0.39215,0.}
%\begin{tikzpicture}[line width=0.8pt,line cap=round,line join=round,>=triangle 45,x=1.0cm,y=1.0cm]
%\clip(-0.4,-0.4) rectangle (2.4,2.6);
%\draw [shift={(1.,2.2)},color=qqwuqq,fill=qqwuqq,fill opacity=0.1] (0,0) -- (-114.444:0.404) arc (-114.444:-65.556:0.404) -- cycle;  % Mark angle C
%\draw [shift={(1.,2.2)},color=qqwuqq,fill=qqwuqq,fill opacity=0.1] (0,0) -- (-114.444:0.55) arc (-114.444:-90.:0.55) -- cycle;  % Mark angle DCA
%\draw [shift={(1.,2.2)},color=qqwuqq,fill=qqwuqq,fill opacity=0.1] (0,0) -- (-90.:0.48) arc (-90.:-65.556:0.48) -- cycle;  % Mark angle DCB
%\draw[line width=0.8pt,color=qqwuqq,fill=qqwuqq,fill opacity=0.1] (1.,0.2) -- (0.8,0.2) -- (0.8,0.) -- (1.,0.) -- cycle; 
%\draw[line width=0.8pt,color=qqwuqq,fill=qqwuqq,fill opacity=0.1] (1.2,0.) -- (1.2,0.2) -- (1.,0.2) -- (1.,0.) -- cycle; 
\draw (0.,0.)-- (1.,2.2);
\draw (0.4171,1.1376) -- (0.5828,1.0623);
\draw (1.,2.2)-- (2.,0.);
\draw (1.5828,1.1376) -- (1.4171,1.0623);
\draw (0.,0.)-- (2.,0.);
\draw (1.,2.2)-- (1.,0.); % segment CD
\draw (1.0910,1.1354) -- (0.9090,1.1354);  %  Mark on CD
\draw (1.0910,1.0646) -- (0.9090,1.0646);  %  Mark on CD
\draw (0.4292,0.0910) -- (0.4292,-0.0910); % Mark on AD
\draw (0.5,0.0910) -- (0.5,-0.0910);       % Mark on AD
\draw (0.5708,0.0910) -- (0.5708,-0.0910); % Mark on AD
\draw (1.4292,0.0910) -- (1.4292,-0.0910); % Mark on DB
\draw (1.5,0.0910) -- (1.5,-0.0910);       % Mark on DB
\draw (1.5708,0.0910) -- (1.5708,-0.0910); % Mark on DB
\begin{scriptsize}
\draw [fill=qqqqff] (0.,0.) circle (1.2pt);
\draw[color=qqqqff] (-0.18,-0.13) node {$A$};
\draw [fill=qqqqff] (2.,0.) circle (1.2pt);
\draw[color=qqqqff] (2.18,-0.13) node {$B$};
\draw [fill=qqqqff] (1.,2.2) circle (1.2pt);
\draw[color=qqqqff] (1.14,2.45) node {$C$};
\draw [fill=qqqqff] (1.,0.) circle (1.2pt);
\draw[color=qqqqff] (0.97,-0.2) node {$D$};
\end{scriptsize}
\end{scope}
\end{tikzpicture}

\end{image}

\begin{enumerate}
\item Beginning with the given figure on the left, Deja draws $\overline{CD}$ and marks the figure intending that this new segment is a(n) \wordChoice{\choice[correct]{median}\choice{angle bisector}\choice{perpendicular bisector}\choice{altitude}}.

\item Based on the marked figure, Deja claims that the $\triangle ACD\cong \triangle\answer[format=string]{BCD}$ by \wordChoice{\choice{SAS}\choice[correct]{SSS}\choice{SSA}\choice{ASA}\choice{HL}}. 

\item Finally, Deja concludes that $\angle A\cong \angle\answer[format=string]{B}$, as they are corresponding parts of congruent triangles. 
\end{enumerate}

\end{problem}

\begin{problem}
Prove that the base angles of an isosceles triangle are congruent.   

\begin{image}
\definecolor{qqqqff}{rgb}{0.,0.,1.}
\definecolor{qqwuqq}{rgb}{0.,0.39215,0.}
\begin{tikzpicture}[line width=0.8pt,line cap=round,line join=round,>=triangle 45,x=1.0cm,y=1.0cm]
%\clip(-0.4,-0.4) rectangle (2.4,2.6);
\clip(-0.4,-0.4) rectangle (5.4,2.6);
%\draw [shift={(1.,2.2)},color=qqwuqq,fill=qqwuqq,fill opacity=0.1] (0,0) -- (-114.444:0.404) arc (-114.444:-65.556:0.404) -- cycle;  % Mark angle C
%\draw [shift={(1.,2.2)},color=qqwuqq,fill=qqwuqq,fill opacity=0.1] (0,0) -- (-114.444:0.55) arc (-114.444:-90.:0.55) -- cycle;  % Mark angle DCA
%\draw [shift={(1.,2.2)},color=qqwuqq,fill=qqwuqq,fill opacity=0.1] (0,0) -- (-90.:0.48) arc (-90.:-65.556:0.48) -- cycle;  % Mark angle DCB
%\draw[line width=0.8pt,color=qqwuqq,fill=qqwuqq,fill opacity=0.1] (1.,0.2) -- (0.8,0.2) -- (0.8,0.) -- (1.,0.) -- cycle; 
%\draw[line width=0.8pt,color=qqwuqq,fill=qqwuqq,fill opacity=0.1] (1.2,0.) -- (1.2,0.2) -- (1.,0.2) -- (1.,0.) -- cycle; 
\draw (0.,0.)-- (1.,2.2);
\draw (0.4171,1.1376) -- (0.5828,1.0623);
\draw (1.,2.2)-- (2.,0.);
\draw (1.5828,1.1376) -- (1.4171,1.0623);
\draw (0.,0.)-- (2.,0.);
%\draw (1.,2.2)-- (1.,0.); % segment CD
%\draw (1.0910,1.1354) -- (0.9090,1.1354);  %  Mark on CD
%\draw (1.0910,1.0646) -- (0.9090,1.0646);  %  Mark on CD
%\draw (0.4292,0.0910) -- (0.4292,-0.0910); % Mark on AD
%\draw (0.5,0.0910) -- (0.5,-0.0910);       % Mark on AD
%\draw (0.5708,0.0910) -- (0.5708,-0.0910); % Mark on AD
%\draw (1.4292,0.0910) -- (1.4292,-0.0910); % Mark on DB
%\draw (1.5,0.0910) -- (1.5,-0.0910);       % Mark on DB
%\draw (1.5708,0.0910) -- (1.5708,-0.0910); % Mark on DB
\begin{scriptsize}
\draw [fill=qqqqff] (0.,0.) circle (1.2pt);
\draw[color=qqqqff] (-0.18,-0.13) node {$A$};
\draw [fill=qqqqff] (2.,0.) circle (1.2pt);
\draw[color=qqqqff] (2.18,-0.13) node {$B$};
\draw [fill=qqqqff] (1.,2.2) circle (1.2pt);
\draw[color=qqqqff] (1.14,2.45) node {$C$};
%\draw [fill=qqqqff] (1.,0.) circle (1.2pt);
%\draw[color=qqqqff] (0.97,-0.2) node {$D$};
\end{scriptsize}
%\end{tikzpicture}

\begin{scope}[shift={(3,0)}]
% Isosceles triangle with altitude
%\definecolor{qqqqff}{rgb}{0.,0.,1.}
%\definecolor{qqwuqq}{rgb}{0.,0.39215,0.}
%\begin{tikzpicture}[line width=0.8pt,line cap=round,line join=round,>=triangle 45,x=1.0cm,y=1.0cm]
%\clip(-0.4,-0.4) rectangle (2.4,2.6);
%\draw [shift={(1.,2.2)},color=qqwuqq,fill=qqwuqq,fill opacity=0.1] (0,0) -- (-114.444:0.404) arc (-114.444:-65.556:0.404) -- cycle;  % Mark angle C
%\draw [shift={(1.,2.2)},color=qqwuqq,fill=qqwuqq,fill opacity=0.1] (0,0) -- (-114.444:0.55) arc (-114.444:-90.:0.55) -- cycle;  % Mark angle DCA
%\draw [shift={(1.,2.2)},color=qqwuqq,fill=qqwuqq,fill opacity=0.1] (0,0) -- (-90.:0.48) arc (-90.:-65.556:0.48) -- cycle;  % Mark angle DCB
\draw[line width=0.8pt,color=qqwuqq,fill=qqwuqq,fill opacity=0.1] (1.,0.2) -- (0.8,0.2) -- (0.8,0.) -- (1.,0.) -- cycle; 
\draw[line width=0.8pt,color=qqwuqq,fill=qqwuqq,fill opacity=0.1] (1.2,0.) -- (1.2,0.2) -- (1.,0.2) -- (1.,0.) -- cycle; 
\draw (0.,0.)-- (1.,2.2);
\draw (0.4171,1.1376) -- (0.5828,1.0623);
\draw (1.,2.2)-- (2.,0.);
\draw (1.5828,1.1376) -- (1.4171,1.0623);
\draw (0.,0.)-- (2.,0.);
\draw (1.,2.2)-- (1.,0.); % segment CD
\draw (1.0910,1.1354) -- (0.9090,1.1354);  %  Mark on CD
\draw (1.0910,1.0646) -- (0.9090,1.0646);  %  Mark on CD
%\draw (0.4292,0.0910) -- (0.4292,-0.0910); % Mark on AD
%\draw (0.5,0.0910) -- (0.5,-0.0910);       % Mark on AD
%\draw (0.5708,0.0910) -- (0.5708,-0.0910); % Mark on AD
%\draw (1.4292,0.0910) -- (1.4292,-0.0910); % Mark on DB
%\draw (1.5,0.0910) -- (1.5,-0.0910);       % Mark on DB
%\draw (1.5708,0.0910) -- (1.5708,-0.0910); % Mark on DB
\begin{scriptsize}
\draw [fill=qqqqff] (0.,0.) circle (1.2pt);
\draw[color=qqqqff] (-0.18,-0.13) node {$A$};
\draw [fill=qqqqff] (2.,0.) circle (1.2pt);
\draw[color=qqqqff] (2.18,-0.13) node {$B$};
\draw [fill=qqqqff] (1.,2.2) circle (1.2pt);
\draw[color=qqqqff] (1.14,2.45) node {$C$};
\draw [fill=qqqqff] (1.,0.) circle (1.2pt);
\draw[color=qqqqff] (0.97,-0.2) node {$D$};
\end{scriptsize}
\end{scope}
\end{tikzpicture}
\end{image}

\begin{enumerate}
\item Beginning with the given figure on the left, Elle draws $\overline{CD}$ and marks the figure intending that this new segment is a(n) \wordChoice{\choice{median}\choice{angle bisector}\choice{perpendicular bisector}\choice[correct]{altitude}}.

\item Based on the marked figure, Deja claims that the $\triangle ACD\cong \triangle\answer[format=string]{BCD}$ by \wordChoice{\choice{SAS}\choice{SSS}\choice{SSA}\choice{ASA}\choice[correct]{HL}}. 

\item Finally, Deja concludes that $\angle A\cong \angle\answer[format=string]{B}$, as they are corresponding parts of congruent triangles. 
\end{enumerate}

\end{problem}


\begin{problem}
Simon and Taylor are trying to prove that the base angles of an isosceles triangle are congruent.   

\begin{image}
\definecolor{qqqqff}{rgb}{0.,0.,1.}
\definecolor{qqwuqq}{rgb}{0.,0.39215,0.}
\begin{tikzpicture}[line width=0.8pt,line cap=round,line join=round,>=triangle 45,x=1.0cm,y=1.0cm]
%\clip(-0.4,-0.4) rectangle (2.4,2.6);
\clip(-0.4,-0.75) rectangle (8.4,2.6);
%\draw [shift={(1.,2.2)},color=qqwuqq,fill=qqwuqq,fill opacity=0.1] (0,0) -- (-114.444:0.404) arc (-114.444:-65.556:0.404) -- cycle;  % Mark angle C
%\draw [shift={(1.,2.2)},color=qqwuqq,fill=qqwuqq,fill opacity=0.1] (0,0) -- (-114.444:0.55) arc (-114.444:-90.:0.55) -- cycle;  % Mark angle DCA
%\draw [shift={(1.,2.2)},color=qqwuqq,fill=qqwuqq,fill opacity=0.1] (0,0) -- (-90.:0.48) arc (-90.:-65.556:0.48) -- cycle;  % Mark angle DCB
%\draw[line width=0.8pt,color=qqwuqq,fill=qqwuqq,fill opacity=0.1] (1.,0.2) -- (0.8,0.2) -- (0.8,0.) -- (1.,0.) -- cycle; 
%\draw[line width=0.8pt,color=qqwuqq,fill=qqwuqq,fill opacity=0.1] (1.2,0.) -- (1.2,0.2) -- (1.,0.2) -- (1.,0.) -- cycle; 
\draw (0.,0.)-- (1.,2.2);
\draw (0.4171,1.1376) -- (0.5828,1.0623);
\draw (1.,2.2)-- (2.,0.);
\draw (1.5828,1.1376) -- (1.4171,1.0623);
\draw (0.,0.)-- (2.,0.);
%\draw (1.,2.2)-- (1.,0.); % segment CD
%\draw (1.0910,1.1354) -- (0.9090,1.1354);  %  Mark on CD
%\draw (1.0910,1.0646) -- (0.9090,1.0646);  %  Mark on CD
%\draw (0.4292,0.0910) -- (0.4292,-0.0910); % Mark on AD
%\draw (0.5,0.0910) -- (0.5,-0.0910);       % Mark on AD
%\draw (0.5708,0.0910) -- (0.5708,-0.0910); % Mark on AD
%\draw (1.4292,0.0910) -- (1.4292,-0.0910); % Mark on DB
%\draw (1.5,0.0910) -- (1.5,-0.0910);       % Mark on DB
%\draw (1.5708,0.0910) -- (1.5708,-0.0910); % Mark on DB
\begin{scriptsize}
\draw [fill=qqqqff] (0.,0.) circle (1.2pt);
\draw[color=qqqqff] (-0.18,-0.13) node {$A$};
\draw [fill=qqqqff] (2.,0.) circle (1.2pt);
\draw[color=qqqqff] (2.18,-0.13) node {$B$};
\draw [fill=qqqqff] (1.,2.2) circle (1.2pt);
\draw[color=qqqqff] (1.14,2.45) node {$C$};
\draw (1,-0.6) node[align=center] {Given figure};
%\draw [fill=qqqqff] (1.,0.) circle (1.2pt);
%\draw[color=qqqqff] (0.97,-0.2) node {$D$};
\end{scriptsize}
%\end{tikzpicture}

\begin{scope}[shift={(3,0)}]
% Isosceles triangle with overspecified perpendicular bisector
%\definecolor{qqqqff}{rgb}{0.,0.,1.}
%\definecolor{qqwuqq}{rgb}{0.,0.39215,0.}
%\begin{tikzpicture}[line width=0.8pt,line cap=round,line join=round,>=triangle 45,x=1.0cm,y=1.0cm]
%\clip(-0.4,-0.4) rectangle (2.4,2.6);
%\draw [shift={(1.,2.2)},color=qqwuqq,fill=qqwuqq,fill opacity=0.1] (0,0) -- (-114.444:0.404) arc (-114.444:-65.556:0.404) -- cycle;  % Mark angle C
%\draw [shift={(1.,2.2)},color=qqwuqq,fill=qqwuqq,fill opacity=0.1] (0,0) -- (-114.444:0.55) arc (-114.444:-90.:0.55) -- cycle;  % Mark angle DCA
%\draw [shift={(1.,2.2)},color=qqwuqq,fill=qqwuqq,fill opacity=0.1] (0,0) -- (-90.:0.48) arc (-90.:-65.556:0.48) -- cycle;  % Mark angle DCB
\draw[line width=0.8pt,color=qqwuqq,fill=qqwuqq,fill opacity=0.1] (1.,0.2) -- (0.8,0.2) -- (0.8,0.) -- (1.,0.) -- cycle; 
\draw[line width=0.8pt,color=qqwuqq,fill=qqwuqq,fill opacity=0.1] (1.2,0.) -- (1.2,0.2) -- (1.,0.2) -- (1.,0.) -- cycle; 
\draw (0.,0.)-- (1.,2.2);
\draw (0.4171,1.1376) -- (0.5828,1.0623);
\draw (1.,2.2)-- (2.,0.);
\draw (1.5828,1.1376) -- (1.4171,1.0623);
\draw (0.,0.)-- (2.,0.);
\draw (1.,2.2)-- (1.,0.); % segment CD
\draw (1.0910,1.1354) -- (0.9090,1.1354);  %  Mark on CD
\draw (1.0910,1.0646) -- (0.9090,1.0646);  %  Mark on CD
\draw (0.4292,0.0910) -- (0.4292,-0.0910); % Mark on AD
\draw (0.5,0.0910) -- (0.5,-0.0910);       % Mark on AD
\draw (0.5708,0.0910) -- (0.5708,-0.0910); % Mark on AD
\draw (1.4292,0.0910) -- (1.4292,-0.0910); % Mark on DB
\draw (1.5,0.0910) -- (1.5,-0.0910);       % Mark on DB
\draw (1.5708,0.0910) -- (1.5708,-0.0910); % Mark on DB
\begin{scriptsize}
\draw [fill=qqqqff] (0.,0.) circle (1.2pt);
\draw[color=qqqqff] (-0.18,-0.13) node {$A$};
\draw [fill=qqqqff] (2.,0.) circle (1.2pt);
\draw[color=qqqqff] (2.18,-0.13) node {$B$};
\draw [fill=qqqqff] (1.,2.2) circle (1.2pt);
\draw[color=qqqqff] (1.14,2.45) node {$C$};
\draw [fill=qqqqff] (1.,0.) circle (1.2pt);
\draw[color=qqqqff] (0.97,-0.2) node {$D$};
\draw (1,-0.6) node[align=center] {Simon's figure};
\end{scriptsize}
\end{scope}
%\end{tikzpicture}

\begin{scope}[shift={(6,0)}]
% Isosceles triangle with a perpendicular bisector that misses vertex
%\definecolor{qqqqff}{rgb}{0.,0.,1.}
%\definecolor{qqwuqq}{rgb}{0.,0.39215,0.}
%\begin{tikzpicture}[line width=0.8pt,line cap=round,line join=round,>=triangle 45,x=1.0cm,y=1.0cm]
%\clip(-0.4,-0.4) rectangle (2.4,2.6);
%\draw [shift={(1.,2.2)},color=qqwuqq,fill=qqwuqq,fill opacity=0.1] (0,0) -- (-114.444:0.404) arc (-114.444:-65.556:0.404) -- cycle;  % Mark angle C
%\draw [shift={(1.,2.2)},color=qqwuqq,fill=qqwuqq,fill opacity=0.1] (0,0) -- (-114.444:0.55) arc (-114.444:-90.:0.55) -- cycle;  % Mark angle DCA
%\draw [shift={(1.,2.2)},color=qqwuqq,fill=qqwuqq,fill opacity=0.1] (0,0) -- (-90.:0.48) arc (-90.:-65.556:0.48) -- cycle;  % Mark angle DCB
\draw[line width=0.8pt,color=qqwuqq,fill=qqwuqq,fill opacity=0.1] (1.,0.2) -- (0.8,0.2) -- (0.8,0.) -- (1.,0.) -- cycle; 
\draw[line width=0.8pt,color=qqwuqq,fill=qqwuqq,fill opacity=0.1] (1.2,0.) -- (1.2,0.2) -- (1.,0.2) -- (1.,0.) -- cycle; 
\draw (0.,0.)-- (1.06,2.2);
\draw (0.4471,1.1376) -- (0.6128,1.0623);
\draw (1.06,2.2)-- (2.,0.);
\draw (1.6128,1.1376) -- (1.4471,1.0623);
\draw (0.,0.)-- (2.,0.);
\draw (0.99,2.4)-- (1.,0.); % segment CD
%\draw (1.0910,1.1354) -- (0.9090,1.1354);  %  Mark on CD
%\draw (1.0910,1.0646) -- (0.9090,1.0646);  %  Mark on CD
\draw (0.4292,0.0910) -- (0.4292,-0.0910); % Mark on AD
\draw (0.5,0.0910) -- (0.5,-0.0910);       % Mark on AD
\draw (0.5708,0.0910) -- (0.5708,-0.0910); % Mark on AD
\draw (1.4292,0.0910) -- (1.4292,-0.0910); % Mark on DB
\draw (1.5,0.0910) -- (1.5,-0.0910);       % Mark on DB
\draw (1.5708,0.0910) -- (1.5708,-0.0910); % Mark on DB
\begin{scriptsize}
\draw [fill=qqqqff] (0.,0.) circle (1.2pt);
\draw[color=qqqqff] (-0.18,-0.13) node {$A$};
\draw [fill=qqqqff] (2.,0.) circle (1.2pt);
\draw[color=qqqqff] (2.18,-0.13) node {$B$};
\draw [fill=qqqqff] (1.06,2.2) circle (1.2pt);
\draw[color=qqqqff] (1.24,2.45) node {$C$};
\draw [fill=qqqqff] (1.,0.) circle (1.2pt);
\draw[color=qqqqff] (0.97,-0.2) node {$D$};
\draw (1,-0.6) node[align=center] {Taylor's figure};
\end{scriptsize}
\end{scope}
\end{tikzpicture}
\end{image}

Beginning with the given figure on the left, Simon draws $\overline{CD}$ and marks the second figure intending that this new segment is a perpendicular bisector of $\overline{AB}$.

Taylor claims that a perpendicular bisector of a side of a triangle usually misses the opposite vertex.  So without using other properties of isosceles triangles or perpendicular bisectors, the figure should allow for that possibility.  

\fixnote{Taylor's claim needs some work, as is the case for the choices below.}

Choose the best response to their argument: 
\begin{multipleChoice}
\choice{Simon is correct, and $\triangle ACD\cong \triangle BCD$ by SAS.} 
\choice{Simon is correct, and $\triangle ACD\cong \triangle BCD$ by SSS}
\choice[correct]{Taylor is correct, and the perpendicular bisector cannot be used to complete this proof.}
\choice{Neither of them are correct.}  
\end{multipleChoice} 
\end{problem}


\begin{problem}
Prove that the base angles of an isosceles triangle are congruent.   

\begin{image}
\definecolor{qqqqff}{rgb}{0.,0.,1.}
\definecolor{qqwuqq}{rgb}{0.,0.39215,0.}
\begin{tikzpicture}[line width=0.8pt,line cap=round,line join=round,>=triangle 45,x=1.0cm,y=1.0cm]
%\clip(-0.4,-0.4) rectangle (2.4,2.6);
\clip(-0.4,-0.75) rectangle (5.4,2.6);
%\draw [shift={(1.,2.2)},color=qqwuqq,fill=qqwuqq,fill opacity=0.1] (0,0) -- (-114.444:0.404) arc (-114.444:-65.556:0.404) -- cycle;  % Mark angle C
%\draw [shift={(1.,2.2)},color=qqwuqq,fill=qqwuqq,fill opacity=0.1] (0,0) -- (-114.444:0.55) arc (-114.444:-90.:0.55) -- cycle;  % Mark angle DCA
%\draw [shift={(1.,2.2)},color=qqwuqq,fill=qqwuqq,fill opacity=0.1] (0,0) -- (-90.:0.48) arc (-90.:-65.556:0.48) -- cycle;  % Mark angle DCB
%\draw[line width=0.8pt,color=qqwuqq,fill=qqwuqq,fill opacity=0.1] (1.,0.2) -- (0.8,0.2) -- (0.8,0.) -- (1.,0.) -- cycle; 
%\draw[line width=0.8pt,color=qqwuqq,fill=qqwuqq,fill opacity=0.1] (1.2,0.) -- (1.2,0.2) -- (1.,0.2) -- (1.,0.) -- cycle; 
\draw (0.,0.)-- (1.,2.2);
\draw (0.4171,1.1376) -- (0.5828,1.0623);
\draw (1.,2.2)-- (2.,0.);
\draw (1.5828,1.1376) -- (1.4171,1.0623);
\draw (0.,0.)-- (2.,0.);
%\draw (1.,2.2)-- (1.,0.); % segment CD
%\draw (1.0910,1.1354) -- (0.9090,1.1354);  %  Mark on CD
%\draw (1.0910,1.0646) -- (0.9090,1.0646);  %  Mark on CD
%\draw (0.4292,0.0910) -- (0.4292,-0.0910); % Mark on AD
%\draw (0.5,0.0910) -- (0.5,-0.0910);       % Mark on AD
%\draw (0.5708,0.0910) -- (0.5708,-0.0910); % Mark on AD
%\draw (1.4292,0.0910) -- (1.4292,-0.0910); % Mark on DB
%\draw (1.5,0.0910) -- (1.5,-0.0910);       % Mark on DB
%\draw (1.5708,0.0910) -- (1.5708,-0.0910); % Mark on DB
\begin{scriptsize}
\draw [fill=qqqqff] (0.,0.) circle (1.2pt);
\draw[color=qqqqff] (-0.18,-0.13) node {$A$};
\draw [fill=qqqqff] (2.,0.) circle (1.2pt);
\draw[color=qqqqff] (2.18,-0.13) node {$B$};
\draw [fill=qqqqff] (1.,2.2) circle (1.2pt);
\draw[color=qqqqff] (1.14,2.45) node {$C$};
\draw (1,-0.6) node[align=center] {Given figure};
%\draw [fill=qqqqff] (1.,0.) circle (1.2pt);
%\draw[color=qqqqff] (0.97,-0.2) node {$D$};
\end{scriptsize}
%\end{tikzpicture}

\begin{scope}[shift={(3,0)}]
% Isosceles triangle using symmetry 
%\definecolor{qqqqff}{rgb}{0.,0.,1.}
%\definecolor{qqwuqq}{rgb}{0.,0.39215,0.}
%\begin{tikzpicture}[line width=0.8pt,line cap=round,line join=round,>=triangle 45,x=1.0cm,y=1.0cm]
%\clip(-0.4,-0.4) rectangle (2.4,2.6);
\draw [shift={(1.,2.2)},color=qqwuqq,fill=qqwuqq,fill opacity=0.1] (0,0) -- (-114.444:0.404) arc (-114.444:-65.556:0.404) -- cycle;  % Mark angle C
%\draw [shift={(1.,2.2)},color=qqwuqq,fill=qqwuqq,fill opacity=0.1] (0,0) -- (-114.444:0.55) arc (-114.444:-90.:0.55) -- cycle;  % Mark angle DCA
%\draw [shift={(1.,2.2)},color=qqwuqq,fill=qqwuqq,fill opacity=0.1] (0,0) -- (-90.:0.48) arc (-90.:-65.556:0.48) -- cycle;  % Mark angle DCB
%\draw[line width=0.8pt,color=qqwuqq,fill=qqwuqq,fill opacity=0.1] (1.,0.2) -- (0.8,0.2) -- (0.8,0.) -- (1.,0.) -- cycle; 
%\draw[line width=0.8pt,color=qqwuqq,fill=qqwuqq,fill opacity=0.1] (1.2,0.) -- (1.2,0.2) -- (1.,0.2) -- (1.,0.) -- cycle; 
\draw (0.,0.)-- (1.,2.2);
\draw (0.4171,1.1376) -- (0.5828,1.0623);
\draw (1.,2.2)-- (2.,0.);
\draw (1.5828,1.1376) -- (1.4171,1.0623);
\draw (0.,0.)-- (2.,0.);
%\draw (1.,2.2)-- (1.,0.); % segment CD
%\draw (1.0910,1.1354) -- (0.9090,1.1354);  %  Mark on CD
%\draw (1.0910,1.0646) -- (0.9090,1.0646);  %  Mark on CD
%\draw (0.4292,0.0910) -- (0.4292,-0.0910); % Mark on AD
%\draw (0.5,0.0910) -- (0.5,-0.0910);       % Mark on AD
%\draw (0.5708,0.0910) -- (0.5708,-0.0910); % Mark on AD
%\draw (1.4292,0.0910) -- (1.4292,-0.0910); % Mark on DB
%\draw (1.5,0.0910) -- (1.5,-0.0910);       % Mark on DB
%\draw (1.5708,0.0910) -- (1.5708,-0.0910); % Mark on DB
\begin{scriptsize}
\draw [fill=qqqqff] (0.,0.) circle (1.2pt);
\draw[color=qqqqff] (-0.18,-0.13) node {$A$};
\draw [fill=qqqqff] (2.,0.) circle (1.2pt);
\draw[color=qqqqff] (2.18,-0.13) node {$B$};
\draw [fill=qqqqff] (1.,2.2) circle (1.2pt);
\draw[color=qqqqff] (1.14,2.45) node {$C$};
%\draw [fill=qqqqff] (1.,0.) circle (1.2pt);
%\draw[color=qqqqff] (0.97,-0.2) node {$D$};
\end{scriptsize}
\end{scope}
\end{tikzpicture}
\end{image}

\begin{enumerate}
\item Examining the given figure on the left, Lissy notices symmetry in the triangle and claims that the triangle is congruent to itself by a \wordChoice{\choice{translation}\choice[correct]{reflection}\choice{rotation}}.  

\item Based on the marked figure, Lissy claims that the $\triangle ACB\cong \triangle\answer[format=string]{BCA}$ by \wordChoice{\choice[correct]{SAS}\choice{SSS}\choice{SSA}\choice{ASA}\choice{HL}}. 

\item Finally, Lissy concludes that $\angle A\cong \angle\answer[format=string]{B}$, as they are corresponding parts of congruent triangles. 
\end{enumerate}

\end{problem}



\end{document}
