\documentclass[nooutcomes]{ximera}
%\documentclass[space,handout,nooutcomes]{ximera}

\usepackage{epsfig}

\graphicspath{
  {./}
  {figures/}
  {../laode}
  {../laode/figures}
}

\usepackage{epstopdf}
\epstopdfsetup{outdir=./}

\usepackage{morewrites}
\makeatletter
\newcommand\subfile[1]{%
\renewcommand{\input}[1]{}%
\begingroup\skip@preamble\otherinput{#1}\endgroup\par\vspace{\topsep}
\let\input\otherinput}
\makeatother

\newcommand{\EXER}{}
\newcommand{\includeexercises}{\EXER\directlua{dofile(kpse.find_file("exercises","lua"))}}

\newenvironment{computerExercise}{\begin{exercise}}{\end{exercise}}

%\newcounter{ccounter}
%\setcounter{ccounter}{1}
%\newcommand{\Chapter}[1]{\setcounter{chapter}{\arabic{ccounter}}\chapter{#1}\addtocounter{ccounter}{1}}

%\newcommand{\section}[1]{\section{#1}\setcounter{thm}{0}\setcounter{equation}{0}}

%\renewcommand{\theequation}{\arabic{chapter}.\arabic{section}.\arabic{equation}}
%\renewcommand{\thefigure}{\arabic{chapter}.\arabic{figure}}
%\renewcommand{\thetable}{\arabic{chapter}.\arabic{table}}

%\newcommand{\Sec}[2]{\section{#1}\markright{\arabic{ccounter}.\arabic{section}.#2}\setcounter{equation}{0}\setcounter{thm}{0}\setcounter{figure}{0}}
  
\newcommand{\Sec}[2]{\section{#1}}

\setcounter{secnumdepth}{2}
%\setcounter{secnumdepth}{1} 

%\newcounter{THM}
%\renewcommand{\theTHM}{\arabic{chapter}.\arabic{section}}

\newcommand{\trademark}{{R\!\!\!\!\!\bigcirc}}
%\newtheorem{exercise}{}

\newcommand{\dfield}{{\sf dfield9}}
\newcommand{\pplane}{{\sf pplane9}}
\newcommand{\PPLANE}{{\sf PPLANE9}}

% BADBAD: \newcommand{\Bbb}{\bf}

\newcommand{\R}{\mbox{$\Bbb{R}$}}
\newcommand{\C}{\mbox{$\Bbb{C}$}}
\newcommand{\Z}{\mbox{$\Bbb{Z}$}}
\newcommand{\N}{\mbox{$\Bbb{N}$}}
\newcommand{\D}{\mbox{{\bf D}}}
\usepackage{amssymb}
%\newcommand{\qed}{\hfill\mbox{\raggedright$\square$} \vspace{1ex}}
%\newcommand{\proof}{\noindent {\bf Proof:} \hspace{0.1in}}

\newcommand{\setmin}{\;\mbox{--}\;}
\newcommand{\Matlab}{{M\small{AT\-LAB}} }
\newcommand{\Matlabp}{{M\small{AT\-LAB}}}
\newcommand{\computer}{\Matlab Instructions}
\newcommand{\half}{\mbox{$\frac{1}{2}$}}
\newcommand{\compose}{\raisebox{.15ex}{\mbox{{\scriptsize$\circ$}}}}
\newcommand{\AND}{\quad\mbox{and}\quad}
\newcommand{\vect}[2]{\left(\begin{array}{c} #1_1 \\ \vdots \\
 #1_{#2}\end{array}\right)}
\newcommand{\mattwo}[4]{\left(\begin{array}{rr} #1 & #2\\ #3
&#4\end{array}\right)}
\newcommand{\mattwoc}[4]{\left(\begin{array}{cc} #1 & #2\\ #3
&#4\end{array}\right)}
\newcommand{\vectwo}[2]{\left(\begin{array}{r} #1 \\ #2\end{array}\right)}
\newcommand{\vectwoc}[2]{\left(\begin{array}{c} #1 \\ #2\end{array}\right)}

\newcommand{\ignore}[1]{}


\newcommand{\inv}{^{-1}}
\newcommand{\CC}{{\cal C}}
\newcommand{\CCone}{\CC^1}
\newcommand{\Span}{{\rm span}}
\newcommand{\rank}{{\rm rank}}
\newcommand{\trace}{{\rm tr}}
\newcommand{\RE}{{\rm Re}}
\newcommand{\IM}{{\rm Im}}
\newcommand{\nulls}{{\rm null\;space}}

\newcommand{\dps}{\displaystyle}
\newcommand{\arraystart}{\renewcommand{\arraystretch}{1.8}}
\newcommand{\arrayfinish}{\renewcommand{\arraystretch}{1.2}}
\newcommand{\Start}[1]{\vspace{0.08in}\noindent {\bf Section~\ref{#1}}}
\newcommand{\exer}[1]{\noindent {\bf \ref{#1}}}
\newcommand{\ans}{\textbf{Answer:} }
\newcommand{\matthree}[9]{\left(\begin{array}{rrr} #1 & #2 & #3 \\ #4 & #5 & #6
\\ #7 & #8 & #9\end{array}\right)}
\newcommand{\cvectwo}[2]{\left(\begin{array}{c} #1 \\ #2\end{array}\right)}
\newcommand{\cmatthree}[9]{\left(\begin{array}{ccc} #1 & #2 & #3 \\ #4 & #5 &
#6 \\ #7 & #8 & #9\end{array}\right)}
\newcommand{\vecthree}[3]{\left(\begin{array}{r} #1 \\ #2 \\
#3\end{array}\right)}
\newcommand{\cvecthree}[3]{\left(\begin{array}{c} #1 \\ #2 \\
#3\end{array}\right)}
\newcommand{\cmattwo}[4]{\left(\begin{array}{cc} #1 & #2\\ #3
&#4\end{array}\right)}

\newcommand{\Matrix}[1]{\ensuremath{\left(\begin{array}{rrrrrrrrrrrrrrrrrr} #1 \end{array}\right)}}

\newcommand{\Matrixc}[1]{\ensuremath{\left(\begin{array}{cccccccccccc} #1 \end{array}\right)}}



\renewcommand{\labelenumi}{\theenumi}
\newenvironment{enumeratea}%
{\begingroup
 \renewcommand{\theenumi}{\alph{enumi}}
 \renewcommand{\labelenumi}{(\theenumi)}
 \begin{enumerate}}
 {\end{enumerate}\endgroup}

\newcounter{help}
\renewcommand{\thehelp}{\thesection.\arabic{equation}}

%\newenvironment{equation*}%
%{\renewcommand\endequation{\eqno (\theequation)* $$}%
%   \begin{equation}}%
%   {\end{equation}\renewcommand\endequation{\eqno \@eqnnum
%$$\global\@ignoretrue}}

%\input{psfig.tex}

\author{Martin Golubitsky and Michael Dellnitz}

%\newenvironment{matlabEquation}%
%{\renewcommand\endequation{\eqno (\theequation*) $$}%
%   \begin{equation}}%
%   {\end{equation}\renewcommand\endequation{\eqno \@eqnnum
% $$\global\@ignoretrue}}

\newcommand{\soln}{\textbf{Solution:} }
\newcommand{\exercap}[1]{\centerline{Figure~\ref{#1}}}
\newcommand{\exercaptwo}[1]{\centerline{Figure~\ref{#1}a\hspace{2.1in}
Figure~\ref{#1}b}}
\newcommand{\exercapthree}[1]{\centerline{Figure~\ref{#1}a\hspace{1.2in}
Figure~\ref{#1}b\hspace{1.2in}Figure~\ref{#1}c}}
\newcommand{\para}{\hspace{0.4in}}

\usepackage{ifluatex}
\ifluatex
\ifcsname displaysolutions\endcsname%
\else
\renewenvironment{solution}{\suppress}{\endsuppress}
\fi
\else
\renewenvironment{solution}{}{}
\fi

%\ifxake
%\newenvironment{matlabEquation}{\begin{equation}}{\end{equation}}
%\else
\newenvironment{matlabEquation}%
{\let\oldtheequation\theequation\renewcommand{\theequation}{\oldtheequation*}\begin{equation}}%
  {\end{equation}\let\theequation\oldtheequation}
%\fi

\makeatother



\title{Inscribed Angles}
\author{Brad Findell}
\begin{document}
\begin{abstract}
Proofs updated. 
\end{abstract}
\maketitle

\begin{problem}
In the figure below, $\overline{AC}$ is a diameter of a circle with center $P$. Prove that $\angle ABC$ is a right angle.  

\begin{image}
\definecolor{qqwuqq}{rgb}{0.,0.3926,0.}
\definecolor{uuuuuu}{rgb}{0.2667,0.2667,0.2667}
\definecolor{qqqqff}{rgb}{0.,0.,1.}
\begin{tikzpicture}[line cap=round,line join=round,>=triangle 45,x=1.0cm,y=1.0cm]
\clip(-4.5,-3.8) rectangle (13.5,3.8);
%\draw [shift={(-3.6056,0.)},line width=0.8pt,color=qqwuqq,fill=qqwuqq,fill opacity=0.1] (0,0) -- (0.:1.115) arc (0.:28.155:1.115) -- cycle;
%\draw [shift={(2.,3.)},line width=0.8pt,color=qqwuqq,fill=qqwuqq,fill opacity=0.1] (0,0) -- (-151.845:1.115) arc (-151.845:-123.690:1.115) -- cycle;
%\draw [shift={(2.,3.)},line width=0.8pt,color=qqwuqq,fill=qqwuqq,fill opacity=0.1] (0,0) -- (-123.690:0.892) arc (-123.690:-61.845:0.892) -- cycle;
%\draw [shift={(2.,3.)},line width=0.8pt,color=qqwuqq] (-123.690:0.669) arc (-123.690:-61.85:0.669);
%\draw [shift={(3.6056,0.)},line width=0.8pt,color=qqwuqq,fill=qqwuqq,fill opacity=0.1] (0,0) -- (118.155:0.892) arc (118.155:180.:0.892) -- cycle;
%\draw [shift={(3.6056,0.)},line width=0.8pt,color=qqwuqq] (118.155:0.669) arc (118.155:180.:0.669);
\draw [line width=0.8pt] (0.,0.) circle (3.606cm);
\draw [line width=0.8pt] (0.,0.) circle (3.606cm);
\draw [line width=0.8pt] (-3.6056,0.)-- (2.,3.);
\draw [line width=0.8pt] (2.,3.)-- (3.6056,0.);
\draw [line width=0.8pt] (-3.6056,0.)-- (0.,0.);
%\draw [line width=0.8pt] (-1.8028,0.2006) -- (-1.8028,-0.2006);
\draw [line width=0.8pt] (0.,0.)-- (3.6056,0.);
%\draw [line width=0.8pt] (1.8028,0.2006) -- (1.8028,-0.2006);
%\draw [line width=0.8pt] (0.,0.)-- (2.,3.);
%\draw [line width=0.8pt] (0.833,1.611) -- (1.167,1.389);
%\draw (-2.2,0.35) node {$x$};
%\draw (0.95,2.1) node {$x$};
%\draw (1.95,1.85) node {$y$};
%\draw (2.6,0.6) node {$y$};
%\draw (0.55,0.3) node {$z$};
%\begin{scriptsize}
\draw [fill=qqqqff] (0.,0.) circle (1.2pt);
\draw[color=qqqqff] (-0.24,-0.28) node {$P$};
\draw [fill=qqqqff] (2.,3.) circle (1.2pt);
\draw[color=qqqqff] (2.25,3.3) node {$B$};
\draw [fill=qqqqff] (-3.60,0.) circle (1.2pt);
\draw[color=qqqqff] (-3.9,0) node {$A$};
\draw [fill=qqqqff] (3.60,0.) circle (1.2pt);
\draw[color=qqqqff] (3.90,0) node {$C$};
%\end{scriptsize}
%\end{tikzpicture}

\begin{scope}[shift={(9,0)}]
%\definecolor{qqwuqq}{rgb}{0.,0.3926,0.}
%\definecolor{qqqqff}{rgb}{0.,0.,1.}
%\begin{tikzpicture}[line cap=round,line join=round,>=triangle 45,x=1.0cm,y=1.0cm]
%\clip(-4.5,-3.8) rectangle (4.5,3.8);
%\draw [shift={(-3.6056,0.)},line width=0.8pt,color=qqwuqq,fill=qqwuqq,fill opacity=0.1] (0,0) -- (0.:1.115) arc (0.:28.155:1.115) -- cycle;
%\draw [shift={(2.,3.)},line width=0.8pt,color=qqwuqq,fill=qqwuqq,fill opacity=0.1] (0,0) -- (-151.845:1.115) arc (-151.845:-123.690:1.115) -- cycle;
%\draw [shift={(2.,3.)},line width=0.8pt,color=qqwuqq,fill=qqwuqq,fill opacity=0.1] (0,0) -- (-123.690:0.892) arc (-123.690:-61.845:0.892) -- cycle;
%\draw [shift={(2.,3.)},line width=0.8pt,color=qqwuqq] (-123.690:0.669) arc (-123.690:-61.85:0.669);
%\draw [shift={(3.6056,0.)},line width=0.8pt,color=qqwuqq,fill=qqwuqq,fill opacity=0.1] (0,0) -- (118.155:0.892) arc (118.155:180.:0.892) -- cycle;
%\draw [shift={(3.6056,0.)},line width=0.8pt,color=qqwuqq] (118.155:0.669) arc (118.155:180.:0.669);
\draw [line width=0.8pt] (0.,0.) circle (3.606cm);
\draw [line width=0.8pt] (-3.6056,0.)-- (2.,3.);
\draw [line width=0.8pt] (2.,3.)-- (3.6056,0.);
\draw [line width=0.8pt] (-3.6056,0.)-- (0.,0.);
\draw [line width=0.8pt] (-1.8028,0.2006) -- (-1.8028,-0.2006);
\draw [line width=0.8pt] (0.,0.)-- (3.6056,0.);
\draw [line width=0.8pt] (1.8028,0.2006) -- (1.8028,-0.2006);
\draw [line width=0.8pt] (0.,0.)-- (2.,3.);
\draw [line width=0.8pt] (0.833,1.611) -- (1.167,1.389);
%\draw (-2.2,0.35) node {$x$};
%\draw (0.95,2.1) node {$x$};
%\draw (1.95,1.85) node {$y$};
%\draw (2.6,0.6) node {$y$};
%\draw (0.55,0.3) node {$z$};
%\begin{scriptsize}
\draw [fill=qqqqff] (0.,0.) circle (1.2pt);
\draw[color=qqqqff] (-0.24,-0.28) node {$P$};
\draw [fill=qqqqff] (2.,3.) circle (1.2pt);
\draw[color=qqqqff] (2.25,3.3) node {$B$};
\draw [fill=qqqqff] (-3.60,0.) circle (1.2pt);
\draw[color=qqqqff] (-3.9,0) node {$A$};
\draw [fill=qqqqff] (3.60,0.) circle (1.2pt);
\draw[color=qqqqff] (3.90,0) node {$C$};
%\end{scriptsize}
\end{scope}
\end{tikzpicture}
\end{image}

\begin{enumerate}
\item Beginning with the diagram on the left, Natalia draws $\overline{PB}$ and marks the diagram to show segments that she knows to be congruent because each one is a $\answer[format=string]{radius}$ of the circle.  

\begin{image}
\definecolor{qqwuqq}{rgb}{0.,0.3926,0.}
\definecolor{qqqqff}{rgb}{0.,0.,1.}
\begin{tikzpicture}[line cap=round,line join=round,>=triangle 45,x=1.0cm,y=1.0cm]
\clip(-4.5,-3.8) rectangle (13.5,3.8);
\draw [shift={(-3.6056,0.)},line width=0.8pt,color=qqwuqq,fill=qqwuqq,fill opacity=0.1] (0,0) -- (0.:1.115) arc (0.:28.155:1.115) -- cycle;
\draw [shift={(2.,3.)},line width=0.8pt,color=qqwuqq,fill=qqwuqq,fill opacity=0.1] (0,0) -- (-151.845:1.115) arc (-151.845:-123.690:1.115) -- cycle;
\draw [shift={(2.,3.)},line width=0.8pt,color=qqwuqq,fill=qqwuqq,fill opacity=0.1] (0,0) -- (-123.690:0.892) arc (-123.690:-61.845:0.892) -- cycle;
\draw [shift={(2.,3.)},line width=0.8pt,color=qqwuqq] (-123.690:0.669) arc (-123.690:-61.85:0.669);
\draw [shift={(3.6056,0.)},line width=0.8pt,color=qqwuqq,fill=qqwuqq,fill opacity=0.1] (0,0) -- (118.155:0.892) arc (118.155:180.:0.892) -- cycle;
\draw [shift={(3.6056,0.)},line width=0.8pt,color=qqwuqq] (118.155:0.669) arc (118.155:180.:0.669);
\draw [line width=0.8pt] (0.,0.) circle (3.606cm);
\draw [line width=0.8pt] (-3.6056,0.)-- (2.,3.);
\draw [line width=0.8pt] (2.,3.)-- (3.6056,0.);
\draw [line width=0.8pt] (-3.6056,0.)-- (0.,0.);
\draw [line width=0.8pt] (-1.8028,0.2006) -- (-1.8028,-0.2006);
\draw [line width=0.8pt] (0.,0.)-- (3.6056,0.);
\draw [line width=0.8pt] (1.8028,0.2006) -- (1.8028,-0.2006);
\draw [line width=0.8pt] (0.,0.)-- (2.,3.);
\draw [line width=0.8pt] (0.833,1.611) -- (1.167,1.389);
%\draw (-2.2,0.35) node {$x$};
%\draw (0.95,2.1) node {$x$};
%\draw (1.95,1.85) node {$y$};
%\draw (2.6,0.6) node {$y$};
%\draw (0.55,0.3) node {$z$};
%\begin{scriptsize}
\draw [fill=qqqqff] (0.,0.) circle (1.2pt);
\draw[color=qqqqff] (-0.24,-0.28) node {$P$};
\draw [fill=qqqqff] (2.,3.) circle (1.2pt);
\draw[color=qqqqff] (2.25,3.3) node {$B$};
\draw [fill=qqqqff] (-3.60,0.) circle (1.2pt);
\draw[color=qqqqff] (-3.9,0) node {$A$};
\draw [fill=qqqqff] (3.60,0.) circle (1.2pt);
\draw[color=qqqqff] (3.90,0) node {$C$};
%\end{scriptsize}
%\end{tikzpicture}

\begin{scope}[shift={(9,0)}]
%\definecolor{qqwuqq}{rgb}{0.,0.3926,0.}
%\definecolor{qqqqff}{rgb}{0.,0.,1.}
%\begin{tikzpicture}[line cap=round,line join=round,>=triangle 45,x=1.0cm,y=1.0cm]
%\clip(-4.5,-3.8) rectangle (4.5,3.8);
\draw [shift={(-3.6056,0.)},line width=0.8pt,color=qqwuqq,fill=qqwuqq,fill opacity=0.1] (0,0) -- (0.:1.115) arc (0.:28.155:1.115) -- cycle;
\draw [shift={(2.,3.)},line width=0.8pt,color=qqwuqq,fill=qqwuqq,fill opacity=0.1] (0,0) -- (-151.845:1.115) arc (-151.845:-123.690:1.115) -- cycle;
\draw [shift={(2.,3.)},line width=0.8pt,color=qqwuqq,fill=qqwuqq,fill opacity=0.1] (0,0) -- (-123.690:0.892) arc (-123.690:-61.845:0.892) -- cycle;
\draw [shift={(2.,3.)},line width=0.8pt,color=qqwuqq] (-123.690:0.669) arc (-123.690:-61.85:0.669);
\draw [shift={(3.6056,0.)},line width=0.8pt,color=qqwuqq,fill=qqwuqq,fill opacity=0.1] (0,0) -- (118.155:0.892) arc (118.155:180.:0.892) -- cycle;
\draw [shift={(3.6056,0.)},line width=0.8pt,color=qqwuqq] (118.155:0.669) arc (118.155:180.:0.669);
\draw [line width=0.8pt] (0.,0.) circle (3.606cm);
\draw [line width=0.8pt] (-3.6056,0.)-- (2.,3.);
\draw [line width=0.8pt] (2.,3.)-- (3.6056,0.);
\draw [line width=0.8pt] (-3.6056,0.)-- (0.,0.);
\draw [line width=0.8pt] (-1.8028,0.2006) -- (-1.8028,-0.2006);
\draw [line width=0.8pt] (0.,0.)-- (3.6056,0.);
\draw [line width=0.8pt] (1.8028,0.2006) -- (1.8028,-0.2006);
\draw [line width=0.8pt] (0.,0.)-- (2.,3.);
\draw [line width=0.8pt] (0.833,1.611) -- (1.167,1.389);
\draw (-2.2,0.35) node {$x$};
\draw (0.95,2.1) node {$x$};
\draw (1.95,1.85) node {$y$};
\draw (2.6,0.6) node {$y$};
%\draw (0.55,0.3) node {$z$};
%\begin{scriptsize}
\draw [fill=qqqqff] (0.,0.) circle (1.2pt);
\draw[color=qqqqff] (-0.24,-0.28) node {$P$};
\draw [fill=qqqqff] (2.,3.) circle (1.2pt);
\draw[color=qqqqff] (2.25,3.3) node {$B$};
\draw [fill=qqqqff] (-3.60,0.) circle (1.2pt);
\draw[color=qqqqff] (-3.9,0) node {$A$};
\draw [fill=qqqqff] (3.60,0.) circle (1.2pt);
\draw[color=qqqqff] (3.90,0) node {$C$};
%\end{scriptsize}
\end{scope}
\end{tikzpicture}
\end{image}


\item Natalia sees that $\triangle APB$ and $\triangle BPC$ are $\answer[format=string]{isosceles}$ triangles, so she marks the figure to show angles that must congruent. 
\fixnote{Do we need a statement or citation of the theorem?}

\item In order to do some algebra with these congruent angles, Natalia labels their measures $x$ and $y$, as shown in the picture on the right.  

\item She writes an equation for the sum of the angles of $\triangle ABC$: 

\[
\answer{x+(x+y)+y} = 180^\circ
\]

\fixnote{Need a prompt about dividing the equation by 2.}
  
\item Since $m\angle ABC = \answer{x+y}$, she concludes that $m\angle ABC = 90^\circ$.  

\end{enumerate}
\end{problem}


\begin{problem}
A special case of the relationship between an inscribed angle and the corresponding central angle.

In the figure below, $\overline{AC}$ is a diameter of a circle with center $P$. Prove that $z=2x$.  

\begin{image}
\definecolor{qqwuqq}{rgb}{0.,0.3926,0.}
\definecolor{qqqqff}{rgb}{0.,0.,1.}
\begin{tikzpicture}[line cap=round,line join=round,>=triangle 45,x=1.0cm,y=1.0cm]
\clip(-4.5,-3.8) rectangle (4.5,3.8);
\draw [shift={(-3.6056,0.)},line width=0.8pt,color=qqwuqq,fill=qqwuqq,fill opacity=0.1] (0,0) -- (0.:1.115) arc (0.:28.155:1.115) -- cycle;
\draw [shift={(2.,3.)},line width=0.8pt,color=qqwuqq,fill=qqwuqq,fill opacity=0.1] (0,0) -- (-151.845:1.115) arc (-151.845:-123.690:1.115) -- cycle;
\draw [shift={(2.,3.)},line width=0.8pt,color=qqwuqq,fill=qqwuqq,fill opacity=0.1] (0,0) -- (-123.690:0.892) arc (-123.690:-61.845:0.892) -- cycle;
\draw [shift={(2.,3.)},line width=0.8pt,color=qqwuqq] (-123.690:0.669) arc (-123.690:-61.85:0.669);
\draw [shift={(3.6056,0.)},line width=0.8pt,color=qqwuqq,fill=qqwuqq,fill opacity=0.1] (0,0) -- (118.155:0.892) arc (118.155:180.:0.892) -- cycle;
\draw [shift={(3.6056,0.)},line width=0.8pt,color=qqwuqq] (118.155:0.669) arc (118.155:180.:0.669);
\draw [line width=0.8pt] (0.,0.) circle (3.606cm);
\draw [line width=0.8pt] (-3.6056,0.)-- (2.,3.);
\draw [line width=0.8pt] (2.,3.)-- (3.6056,0.);
\draw [line width=0.8pt] (-3.6056,0.)-- (0.,0.);
\draw [line width=0.8pt] (-1.8028,0.2006) -- (-1.8028,-0.2006);
\draw [line width=0.8pt] (0.,0.)-- (3.6056,0.);
\draw [line width=0.8pt] (1.8028,0.2006) -- (1.8028,-0.2006);
\draw [line width=0.8pt] (0.,0.)-- (2.,3.);
\draw [line width=0.8pt] (0.833,1.611) -- (1.167,1.389);
\draw (-2.2,0.35) node {$x$};
\draw (0.95,2.1) node {$x$};
%\draw (1.95,1.85) node {$y$};
%\draw (2.6,0.6) node {$y$};
\draw (0.55,0.3) node {$z$};
%\begin{scriptsize}
\draw [fill=qqqqff] (0.,0.) circle (1.2pt);
\draw[color=qqqqff] (-0.24,-0.28) node {$P$};
\draw [fill=qqqqff] (2.,3.) circle (1.2pt);
\draw[color=qqqqff] (2.25,3.3) node {$B$};
\draw [fill=qqqqff] (-3.60,0.) circle (1.2pt);
\draw[color=qqqqff] (-3.9,0) node {$A$};
\draw [fill=qqqqff] (3.60,0.) circle (1.2pt);
\draw[color=qqqqff] (3.90,0) node {$C$};
%\end{scriptsize}
\end{tikzpicture}
\end{image}

Because $z$ is the measure of an angle exterior to $\triangle APB$, it is equal to the sum of the measures of the remote interior angles.  In other words $z=2x$.  

Alternatively, without using the exterior angle theorem, one might proceed as follows:
\begin{enumerate}
\item $\angle ABP + x + x = 180^\circ$ because of the angle sum in $\triangle ABP$.
\item $\angle ABP + z = 180^\circ$ because they form a linear pair. 
\item Then $z = 2x$ by comparing the two equations. 
\end{enumerate}

\fixnote{This handles the special case in which one side of the inscribed angle is a diameter.  For the general result, consider two cases:  (1) When the center of the circle is in the interior of the inscribed angle; and (2) When the center of the circle is not in the interior of the inscribed angle.}

\end{problem}

\end{document}
