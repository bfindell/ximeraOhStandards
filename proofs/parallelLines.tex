\documentclass[nooutcomes]{ximera}
%\documentclass[space,handout,nooutcomes]{ximera}

\usepackage{epsfig}

\graphicspath{
  {./}
  {figures/}
  {../laode}
  {../laode/figures}
}

\usepackage{epstopdf}
\epstopdfsetup{outdir=./}

\usepackage{morewrites}
\makeatletter
\newcommand\subfile[1]{%
\renewcommand{\input}[1]{}%
\begingroup\skip@preamble\otherinput{#1}\endgroup\par\vspace{\topsep}
\let\input\otherinput}
\makeatother

\newcommand{\EXER}{}
\newcommand{\includeexercises}{\EXER\directlua{dofile(kpse.find_file("exercises","lua"))}}

\newenvironment{computerExercise}{\begin{exercise}}{\end{exercise}}

%\newcounter{ccounter}
%\setcounter{ccounter}{1}
%\newcommand{\Chapter}[1]{\setcounter{chapter}{\arabic{ccounter}}\chapter{#1}\addtocounter{ccounter}{1}}

%\newcommand{\section}[1]{\section{#1}\setcounter{thm}{0}\setcounter{equation}{0}}

%\renewcommand{\theequation}{\arabic{chapter}.\arabic{section}.\arabic{equation}}
%\renewcommand{\thefigure}{\arabic{chapter}.\arabic{figure}}
%\renewcommand{\thetable}{\arabic{chapter}.\arabic{table}}

%\newcommand{\Sec}[2]{\section{#1}\markright{\arabic{ccounter}.\arabic{section}.#2}\setcounter{equation}{0}\setcounter{thm}{0}\setcounter{figure}{0}}
  
\newcommand{\Sec}[2]{\section{#1}}

\setcounter{secnumdepth}{2}
%\setcounter{secnumdepth}{1} 

%\newcounter{THM}
%\renewcommand{\theTHM}{\arabic{chapter}.\arabic{section}}

\newcommand{\trademark}{{R\!\!\!\!\!\bigcirc}}
%\newtheorem{exercise}{}

\newcommand{\dfield}{{\sf dfield9}}
\newcommand{\pplane}{{\sf pplane9}}
\newcommand{\PPLANE}{{\sf PPLANE9}}

% BADBAD: \newcommand{\Bbb}{\bf}

\newcommand{\R}{\mbox{$\Bbb{R}$}}
\newcommand{\C}{\mbox{$\Bbb{C}$}}
\newcommand{\Z}{\mbox{$\Bbb{Z}$}}
\newcommand{\N}{\mbox{$\Bbb{N}$}}
\newcommand{\D}{\mbox{{\bf D}}}
\usepackage{amssymb}
%\newcommand{\qed}{\hfill\mbox{\raggedright$\square$} \vspace{1ex}}
%\newcommand{\proof}{\noindent {\bf Proof:} \hspace{0.1in}}

\newcommand{\setmin}{\;\mbox{--}\;}
\newcommand{\Matlab}{{M\small{AT\-LAB}} }
\newcommand{\Matlabp}{{M\small{AT\-LAB}}}
\newcommand{\computer}{\Matlab Instructions}
\newcommand{\half}{\mbox{$\frac{1}{2}$}}
\newcommand{\compose}{\raisebox{.15ex}{\mbox{{\scriptsize$\circ$}}}}
\newcommand{\AND}{\quad\mbox{and}\quad}
\newcommand{\vect}[2]{\left(\begin{array}{c} #1_1 \\ \vdots \\
 #1_{#2}\end{array}\right)}
\newcommand{\mattwo}[4]{\left(\begin{array}{rr} #1 & #2\\ #3
&#4\end{array}\right)}
\newcommand{\mattwoc}[4]{\left(\begin{array}{cc} #1 & #2\\ #3
&#4\end{array}\right)}
\newcommand{\vectwo}[2]{\left(\begin{array}{r} #1 \\ #2\end{array}\right)}
\newcommand{\vectwoc}[2]{\left(\begin{array}{c} #1 \\ #2\end{array}\right)}

\newcommand{\ignore}[1]{}


\newcommand{\inv}{^{-1}}
\newcommand{\CC}{{\cal C}}
\newcommand{\CCone}{\CC^1}
\newcommand{\Span}{{\rm span}}
\newcommand{\rank}{{\rm rank}}
\newcommand{\trace}{{\rm tr}}
\newcommand{\RE}{{\rm Re}}
\newcommand{\IM}{{\rm Im}}
\newcommand{\nulls}{{\rm null\;space}}

\newcommand{\dps}{\displaystyle}
\newcommand{\arraystart}{\renewcommand{\arraystretch}{1.8}}
\newcommand{\arrayfinish}{\renewcommand{\arraystretch}{1.2}}
\newcommand{\Start}[1]{\vspace{0.08in}\noindent {\bf Section~\ref{#1}}}
\newcommand{\exer}[1]{\noindent {\bf \ref{#1}}}
\newcommand{\ans}{\textbf{Answer:} }
\newcommand{\matthree}[9]{\left(\begin{array}{rrr} #1 & #2 & #3 \\ #4 & #5 & #6
\\ #7 & #8 & #9\end{array}\right)}
\newcommand{\cvectwo}[2]{\left(\begin{array}{c} #1 \\ #2\end{array}\right)}
\newcommand{\cmatthree}[9]{\left(\begin{array}{ccc} #1 & #2 & #3 \\ #4 & #5 &
#6 \\ #7 & #8 & #9\end{array}\right)}
\newcommand{\vecthree}[3]{\left(\begin{array}{r} #1 \\ #2 \\
#3\end{array}\right)}
\newcommand{\cvecthree}[3]{\left(\begin{array}{c} #1 \\ #2 \\
#3\end{array}\right)}
\newcommand{\cmattwo}[4]{\left(\begin{array}{cc} #1 & #2\\ #3
&#4\end{array}\right)}

\newcommand{\Matrix}[1]{\ensuremath{\left(\begin{array}{rrrrrrrrrrrrrrrrrr} #1 \end{array}\right)}}

\newcommand{\Matrixc}[1]{\ensuremath{\left(\begin{array}{cccccccccccc} #1 \end{array}\right)}}



\renewcommand{\labelenumi}{\theenumi}
\newenvironment{enumeratea}%
{\begingroup
 \renewcommand{\theenumi}{\alph{enumi}}
 \renewcommand{\labelenumi}{(\theenumi)}
 \begin{enumerate}}
 {\end{enumerate}\endgroup}

\newcounter{help}
\renewcommand{\thehelp}{\thesection.\arabic{equation}}

%\newenvironment{equation*}%
%{\renewcommand\endequation{\eqno (\theequation)* $$}%
%   \begin{equation}}%
%   {\end{equation}\renewcommand\endequation{\eqno \@eqnnum
%$$\global\@ignoretrue}}

%\input{psfig.tex}

\author{Martin Golubitsky and Michael Dellnitz}

%\newenvironment{matlabEquation}%
%{\renewcommand\endequation{\eqno (\theequation*) $$}%
%   \begin{equation}}%
%   {\end{equation}\renewcommand\endequation{\eqno \@eqnnum
% $$\global\@ignoretrue}}

\newcommand{\soln}{\textbf{Solution:} }
\newcommand{\exercap}[1]{\centerline{Figure~\ref{#1}}}
\newcommand{\exercaptwo}[1]{\centerline{Figure~\ref{#1}a\hspace{2.1in}
Figure~\ref{#1}b}}
\newcommand{\exercapthree}[1]{\centerline{Figure~\ref{#1}a\hspace{1.2in}
Figure~\ref{#1}b\hspace{1.2in}Figure~\ref{#1}c}}
\newcommand{\para}{\hspace{0.4in}}

\usepackage{ifluatex}
\ifluatex
\ifcsname displaysolutions\endcsname%
\else
\renewenvironment{solution}{\suppress}{\endsuppress}
\fi
\else
\renewenvironment{solution}{}{}
\fi

%\ifxake
%\newenvironment{matlabEquation}{\begin{equation}}{\end{equation}}
%\else
\newenvironment{matlabEquation}%
{\let\oldtheequation\theequation\renewcommand{\theequation}{\oldtheequation*}\begin{equation}}%
  {\end{equation}\let\theequation\oldtheequation}
%\fi

\makeatother



\title{Parallel Lines}
\author{Brad Findell}
\begin{document}
\begin{abstract}
Proofs updated. 
\end{abstract}
\maketitle

This page develops important results regarding parallel lines and transversals.  \textbf{Read carefully, and complete the proofs.} 

\begin{axiom}
Parallel postulate (uniqueness of parallels):  Given a line and a point not on the line, there is exactly one line through the given point parallel to the given line.  
\end{axiom}

\begin{theorem}
A $180^\circ$ rotation about a point on a line takes the line to itself. 
\end{theorem}

\begin{problem}
\begin{proof}
Suppose point $P$ is on line $k$.  The point cuts the line into two opposite $\answer[format=string]{rays}$.  A $180^\circ$ rotation about $P$ swaps the two opposite rays, thereby mapping the line onto itself.  
\end{proof}
\end{problem}

\begin{theorem}
A $180^\circ$ rotation about a point not on a line takes the line to a parallel line.
\end{theorem}

\begin{problem}
\begin{proof}
Let $O$ be a point not on line $l$.  Let $P$ be an arbitrary point on $R(l)$, the rotated image of $l$.  
To show that $R(l)$ is parallel to $l$, 
it is sufficient to show that $P$ cannot lie also on $l$.  

\begin{image}
\definecolor{qqqqff}{rgb}{0.,0.,1.}
\begin{tikzpicture}[line cap=round,line join=round,>=triangle 45,x=1.0cm,y=1.0cm]
\clip(-0.8,6.0) rectangle (8.2,10.3);
\draw [line width=0.8pt,domain=-0.84:8.24] plot(\x,{(--43.8512--1.22*\x)/6.92});
\draw [line width=0.8pt,dash pattern=on 2pt off 2pt,domain=-0.84:8.24] plot(\x,{(-61.52-1.22*\x)/-6.92});
\begin{scriptsize}
\draw[color=black] (-0.04,6.7) node {$l$};
\draw [fill=qqqqff] (3.44,8.22) circle (1.2pt);
\draw[color=qqqqff] (3.55,8.50) node {$O$};
\draw[color=black] (-0.16,8.6) node {$R(l)$};
\draw [fill=qqqqff] (1.48135,9.1513) circle (1.2pt);
\draw[color=qqqqff] (1.62,9.45) node {$P$};
\draw [fill=qqqqff] (5.3986,7.2887) circle (1.2pt);
\draw[color=qqqqff] (5.54,7.05) node {$Q$};
\end{scriptsize}
\end{tikzpicture}
\end{image}
%\[
%\includegraphics[scale=0.6]{../graphics/lineRotation.pdf}
%\]

Because $P$ is on $R(l)$, there is a point $Q$ on $l$ such that $P = R(Q)$.  The rotated image of 
$\overrightarrow{OQ}$ is \wordChoice{\choice{$\overrightarrow{QO}$}\choice[correct]{$\overrightarrow{OP}$}\choice{$\overrightarrow{QP}$}}, and because $\angle QOP$ is $180^\circ$, it 
follows that $Q$, $O$, and $P$ are $\answer[format=string]{collinear}$.  Call that line $k$.  We know line $k$ is distinct 
from $l$ because point $\answer{O}$ is on $k$ but not on $l$.  Now, if $P$ were on $l$, then points $P$ and $Q$ 
would be on two distinct lines, $k$ and $l$, contradicting the assumption that on two points there 
is a unique line.  The theorem is proved.  
\end{proof}
\end{problem}

\begin{theorem}
If two parallel lines are cut by a transversal, alternate interior angles and corresponding angles are congruent.
\end{theorem}

\begin{problem}
\begin{proof}
Given that parallel lines $m$ and $n$ are cut by transversal $k$, prove that alternate interior angles are congruent.  

\begin{image}
\definecolor{uuuuuu}{rgb}{0.2667,0.2667,0.2667}
\definecolor{qqqqff}{rgb}{0.,0.,1.}
\begin{tikzpicture}[line cap=round,line join=round,>=triangle 45,x=1.0cm,y=1.0cm]
\clip(-0.2,1.7) rectangle (7.8,6.9);
\draw [line width=0.8pt,domain=-0.22:7.8] plot(\x,{(-0.--4.*\x)/2.});
\draw [line width=0.8pt,domain=-0.22:7.8] plot(\x,{(-10.--4.*\x)/2.});
\draw [line width=0.8pt,domain=-0.22:7.8] plot(\x,{(--10.--1.*\x)/3.});
\begin{scriptsize}
\draw [fill=qqqqff] (0.,0.) circle (1.5pt);
\draw [fill=qqqqff] (2.,4.) circle (1.5pt);
\draw[color=qqqqff] (2.1,3.8) node {$B$};
\draw[color=qqqqff] (1.7,3.7) node {$1$};
\draw[color=qqqqff] (2.35,4.3) node {$2$};
\draw[color=black] (0.77,2.21) node {$m$};
\draw [fill=qqqqff] (5.,5.) circle (1.5pt);
\draw[color=qqqqff] (5.1,4.8) node {$C$};
\draw[color=qqqqff] (4.65,4.7) node {$3$};
\draw[color=qqqqff] (4.8,5.2) node {$4$};
\draw[color=black] (3.8,2.19) node {$n$};
\draw[color=black] (0.82,3.91) node {$k$};
\draw [fill=qqqqff] (3.5,4.5) circle (1.5pt);
\draw[color=qqqqff] (3.64,4.79) node {$P$};
\end{scriptsize}
\end{tikzpicture}
\end{image}

Let $B$ and $C$ be the intersections of transversal $k$ with lines $m$ and $n$, respectively. Let $P$ be the midpoint of $\overline{BC}$.  

\begin{enumerate}
\item Rotate $180^\circ$ about $P$, which takes $k$ to \wordChoice{\choice[correct]{itself}\choice{$m$}\choice{$n$}}.  
\item The rotation maps $B$ to $\answer{C}$ because $PB = PC$ and the rotation preserves distances.  
\item Because $P$ is not on $m$, the rotation maps $m$ to a parallel line through $C$, which must be \wordChoice{\choice{$k$}\choice{$m$}\choice[correct]{$n$}} by the uniqueness of parallels.
%\item The rotation maps $n$ to \wordChoice{\choice{$k$}\choice[correct]{$m$}\choice{$n$}} by the same reasoning.
\item Thus, the rotation maps $\angle 2$ to \wordChoice{\choice{$\angle 1$}\choice{$\angle 2$}\choice[correct]{$\angle 3$}\choice{$\angle 4$}}.  These alternate interior angles must be congruent because the rotation preserves angle measures. 
\end{enumerate}
\end{proof}

\textbf{Note}: The congruence of corresponding angles now follows from the congruence of vertical angles.  But the next problem is another approach that uses a translation.  
\end{problem}

\begin{problem}
\begin{proof}
Given that parallel lines $m$ and $n$ are cut by transversal $k$, prove that corresponding angles are congruent.  

\begin{image}
\definecolor{uuuuuu}{rgb}{0.2667,0.2667,0.2667}
\definecolor{qqqqff}{rgb}{0.,0.,1.}
\begin{tikzpicture}[line cap=round,line join=round,>=triangle 45,x=1.0cm,y=1.0cm]
\clip(-0.2,1.7) rectangle (7.8,6.9);
\draw [line width=0.8pt,domain=-0.22:7.8] plot(\x,{(-0.--4.*\x)/2.});
\draw [line width=0.8pt,domain=-0.22:7.8] plot(\x,{(-10.--4.*\x)/2.});
\draw [line width=0.8pt,domain=-0.22:7.8] plot(\x,{(--10.--1.*\x)/3.});
\begin{scriptsize}
\draw [fill=qqqqff] (0.,0.) circle (1.5pt);
\draw [fill=qqqqff] (2.,4.) circle (1.5pt);
\draw[color=qqqqff] (2.1,3.8) node {$B$};
\draw[color=qqqqff] (1.7,3.7) node {$1$};
\draw[color=qqqqff] (2.35,4.3) node {$2$};
\draw[color=black] (0.77,2.21) node {$m$};
\draw [fill=qqqqff] (5.,5.) circle (1.5pt);
\draw[color=qqqqff] (5.1,4.8) node {$C$};
\draw[color=qqqqff] (4.65,4.7) node {$3$};
\draw[color=qqqqff] (4.8,5.2) node {$4$};
\draw[color=black] (3.8,2.19) node {$n$};
\draw[color=black] (0.82,3.91) node {$k$};
%\draw [fill=qqqqff] (3.5,4.5) circle (1.5pt);
%\draw[color=qqqqff] (3.64,4.79) node {$P$};
\end{scriptsize}
\end{tikzpicture}
\end{image}

Let $B$ and $C$ be the intersections of transversal $k$ with lines $m$ and $n$, respectively. 

\begin{enumerate}
\item Translate to the right along line $k$ by distance $BC$, which takes $k$ to \wordChoice{\choice[correct]{itself}\choice{$m$}\choice{$n$}}.
\item The translation maps $B$ to $\answer{C}$, and it maps $m$ to \wordChoice{\choice{$k$}\choice{$m$}\choice[correct]{$n$}} because the translation maintains parallels, and there is a unique parallel to $m$ through $C$.
\item The translation maps $\angle 1$ to \wordChoice{\choice{$\angle 1$}\choice{$\angle 2$}\choice[correct]{$\angle 3$}\choice{$\angle 4$}}.  These corresponding angles must be congruent because the translation preserves angle measures. 
\end{enumerate}
\end{proof}
\end{problem}

\begin{theorem}
If two lines are cut by a transversal so that alternate interior angles are congruent, then the lines are parallel. 

\begin{question}
This theorem is the $\answer[format=string]{converse}$ of the previous theorem about alternate interior angles.
\end{question}
\end{theorem}

\begin{problem}
\begin{proof}
Given that $m$ and $n$ are cut by transversal $k$ with alternate interior angles congruent, prove that lines $m$ and $n$ are parallel.  

\begin{image}
\definecolor{uuuuuu}{rgb}{0.2667,0.2667,0.2667}
\definecolor{qqqqff}{rgb}{0.,0.,1.}
\begin{tikzpicture}[line cap=round,line join=round,>=triangle 45,x=1.0cm,y=1.0cm]
\clip(-0.2,1.7) rectangle (7.8,6.9);
\draw [line width=0.8pt,domain=-0.22:7.8] plot(\x,{(-0.--4.*\x)/2.});
\draw [line width=0.8pt,domain=-0.22:7.8] plot(\x,{(-10.--4.*\x)/2.});
\draw [line width=0.8pt,domain=-0.22:7.8] plot(\x,{(--10.--1.*\x)/3.});
\begin{scriptsize}
\draw [fill=qqqqff] (0.,0.) circle (1.5pt);
\draw [fill=qqqqff] (2.,4.) circle (1.5pt);
\draw[color=qqqqff] (2.1,3.8) node {$B$};
\draw[color=qqqqff] (1.7,3.7) node {$1$};
\draw[color=qqqqff] (2.35,4.3) node {$2$};
\draw[color=black] (0.77,2.21) node {$m$};
\draw [fill=qqqqff] (5.,5.) circle (1.5pt);
\draw[color=qqqqff] (5.1,4.8) node {$C$};
\draw[color=qqqqff] (4.65,4.7) node {$3$};
\draw[color=qqqqff] (4.8,5.2) node {$4$};
\draw[color=black] (3.8,2.19) node {$n$};
\draw[color=black] (0.82,3.91) node {$k$};
\draw [fill=qqqqff] (3.5,4.5) circle (1.5pt);
\draw[color=qqqqff] (3.64,4.79) node {$P$};
\end{scriptsize}
\end{tikzpicture}
\end{image}

Let $B$ and $C$ be the intersections of transversal $k$ with lines $m$ and $n$, respectively. Let $P$ be the midpoint of $\overline{BC}$. 

\begin{enumerate}
\item Rotate $180^\circ$ about $P$, which takes $k$ to \wordChoice{\choice[correct]{itself}\choice{$m$}\choice{$n$}}, and which swaps $B$ and $\answer{C}$ because distances are preserved.  
\item Because $\angle 2 \cong \angle 3$ and because a side of $\angle 2$ (i.e., $\overrightarrow{BP}$) is mapped to a side of $\angle 3$ (i.e., \wordChoice{\choice[correct]{$\overrightarrow{CP}$}\choice{$\overrightarrow{PC}$}\choice{$\overrightarrow{BP}$}}), it must be that the other side of $\angle 2$ (which lies on $m$) is mapped to the other side of $\angle 3$ (which lies on line $\answer{n}$).  Thus, $n$ is the image of $m$.  
\item Because $P$ is not on $m$, the $180^\circ$ rotation maps $m$ to a parallel line through $C$.  Thus, $n$ must be parallel to $m$.  
\end{enumerate}
\end{proof}
\end{problem}

\end{document}
