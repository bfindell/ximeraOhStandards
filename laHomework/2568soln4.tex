\documentclass{article}


% For preamble materials

\usepackage{pgf,tikz}
\usepackage{mathrsfs}
\usetikzlibrary{shapes,arrows}
\usepackage{framed}
\pgfplotsset{compat=1.16}

\def\fixnote#1{\begin{framed}{\textcolor{red}{Fix note: #1}}\end{framed}}  % Allows insertion of red notes about needed edits
%\def\fixnote#1{}

\def\detail#1{{\textcolor{blue}{Detail: #1}}}   

\graphicspath{
  {./}
  {proofs/}
}

%\pdfOnly{\renewcommand{\answer}[1][[yy]{\fbox{\hspace{1in}\rule[-.3\baselineskip]{0pt}{15pt}}}}


\newcommand{\N}{\mathbb N}
\newcommand{\W}{\mathbb W}
\newcommand{\C}{\mathbb C}
\newcommand{\Z}{\mathbb Z}
\newcommand{\Q}{\mathbb Q}
\newcommand{\R}{\mathbb R}




\title{Math 2568 Homework 4}
\author{\phantom{Dr. Golubitsky}}
\date{Due: Monday, September 16, 2019}
  

\makeatletter
\newlabel{c4.3.6a}{{6}{123}}
\newlabel{c4.3.6d}{{9}{124}}
\newlabel{lin-matrices}{{3.3.5}{121}}
\newlabel{columnsA}{{3.3.4}{120}}
\newlabel{E:inhom}{{3.4.2}{135}}
\newlabel{c4.6.0a}{{5}{142}}
\newlabel{c4.6.0d}{{8}{142}}
\makeatother
\begin{document}
\maketitle


\problemlabel



\exerciselabel{15}{2.4}\begin{exercise}  \label{A:2.4.2}
Consider the augmented matrix 
\[
A=\left(\begin{array}{cc|c}  
 1 & -r & 1\\  
 r & -1 & 1  
\end{array}\right)
\]
where $r$ is a real parameter. 
\begin{enumerate}
\item Find all $r$ so that $\text{rank}(A)=2$.

\item Find all $r$ for which the corresponding linear system has 
\begin{enumerate}
\item no solution, 
\item one solution, and 
\item infinitely many solutions.
\end{enumerate}
\end{enumerate}

\begin{solution}
\soln Subtracting $r$ times the first row of $A$ from the second row of that matrix yields
\[
\left(\begin{array}{cc|c}  
 1 & -r & 1\\  
 0 & r^2-1 & 1-r 
\end{array}\right)=\left(\begin{array}{cc|c}  
 1 & -r & 1\\  
 0 & (r+1)(r-1) & 1-r 
\end{array}\right)
\]
So the reduced row echelon form of $A$ is
\[
\text{RREF}(A)=\begin{cases}
\left(\begin{array}{cc|c}  
 1 & 0 & \frac{1}{1+r}\\  
 0 & 1 & -\frac{1}{1+r}
\end{array}\right) & r\neq \pm 1\\
\left(\begin{array}{cc|c}  
 1 & -1 & 1\\  
 0 & 0 & 0 
\end{array}\right) & r=1\\
\left(\begin{array}{cc|c}  
 1 & 1 & 0\\  
 0 & 0 & 1 
\end{array}\right) & r=-1
\end{cases}
\]
\begin{enumerate}
\item $\text{rank}(A)=2$ if $r\neq 1$.
\item The linear system corresponding to the augmented matrix $A$ has 
\begin{enumerate}
\item no solution if $r=-1$, 
\item one solution if $r\neq \pm 1$, and  
\item infinitely many solutions if $r=1$.
\end{enumerate}
\end{enumerate}
\end{solution}
\end{exercise}


%%%%%%%%%%%%%%%%%%%%%%%%%%%%%%%%%%%%%%%%%%%%%%%%%%%%%%%%%%%%%%%%



\problemlabel

\exerciselabel{12}{3.1}\begin{exercise} \label{c4.1.8}
Let $A$ be a $2\times 2$ matrix.  Find $A$ so that
\begin{eqnarray*}
A\left(\begin{array}{r} 1 \\ 1 \end{array}\right) =
\left(\begin{array}{r} 2 \\ -1 \end{array}\right) \\
A\left(\begin{array}{r} 1 \\ -1 \end{array}\right) =
\left(\begin{array}{r} 4 \\ 3 \end{array}\right).
\end{eqnarray*}

\begin{solution}

\ans The equations are valid when
\[ A = \left(\begin{array}{rr} 3 & -1 \\ 1 & -2\end{array}\right). \]
\soln Let
\[ A = \left(\begin{array}{rr} a_{11} & a_{12} \\ a_{21} & 
a_{22}\end{array}\right) \]
Then
\[ \left(\begin{array}{rr} a_{11} & a_{12} \\ a_{21} & a_{22}\end{array}\right)
\left(\begin{array}{r} 1 \\ 1\end{array}\right) =
\left(\begin{array}{r} 2 \\ -1\end{array}\right) \AND
\left(\begin{array}{rr} a_{11} & a_{12} \\ a_{21} & a_{22}\end{array}\right)
\left(\begin{array}{r} 1 \\ -1\end{array}\right) =
\left(\begin{array}{r} 4 \\ 3.\end{array}\right) \]
These matrix equations yield the linear system
\[ \begin{array}{rrrrrrrrr}
a_{11} & + & a_{12} & & & & & = & 2 \\
& & & & a_{21} & + & a_{22} & = & -1 \\
a_{11} & - & a_{12} & & & & & = & 4 \\
& & & & a_{21} & - & a_{22} & = & 3, \end{array} \]
which can be written as an augmented matrix and row-reduced to yield
the values $a_{ij}$:
\[ \left(\begin{array}{rrrr|r}
1 & 1 & 0 & 0 & 2 \\
0 & 0 & 1 & 1 & -1 \\
1 & -1 & 0 & 0 & 4 \\
0 & 0 & 1 & -1 & 3\end{array}\right)
\longrightarrow
\left(\begin{array}{rrrr|r}
1 & 0 & 0 & 0 & 3 \\
0 & 1 & 0 & 0 & -1 \\
0 & 0 & 1 & 0 & 1 \\
0 & 0 & 0 & 1 & 2\end{array}\right). \]



\end{solution}
\end{exercise}


%%%%%%%%%%%%%%%%%%%%%%%%%%%%%%%%%%%%%%%%%%%%%%%%%%%%%%%%%%%%%%%%



\problemlabel

\noindent Determine whether the given transformation is linear.

\exerciselabel{7}{3.3}\begin{exercise} \label{c4.3.6b}
  $T:\R^2\to\R^2$ defined by $T(x_1,x_2)=(x_1+x_1x_2,2x_2)$.

\begin{solution}
\ans The transformation $T(x,y) = (x + xy, 2y)$ is not linear.

\soln If $T$ is a linear transformation, then
\[
T(x_1 + x_2,y_1 + y_2) = T(x_1,y_1) + T(x_2,y_2)
\]
for any real numbers $x_1$,$x_2$,$y_1$,$y_2$.  However,
\[
\begin{array}{rcl}
T(1,1) & = & (2,2) \\
T(1,0) + T(0,1) & = & (1,0) + (0,2) = (1,2).\end{array}
\]
Therefore $T(1,1) \neq T(1,0) + T(0,1)$ and $T$ is not linear.

\end{solution}
\end{exercise}


%%%%%%%%%%%%%%%%%%%%%%%%%%%%%%%%%%%%%%%%%%%%%%%%%%%%%%%%%%%%%%%%



\problemlabel

\noindent Determine whether the given transformation is linear.

\exerciselabel{9}{3.3}\begin{exercise} \label{c4.3.6d}
  $T:\R^2\to\R^3$ defined by $T(x_1,x_2)=(1,x_1+x_2,2x_2)$

\begin{solution}
The transformation $T(x,y) = (1,x + y, 2y)$ is not linear
because $T(0,0) = (1,0,0) \neq 0$.

\end{solution}
\end{exercise}


%%%%%%%%%%%%%%%%%%%%%%%%%%%%%%%%%%%%%%%%%%%%%%%%%%%%%%%%%%%%%%%%



\problemlabel

\exerciselabel{14}{3.3}\begin{exercise} \label{c4.3.10}
Let $\sigma:\R^3\to\R^3$ permute coordinates cyclically; that is,
\[
\sigma(x_1,x_2,x_3) = (x_2,x_3,x_1).
\]
Find the $3\times 3$ matrix $A$ such that $\sigma = L_A$.

\begin{solution}

\ans The matrix $A$ of the linear mapping $L_A$ is
\[ 
A = \matthree{0}{1}{0}{0}{0}{1}{1}{0}{0}. 
\]

\soln Note that if $\sigma = L_A$, then $\sigma(e_j) = Ae_j$ is the
$j^{th}$ column of matrix $A$.  Thus $A$ is determined by
\[
\begin{array}{l}
\sigma(e_1) = \sigma(1,0,0) = (0,0,1) \\
\sigma(e_2) = \sigma(0,1,0) = (1,0,0) \\
\sigma(e_3) = \sigma(0,0,1) = (0,1,0). \end{array}
\]


\end{solution}
\end{exercise}


%%%%%%%%%%%%%%%%%%%%%%%%%%%%%%%%%%%%%%%%%%%%%%%%%%%%%%%%%%%%%%%%



\problemlabel

\exerciselabel{16}{3.3}\begin{exercise}  \label{c4.3.12}
Let $P:\R^n\to\R^m$ and $Q:\R^n\to\R^m$ be linear mappings. 
\begin{enumeratea}
\item Prove that $S:\R^n\to\R^m$ defined by
\[
S(x) = P(x) + Q(x)
\]
is also a linear mapping.  
\item Theorem~\ref{lin-matrices} states that there are matrices $A$, $B$ and $C$ such that
\[
P = L_A \AND Q = L_B \AND S = L_C .
\]
What is the relationship between the matrices $A$, $B$, and $C$?
\end{enumeratea}

\begin{solution}

\soln The mapping $L$ is linear if $L(x + y) = L(x) + L(y)$ and if $cL(x) = L(cx)$.  
\begin{enumeratea}
\item We can use the assumption that $P(x)$ and $Q(x)$ are linear mappings to show:
\[ 
\begin{array}{rcl}
S(x + y) & = & P(x + y) + Q(x + y) \\
& = & P(x) + P(y) + Q(x) + Q(y) \\
& = & [P(x) + Q(x)] + [P(y) + Q(y)] \\
& = & S(x) + S(y) 
\end{array} 
\]
and
\[ 
\begin{array}{rcl}
cS(x) & = & cP(x) + cQ(x) \\
& = & P(cx) + Q(cx) \\
& = & S(cx). 
\end{array} 
\]

\item Assume that $S = L_C$, $P = L_A$ and $Q = L_B$ for
$m \times n$ matrices $A$, $B$, $C$.  We claim that
$A = B + C$.  By definition, $A(e_j) = L_A(e_j) =  L_B(e_j) + L_C(e_j) = (B+C)(e_j)$.  
Lemma~\ref{columnsA} implies that the $j^{th}$ column of 
$C$ is the sum of the $j^{th}$ column of $A$  and the $j^{th}$ column of $B$ 
for all columns $j$, so $C = A + B$.
\end{enumeratea}
\end{solution}
\end{exercise}


%%%%%%%%%%%%%%%%%%%%%%%%%%%%%%%%%%%%%%%%%%%%%%%%%%%%%%%%%%%%%%%%



\problemlabel

\exerciselabel{3}{3.4}\begin{exercise} \label{c4.4.3}
\begin{itemize}
\item[(a)] Find all solutions to the homogeneous equation
$Ax=0$ where
\[
A = \left(\begin{array}{ccc} 2 & 3 & 1 \\ 1 & 1 & 4 \end{array}
\right).
\]
\item[(b)] Find a single solution to the inhomogeneous equation
\begin{equation}  \label{E:inhom}
Ax =\vectwo{6}{6}.
\end{equation}
\item[(c)] Use your answers in (a) and (b) to find all solutions
to \eqref{E:inhom}.
\end{itemize}

\begin{solution}

(a) \ans All solutions to the homogeneous equation are of the form
\[
x = \vecthree{x_1}{x_2}{x_3} = s\vecthree{-11}{7}{1}.
\]

\soln Row reduce the matrix of the homogeneous system
$Ax = 0$ to obtain:
\[
\left(\begin{array}{rrr} 1 & 0 & 11 \\ 0 & 1 & -7 \end{array}\right).
\]
So $x_1 = -11s$, $x_2 = 7s$ and $x_3 = s$.

(b) \ans One possible solution is
\[ \vecthree{x_1}{x_2}{x_3} = \vecthree{1}{1}{1}. \]

\soln Assign a value to $x_3$, then substitute into the two equations
of the inhomogeneous system to obtain values for $x_1$ and $x_2$.

(c) All solutions to \eqref{E:inhom} can be found by adding a
single solution of the inhomogeneous system to all solutions
of the homogeneous system, so:
\[
x = \vecthree{1}{1}{1} + s\vecthree{-11}{7}{1}.
\]
\end{solution}
\end{exercise}


%%%%%%%%%%%%%%%%%%%%%%%%%%%%%%%%%%%%%%%%%%%%%%%%%%%%%%%%%%%%%%%%



\problemlabel

\exerciselabel{4}{3.4}\begin{exercise} \label{A.3.4.1}
How many solutions can a homogeneous system of $4$ linear equations in $7$ unknowns have?

\begin{solution}
\ans The system must have infinitely many solutions.

The system must have a solution because homogeneous systems are always consistent.
The system cannot have a unique solution because the rank of the corresponding augmented matrix cannot exceed $4$ which is less than the number of variables $7$.
\end{solution}
\end{exercise}


%%%%%%%%%%%%%%%%%%%%%%%%%%%%%%%%%%%%%%%%%%%%%%%%%%%%%%%%%%%%%%%%



\problemlabel

\noindent Compute the given matrix product.

\exerciselabel{8}{3.5}\begin{exercise}  \label{c4.6.0d}
$\left(\begin{array}{rrr} 2 & -1 & 3\\ 1 & 0 & 5\\1 & 5 & -1\end{array}\right)
\left(\begin{array}{rrr} 1 & 7 \\ -2 & -1 \\ -5 & 3\end{array}\right)$.

\begin{solution}

$\left(\begin{array}{rrr} 2 &  -1 &3\\ 1 & 0 & 5 \\1 & 5 & -1
\end{array}\right)
\left(\begin{array}{rrr} 1 & 7 \\ -2 & -1 \\ -5 & 3\end{array}\right)
= \left(\begin{array}{cc} 2+2-15 & 14+1+9 \\ 1-25 & 7+15 \\
1-10+5 & 7-5-3 \end{array}\right)
= \left(\begin{array}{rr} -11 & 24 \\ -24 & 22 \\ -4 & -1
\end{array}\right)$.


\end{solution}
\end{exercise}


%%%%%%%%%%%%%%%%%%%%%%%%%%%%%%%%%%%%%%%%%%%%%%%%%%%%%%%%%%%%%%%%



\problemlabel

\exerciselabel{4}{3.6}\begin{exercise} \label{c4.7.5}
Let
\[
I = \mattwo{1}{0}{0}{1} \AND J =\mattwo{0}{-1}{1}{0}.
\]
\begin{itemize}
\item[(a)] Show that $J^2=-I$.
\item[(b)] Evaluate $(aI+bJ)(cI+dJ)$ in terms of $I$ and $J$.
\end{itemize}

\begin{solution}

(a) Verify $J^2 = -I$ by computation:
\[ J^2 = \mattwo{0}{-1}{1}{0}\mattwo{0}{-1}{1}{0} =
\mattwo{-1}{0}{0}{-1} = -I. \]

(b) \ans $(aI + bJ)(cI + dJ) = (ac - bd)I + (ad + bc)J$.

\soln Evaluate $(aI + bJ)(cI + dJ)$, yielding
$acI^2 + adIJ + bcJI + bdJ^2$.  Then, use the identities $IJ = JI = J$,
$I^2 = I$, and $J^2 = -I$ to rewrite the expression in terms of $I$
and $J$.

\end{solution}
\end{exercise}


%%%%%%%%%%%%%%%%%%%%%%%%%%%%%%%%%%%%%%%%%%%%%%%%%%%%%%%%%%%%%%%%



\problemlabel

\exerciselabel{1}{3.7}\begin{exercise} \label{c4.8.1}
Verify by matrix multiplication that the following matrices are inverses
of each other:
\[
\left( \begin{array}{rrr}
1  &  0  &  2\\
0  & -1  &  2\\
1  &  0  &  1
\end{array} \right)\AND
\left( \begin{array}{rrr}
-1 &   0 &   2\\
 2 &  -1 &  -2\\
 1 &   0 &  -1
\end{array} \right).
\]

\begin{solution}

If two matrices are inverses of each other, then their product is the
identity matrix.  So:
\[ \matthree{1}{0}{2}{0}{-1}{2}{1}{0}{1}
\matthree{-1}{0}{2}{2}{-1}{-2}{1}{0}{-1} =
\matthree{1}{0}{0}{0}{1}{0}{0}{0}{1}. \]

\end{solution}
\end{exercise}


%%%%%%%%%%%%%%%%%%%%%%%%%%%%%%%%%%%%%%%%%%%%%%%%%%%%%%%%%%%%%%%%


\matlab

\matlabproblemlabel

\exerciselabel{10}{3.6}\begin{computerExercise} \label{c4.7.2.1}
Experimentally, find two symmetric $2\times 2$ matrices $A$ and $B$ for
which the matrix product $AB$ is {\em not\/} symmetric.

\begin{solution}
Let
\[ A = \mattwo{1}{2}{2}{-1} \AND B = \mattwo{2}{-1}{-1}{2} \]
be symmetric matrices.  Then
\[ AB = \mattwo{0}{3}{5}{-4} \]
is not symmetric.  In general, for
\[ A = \mattwo{a_{11}}{a_{12}}{a_{12}}{a_{22}} \AND
B = \mattwo{b_{11}}{b_{12}}{b_{12}}{b_{22}}, \]
$AB$ is symmetric if $a_{12}b_{11} + a_{22}b_{12} = a_{11}b_{12}
+ a_{12}b_{22}$.



\end{solution}
\end{computerExercise}


%%%%%%%%%%%%%%%%%%%%%%%%%%%%%%%%%%%%%%%%%%%%%%%%%%%%%%%%%%%%%%%%





\end{document}
