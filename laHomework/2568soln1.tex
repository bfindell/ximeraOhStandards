\documentclass{ximera}

%\makeatletter
%\newlabel{c1.1.1A}{{1}{3}}
%\newlabel{c1.1.1C}{{3}{4}}
%\newlabel{c1.1.3a}{{5}{4}}
%\newlabel{c1.1.3e}{{9}{5}}
%\newlabel{c1.1.4A}{{10}{5}}
%\newlabel{c1.1.4B}{{11}{5}}
%\newlabel{c1.2.3a}{{3}{10}}
%\newlabel{c1.2.3b}{{4}{10}}
%\newlabel{c1.2.4a}{{5}{10}}
%\newlabel{c1.2.4b}{{6}{10}}
%\newlabel{c1.1.01a}{{1}{13}}
%\newlabel{c1.1.01e}{{5}{13}}
%\newlabel{c1.3.1a}{{11}{14}}
%\newlabel{c1.3.3c}{{16}{15}}
%\newlabel{c1.3.4a}{{17}{15}}
%\newlabel{c1.3.4c}{{19}{15}}
%\newlabel{c1.4.6a}{{21}{25}}
%\newlabel{c1.4.6c}{{23}{25}}
%\makeatother

\usepackage{epsfig}

\graphicspath{
  {./}
  {figures/}
  {../laode}
  {../laode/figures}
}

\usepackage{epstopdf}
\epstopdfsetup{outdir=./}

\usepackage{morewrites}
\makeatletter
\newcommand\subfile[1]{%
\renewcommand{\input}[1]{}%
\begingroup\skip@preamble\otherinput{#1}\endgroup\par\vspace{\topsep}
\let\input\otherinput}
\makeatother

\newcommand{\EXER}{}
\newcommand{\includeexercises}{\EXER\directlua{dofile(kpse.find_file("exercises","lua"))}}

\newenvironment{computerExercise}{\begin{exercise}}{\end{exercise}}

%\newcounter{ccounter}
%\setcounter{ccounter}{1}
%\newcommand{\Chapter}[1]{\setcounter{chapter}{\arabic{ccounter}}\chapter{#1}\addtocounter{ccounter}{1}}

%\newcommand{\section}[1]{\section{#1}\setcounter{thm}{0}\setcounter{equation}{0}}

%\renewcommand{\theequation}{\arabic{chapter}.\arabic{section}.\arabic{equation}}
%\renewcommand{\thefigure}{\arabic{chapter}.\arabic{figure}}
%\renewcommand{\thetable}{\arabic{chapter}.\arabic{table}}

%\newcommand{\Sec}[2]{\section{#1}\markright{\arabic{ccounter}.\arabic{section}.#2}\setcounter{equation}{0}\setcounter{thm}{0}\setcounter{figure}{0}}
  
\newcommand{\Sec}[2]{\section{#1}}

\setcounter{secnumdepth}{2}
%\setcounter{secnumdepth}{1} 

%\newcounter{THM}
%\renewcommand{\theTHM}{\arabic{chapter}.\arabic{section}}

\newcommand{\trademark}{{R\!\!\!\!\!\bigcirc}}
%\newtheorem{exercise}{}

\newcommand{\dfield}{{\sf dfield9}}
\newcommand{\pplane}{{\sf pplane9}}
\newcommand{\PPLANE}{{\sf PPLANE9}}

% BADBAD: \newcommand{\Bbb}{\bf}

\newcommand{\R}{\mbox{$\Bbb{R}$}}
\newcommand{\C}{\mbox{$\Bbb{C}$}}
\newcommand{\Z}{\mbox{$\Bbb{Z}$}}
\newcommand{\N}{\mbox{$\Bbb{N}$}}
\newcommand{\D}{\mbox{{\bf D}}}
\usepackage{amssymb}
%\newcommand{\qed}{\hfill\mbox{\raggedright$\square$} \vspace{1ex}}
%\newcommand{\proof}{\noindent {\bf Proof:} \hspace{0.1in}}

\newcommand{\setmin}{\;\mbox{--}\;}
\newcommand{\Matlab}{{M\small{AT\-LAB}} }
\newcommand{\Matlabp}{{M\small{AT\-LAB}}}
\newcommand{\computer}{\Matlab Instructions}
\newcommand{\half}{\mbox{$\frac{1}{2}$}}
\newcommand{\compose}{\raisebox{.15ex}{\mbox{{\scriptsize$\circ$}}}}
\newcommand{\AND}{\quad\mbox{and}\quad}
\newcommand{\vect}[2]{\left(\begin{array}{c} #1_1 \\ \vdots \\
 #1_{#2}\end{array}\right)}
\newcommand{\mattwo}[4]{\left(\begin{array}{rr} #1 & #2\\ #3
&#4\end{array}\right)}
\newcommand{\mattwoc}[4]{\left(\begin{array}{cc} #1 & #2\\ #3
&#4\end{array}\right)}
\newcommand{\vectwo}[2]{\left(\begin{array}{r} #1 \\ #2\end{array}\right)}
\newcommand{\vectwoc}[2]{\left(\begin{array}{c} #1 \\ #2\end{array}\right)}

\newcommand{\ignore}[1]{}


\newcommand{\inv}{^{-1}}
\newcommand{\CC}{{\cal C}}
\newcommand{\CCone}{\CC^1}
\newcommand{\Span}{{\rm span}}
\newcommand{\rank}{{\rm rank}}
\newcommand{\trace}{{\rm tr}}
\newcommand{\RE}{{\rm Re}}
\newcommand{\IM}{{\rm Im}}
\newcommand{\nulls}{{\rm null\;space}}

\newcommand{\dps}{\displaystyle}
\newcommand{\arraystart}{\renewcommand{\arraystretch}{1.8}}
\newcommand{\arrayfinish}{\renewcommand{\arraystretch}{1.2}}
\newcommand{\Start}[1]{\vspace{0.08in}\noindent {\bf Section~\ref{#1}}}
\newcommand{\exer}[1]{\noindent {\bf \ref{#1}}}
\newcommand{\ans}{\textbf{Answer:} }
\newcommand{\matthree}[9]{\left(\begin{array}{rrr} #1 & #2 & #3 \\ #4 & #5 & #6
\\ #7 & #8 & #9\end{array}\right)}
\newcommand{\cvectwo}[2]{\left(\begin{array}{c} #1 \\ #2\end{array}\right)}
\newcommand{\cmatthree}[9]{\left(\begin{array}{ccc} #1 & #2 & #3 \\ #4 & #5 &
#6 \\ #7 & #8 & #9\end{array}\right)}
\newcommand{\vecthree}[3]{\left(\begin{array}{r} #1 \\ #2 \\
#3\end{array}\right)}
\newcommand{\cvecthree}[3]{\left(\begin{array}{c} #1 \\ #2 \\
#3\end{array}\right)}
\newcommand{\cmattwo}[4]{\left(\begin{array}{cc} #1 & #2\\ #3
&#4\end{array}\right)}

\newcommand{\Matrix}[1]{\ensuremath{\left(\begin{array}{rrrrrrrrrrrrrrrrrr} #1 \end{array}\right)}}

\newcommand{\Matrixc}[1]{\ensuremath{\left(\begin{array}{cccccccccccc} #1 \end{array}\right)}}



\renewcommand{\labelenumi}{\theenumi}
\newenvironment{enumeratea}%
{\begingroup
 \renewcommand{\theenumi}{\alph{enumi}}
 \renewcommand{\labelenumi}{(\theenumi)}
 \begin{enumerate}}
 {\end{enumerate}\endgroup}

\newcounter{help}
\renewcommand{\thehelp}{\thesection.\arabic{equation}}

%\newenvironment{equation*}%
%{\renewcommand\endequation{\eqno (\theequation)* $$}%
%   \begin{equation}}%
%   {\end{equation}\renewcommand\endequation{\eqno \@eqnnum
%$$\global\@ignoretrue}}

%\input{psfig.tex}

\author{Martin Golubitsky and Michael Dellnitz}

%\newenvironment{matlabEquation}%
%{\renewcommand\endequation{\eqno (\theequation*) $$}%
%   \begin{equation}}%
%   {\end{equation}\renewcommand\endequation{\eqno \@eqnnum
% $$\global\@ignoretrue}}

\newcommand{\soln}{\textbf{Solution:} }
\newcommand{\exercap}[1]{\centerline{Figure~\ref{#1}}}
\newcommand{\exercaptwo}[1]{\centerline{Figure~\ref{#1}a\hspace{2.1in}
Figure~\ref{#1}b}}
\newcommand{\exercapthree}[1]{\centerline{Figure~\ref{#1}a\hspace{1.2in}
Figure~\ref{#1}b\hspace{1.2in}Figure~\ref{#1}c}}
\newcommand{\para}{\hspace{0.4in}}

\usepackage{ifluatex}
\ifluatex
\ifcsname displaysolutions\endcsname%
\else
\renewenvironment{solution}{\suppress}{\endsuppress}
\fi
\else
\renewenvironment{solution}{}{}
\fi

%\ifxake
%\newenvironment{matlabEquation}{\begin{equation}}{\end{equation}}
%\else
\newenvironment{matlabEquation}%
{\let\oldtheequation\theequation\renewcommand{\theequation}{\oldtheequation*}\begin{equation}}%
  {\end{equation}\let\theequation\oldtheequation}
%\fi

\makeatother



\def\R{\mathbb R}
\def\mattwo#1#2#3#4{\left(\begin{array}{rr} #1 & #2 \\ #3 & #4\end{array}\right)}
\def\AND{\quad\mbox{and}\quad}


\date{Due: September 1, 2020}
\title{Math 2568 Homework 1}
\author{Martin Golubitsky and Michael Dellnitz}
\begin{document}
\begin{abstract}
Online versions of Homework 1.
\end{abstract}
\maketitle

\section*{This is a testing section}
pplane \pplane 
PPLANE \PPLANE

The answer is, \ans Yes! 

\ignore{this is not to be seen}
Did you see the ignored stuff?  

\detail{Here is some detail. Even $3+4=7$.}



\problemlabel

\noindent Let $x=(2,1,3)$ and  $y=(1,1,-1)$ and compute the given expression.

\exerciselabel{2}{1.1}\begin{exercise}  \label{c1.1.1B}
  $2x-3y\begin{prompt}
    = \left(\answer{1},\answer{-1},\answer{9}\right)
  \end{prompt}$.
  \begin{hint}
    $2x - 3y = (4,2,6) - (3,3,-3)$.
  \end{hint}
  \begin{hint}
    $(4,2,6) - (3,3,-3) = (1,-1,9)$.
  \end{hint}  

\begin{solution}
$2x - 3y = (4,2,6) - (3,3,-3) = (1,-1,9)$.

\end{solution}
\end{exercise}


%%%%%%%%%%%%%%%%%%%%%%%%%%%%%%%%%%%%%%%%%%%%%%%%%%%%%%%%%%%%%%%%



\problemlabel



\exerciselabel{4}{1.1}\begin{exercise} \label{c1.1.2}
Let $A$ be the $3\times 4$ matrix
\[
A=\left(\begin{array}{rrrr} 2 & -1 & 0 & 1 \\ 3 & 4 & -7 & 10\\
        6 & -3 & 4 & 2 \end{array}\right).
\]
\begin{enumerate}
\item[(a)]  For which $n$ is a row of $A$ a vector in $\R^n$? \begin{prompt}\[n = \answer{4}\]\end{prompt}.
\item[(b)]  What is the $2^{nd}$ column of $A$?
  \begin{prompt}
    \[
      \left(\begin{array}{r} \answer{-1} \\ \answer{4} \\ \answer{-3} \end{array} \right)
    \]
  \end{prompt}
\item[(c)] Let $a_{ij}$ be the entry of $A$ in the $i^{th}$ row
  and the $j^{th}$ column.  What is $a_{23}-a_{31}$?
  \begin{prompt}
    \[
      a_{23}-a_{31} = \answer{-13}.
    \]
  \end{prompt}
\end{enumerate}

\begin{solution}

(a) The number of entries in a row is the number of columns.  Thus, $n = 4$; \\
(b) $\left(\begin{array}{r} -1 \\ 4 \\ -3 \end{array} \right)$;
(c) $a_{23}-a_{31} =  -7 - 6 = -13$.

\end{solution}
\end{exercise}


%%%%%%%%%%%%%%%%%%%%%%%%%%%%%%%%%%%%%%%%%%%%%%%%%%%%%%%%%%%%%%%%



\problemlabel

\noindent For each of the pairs of vectors or matrices decide whether addition of the members of the pair is possible; and, if addition is possible, perform the addition.

\exerciselabel{7}{1.1}\begin{exercise}\label{c1.1.3c}
  $x=(1,2,3)$ and $y=(-2,1)$.
  
  \begin{multipleChoice}
    \choice{Addition is possible.}
    \choice[correct]{Addition is not possible.}
  \end{multipleChoice}  

\begin{solution}
$x$ has three entries; $y$ has two entries; addition is not possible.

\end{solution}
\end{exercise}


%%%%%%%%%%%%%%%%%%%%%%%%%%%%%%%%%%%%%%%%%%%%%%%%%%%%%%%%%%%%%%%%



\problemlabel

\noindent Let $A=\mattwo{2}{1}{-1}{4}$ and $B=\mattwo{0}{2}{3}{-1}$ and compute the given  expression.

\exerciselabel{10}{1.1}\begin{exercise}\label{c1.1.4A}
  $4A+B\begin{prompt}= \mattwo{\answer{8}}{\answer{6}}{\answer{-1}}{\answer{15}}\end{prompt}$.

\begin{solution}
$4A + B = \mattwo{8}{6}{-1}{15}$.


\end{solution}
\end{exercise}


%%%%%%%%%%%%%%%%%%%%%%%%%%%%%%%%%%%%%%%%%%%%%%%%%%%%%%%%%%%%%%%%



\matlabproblemlabel

\noindent Let  $x=(1.2,1.4,-2.45) \AND y=(-2.6,1.1,0.65)$ and use \Matlab to compute the  given expression.

\exerciselabel{3}{1.2}\begin{computerExercise}  \label{c1.2.3a}
$3.27x-7.4y$.

\begin{solution}
$3.27x - 7.4y = (23.1640, -3.5620, -12.8215)$.

\end{solution}
\end{computerExercise}


%%%%%%%%%%%%%%%%%%%%%%%%%%%%%%%%%%%%%%%%%%%%%%%%%%%%%%%%%%%%%%%%



\matlabproblemlabel

\noindent Let  \[ A = \left(\begin{array}{rrr} 1.2 & 2.3 & -0.5\\ 0.7 & -1.4 & 2.3 \end{array}\right) \AND B = \left(\begin{array}{rrr} -2.9 & 1.23 & 1.6\\ -2.2 & 1.67 & 0 \end{array}\right) \] and use \Matlab to compute the given expression.

\exerciselabel{5}{1.2}\begin{computerExercise}  \label{c1.2.4a}
$-4.2A+3.1B$.

\begin{solution}
$-4.2A + 3.1B = \left(\begin{array}{rrr} 
-14.0300 & -5.8470 &    7.0600 \\
 -9.7600 & 11.0570 &   -9.6600\end{array}\right)$.

\end{solution}
\end{computerExercise}


%%%%%%%%%%%%%%%%%%%%%%%%%%%%%%%%%%%%%%%%%%%%%%%%%%%%%%%%%%%%%%%%



\problemlabel

\noindent Decide whether or not the given matrix is symmetric.

\exerciselabel{5}{1.3}\begin{exercise} \label{c1.1.01e}
 $A = \left( \begin{array}{rrr}
 3 & 4 & -1\\
 4 & 3 &  1\\
 -1 & 1 & 10\end{array} \right)$.
  \begin{multipleChoice}
    \choice[correct]{The matrix is symmetric.}
    \choice{The matrix is not symmetric.}    
  \end{multipleChoice}
       

\begin{solution}
Since $a_{21} = a_{12}$, $a_{31} = a_{13}$, and $a_{32} = a_{23}$, the matrix is symmetric.

\end{solution}
\end{exercise}


%%%%%%%%%%%%%%%%%%%%%%%%%%%%%%%%%%%%%%%%%%%%%%%%%%%%%%%%%%%%%%%%



\problemlabel

\noindent A general $2\times 2$ diagonal matrix has the form $\mattwo{a}{0}{0}{b}$.  Thus the two unknown real numbers $a$ and $b$ are needed to specify each $2\times 2$ diagonal matrix.  how many unknown real numbers are needed to specify each of the given matrices:

\exerciselabel{11}{1.3}\begin{exercise}  \label{c1.3.1a}
An upper triangular $2\times 2$ matrix? \begin{prompt}$\answer{3}$\end{prompt}

\begin{solution}
A $2\times 2$ upper triangular matrix $A$ has the form $A = \left( \begin{array}{cc}
            a_{11} & a_{12} \\
            0 & a_{22} \end{array} \right)$.  Thus the number of entries needed to define $A$ is $3$.  

\end{solution}
\end{exercise}


%%%%%%%%%%%%%%%%%%%%%%%%%%%%%%%%%%%%%%%%%%%%%%%%%%%%%%%%%%%%%%%%



\problemlabel

\noindent A general $2\times 2$ diagonal matrix has the form $\mattwo{a}{0}{0}{b}$.  Thus the two unknown real numbers $a$ and $b$ are needed to specify each $2\times 2$ diagonal matrix.  how many unknown real numbers are needed to specify each of the given matrices:

\exerciselabel{13}{1.3}\begin{exercise}  \label{c1.3.2}
An $m\times n$ matrix? \begin{prompt}$\answer{mn}$\end{prompt}

\begin{solution}
Each row of the matrix has $n$ entries and there are $m$ rows.  Hence the number of unknown entries is $mn$.

\end{solution}
\end{exercise}


%%%%%%%%%%%%%%%%%%%%%%%%%%%%%%%%%%%%%%%%%%%%%%%%%%%%%%%%%%%%%%%%



\problemlabel

\noindent A general $2\times 2$ diagonal matrix has the form $\mattwo{a}{0}{0}{b}$.  Thus the two unknown real numbers $a$ and $b$ are needed to specify each $2\times 2$ diagonal matrix.  how many unknown real numbers are needed to specify each of the given matrices:

\exerciselabel{16}{1.3}\begin{exercise}  \label{c1.3.3c}
A symmetric $n\times n$ matrix?   
\begin{prompt}$\answer{n(n+1)/2}$\end{prompt}

\begin{solution}
The number of independent entries in row $k$ of an $n\times n$ symmetric matrix is $n-k+1$.  Thus the number of independent entries in the matrix is 
\[
n + (n-1) + \cdots + 1 = 1 + 2 + \cdots + n = \sum_{k=1}^n k = \frac{n(n + 1)}{2}.
\]
\end{solution}
\end{exercise}


%%%%%%%%%%%%%%%%%%%%%%%%%%%%%%%%%%%%%%%%%%%%%%%%%%%%%%%%%%%%%%%%



\problemlabel

\noindent Determine whether the statement is {\em True\/} or {\em False\/}?

\exerciselabel{18}{1.3}\begin{exercise} \label{c1.3.4b}
  Every diagonal matrix is a multiple of the identity matrix.
  \begin{multipleChoice}
    \choice{True}
    \choice[correct]{False}    
  \end{multipleChoice}
  \begin{feedback}
    That's right.   An example is
    \[\left(\begin{array}{ccc}
              2 & 0 & 0 \\
              0 & 1 & 0 \\
              0 & 0 & 3 \end{array}\right).\]
  \end{feedback}

\begin{solution}
$False$ --- for example:
$\left(\begin{array}{ccc}
2 & 0 & 0 \\
0 & 1 & 0 \\
0 & 0 & 3 \end{array}\right)$.

\end{solution}
\end{exercise}


%%%%%%%%%%%%%%%%%%%%%%%%%%%%%%%%%%%%%%%%%%%%%%%%%%%%%%%%%%%%%%%%



\problemlabel

\exerciselabel{9}{1.4}\begin{exercise} \label{c1.4.2}
Find a real number $a\begin{prompt}=\answer{10/3}\end{prompt}$ so that the vectors
\[
x = (1,3,2) \AND y = (2,a,-6)
\]
are perpendicular.
\begin{hint}
  The vectors $x$ and $y$ are perpendicular when
$(1,3,2) \cdot (2,a,-6) = 3a - 10 = 0$.
\end{hint}
\begin{hint}
  This means that $a = \frac{10}{3}$.
\end{hint}

\begin{solution}

The vectors $x$ and $y$ are perpendicular when $(1,3,2) \cdot (2,a,-6) = 3a - 10 = 0$.  Thus, $a = \frac{10}{3}$.

\end{solution}
\end{exercise}


%%%%%%%%%%%%%%%%%%%%%%%%%%%%%%%%%%%%%%%%%%%%%%%%%%%%%%%%%%%%%%%%



\matlabproblemlabel

\noindent Find the angle in degrees between the given pair of vectors.

\exerciselabel{21}{1.4}\begin{computerExercise} \label{c1.4.6a}
$x=(2,1,-3,4)$ and $y=(1,1,-5,7)$.

\end{computerExercise}


%%%%%%%%%%%%%%%%%%%%%%%%%%%%%%%%%%%%%%%%%%%%%%%%%%%%%%%%%%%%%%%%



\matlabproblemlabel

\noindent Enter a matrix $M$ with $x$ as row 1 and $y$ as row 2.  

\begin{exercise}
$x=(2,1)$ and $y=(-5,7)$.

\begin{solution}
\[
M = \left(\begin{array}{rr} \answer{2} & \answer{1} \\ \answer{-5} & \answer{7} \end{array}\right)
\]
\end{solution}
\end{exercise}


%%%%%%%%%%%%%%%%%%%%%%%%%%%%%%%%%%%%%%%%%%%%%%%%%%%%%%%%%%%%%%%%


\end{document}
