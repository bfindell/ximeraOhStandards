\documentclass{article}

% For preamble materials

\usepackage{pgf,tikz}
\usepackage{mathrsfs}
\usetikzlibrary{shapes,arrows}
\usepackage{framed}
\pgfplotsset{compat=1.16}

\def\fixnote#1{\begin{framed}{\textcolor{red}{Fix note: #1}}\end{framed}}  % Allows insertion of red notes about needed edits
%\def\fixnote#1{}

\def\detail#1{{\textcolor{blue}{Detail: #1}}}   

\graphicspath{
  {./}
  {proofs/}
}

%\pdfOnly{\renewcommand{\answer}[1][[yy]{\fbox{\hspace{1in}\rule[-.3\baselineskip]{0pt}{15pt}}}}


\newcommand{\N}{\mathbb N}
\newcommand{\W}{\mathbb W}
\newcommand{\C}{\mathbb C}
\newcommand{\Z}{\mathbb Z}
\newcommand{\Q}{\mathbb Q}
\newcommand{\R}{\mathbb R}




\newcommand{\MATLAB}{{\bf Matlab. }}

\title{Math 2568 Homework 5}
\author{\phantom{Dr. Golubitsky}}
\date{Due: Monday, September 23, 2019}


\makeatletter
\newlabel{sum}{{3.3.1}{117}}
\newlabel{product}{{3.3.2}{117}}
\newlabel{c4.6.-1a}{{1}{141}}
\newlabel{c4.6.-1d}{{4}{142}}
\newlabel{c4.9.3a}{{5}{162}}
\newlabel{c4.9.3b}{{6}{162}}
\newlabel{invertequiv}{{3.7.8}{156}}
\newlabel{e:formAinv}{{3.8.1}{168}}
\makeatother
\begin{document}
\maketitle


\problemlabel



\exerciselabel{2}{3.4}\begin{exercise} \label{c4.4.2}
Write all solutions to the homogeneous system of linear
equations
\begin{eqnarray*}
x_1+2x_2+x_4-x_5 = 0\\
x_3-2x_4+x_5 = 0
\end{eqnarray*}
as the general superposition of three vectors.

\begin{solution}

\ans Every solution can be written as a superposition of the vectors
\[
\left(\begin{array}{r} -2 \\ 1 \\ 0 \\ 0 \\ 0\end{array}\right), \quad
\left(\begin{array}{r} -1 \\ 0 \\ 2 \\ 1 \\ 0 \end{array}\right) \AND
\left(\begin{array}{r} 1 \\ 0 \\ -1 \\ 0 \\ 1 \end{array}\right).
\]

\soln Write the matrix of the homogeneous system:
\[
\left(\begin{array}{rrrrr} 1 & 2 & 0 & 1 & -1 \\
0 & 0 & 1 & -2 & 1 \end{array}\right).
\]
This matrix cannot be row reduced further.  Every solution has the form
\[
\left(\begin{array}{r} x_1 \\ x_2 \\ x_3 \\ x_4 \\ x_5
\end{array}\right) = \left(\begin{array}{c} x_5 - x_4 - 2x_2 \\
x_2 \\ -x_5 + 2x_4 \\ x_4 \\ x_5 \end{array}\right) = 
x_2\left(\begin{array}{r} -2 \\ 1 \\ 0 \\ 0 \\ 0
\end{array}\right) + x_4\left(\begin{array}{r} -1 \\ 0 \\ 2 \\
1 \\ 0 \end{array}\right) + x_5\left(\begin{array}{r} 1 \\ 0 \\
-1 \\ 0 \\ 1 \end{array}\right).
\]

\end{solution}
\end{exercise}


%%%%%%%%%%%%%%%%%%%%%%%%%%%%%%%%%%%%%%%%%%%%%%%%%%%%%%%%%%%%%%%%


\newpage

\problemlabel

\exerciselabel{12}{3.3}\begin{exercise} \label{c4.3.8}
The {\em cross product\/} of two $3$-vectors $x=(x_1,x_2,x_3)$
and $y=(y_1,y_2,y_3)$ is the $3$-vector
\[
x\times y = (x_2y_3-x_3y_2,-(x_1y_3-x_3y_1),x_1y_2-x_2y_1).
\]
Let $K=(2,1,-1)$.  
\begin{enumeratea}
\item Show that the mapping $L:\R^3\to\R^3$ defined by
\[
L(x) = x\times K
\]
is a linear mapping.  
\item Find the $3\times 3$ matrix $A$ such that
\[
L(x) = Ax,
\]
that is, $L=L_A$.
\end{enumeratea}
\begin{solution}

\ans The matrix of linear mapping $L$ is
\[
A = \matthree{0}{-1}{-1}{1}{0}{2}{1}{-2}{0}.
\]

\soln Let $X = (x_1,x_2,x_3)$ and let $Y = (y_1,y_2,y_3)$.  
Since $K = (2,1,-1)$,
\[
L(X) = (x_1,x_2,x_3) \times K = 
(-x_2 - x_3, x_1 + 2x_3, x_1 - 2x_2).
\]

To show that $L(X)$ is a linear mapping, first demonstrate that
\eqref{sum} is valid:
\[
\begin{array}{rcl}
L(X + Y) & = & L(x_1 + y_1,x_2 + y_2,x_3 + y_3) \\
& = & (-(x_2 + y_2) - (x_3 + y_3), (x_1 + y_1) + 2(x_3 + y_3),
(x_1 + y_1) - 2(x_2 + y_2)) \\
& = & (-x_2 - x_3, x_1 + 2x_3, x_1 - 2x_2) +
(-y_2 - y_3, y_1 + 2y_3, y_1 - 2y_2) \\
& = & L(X) + L(Y), \end{array}
\]
then show that \eqref{product} is valid:
\[
\begin{array}{rcl}
cL(X) & = & cL(x_1,x_2,x_3) \\
& = & c(-x_2 - x_3, x_1 + 2x_3, x_1 - 2x_2) \\
& = & (-cx_2 - cx_3, cx_1 + 2cx_3, cx_1 - 2cx_2) \\
& = & L(cx_1,cx_2,cx_3) \\
& = & L(cX). \end{array}
\]

Find $A$ by noting that $L(e_j) = Ae_j$ is the $j^{th}$ column of $A$,
and computing
\[ \begin{array}{l}
L(e_1) = L(1,0,0) = (0,1,1) \\
L(e_2) = L(0,1,0) = (-1,0,-2) \\
L(e_3) = L(0,0,1) = (-1,2,0). \end{array} \]


\end{solution}
\end{exercise}


%%%%%%%%%%%%%%%%%%%%%%%%%%%%%%%%%%%%%%%%%%%%%%%%%%%%%%%%%%%%%%%%



\problemlabel

\noindent Determine whether or  not the matrix products $AB$ or $BA$ can be computed for each given pair of  matrices $A$ and $B$.  If the product is possible, perform the computation.

\exerciselabel{2}{3.5}\begin{exercise}  \label{c4.6.-1b}
$A=\left(\begin{array}{rrr} 0 & -2 & 1\\ 4 & 10 & 0 \end{array}\right)$
and $B=\left(\begin{array}{rr} 0 & 2 \\ 3 & -1 \end{array}\right)$.

\begin{solution}
$AB$ is not defined. $BA=\left(\begin{array}{rrr} 8 &  20 &  0\\
 -4 & -16  &  3\end{array}\right)$


\end{solution}
\end{exercise}


%%%%%%%%%%%%%%%%%%%%%%%%%%%%%%%%%%%%%%%%%%%%%%%%%%%%%%%%%%%%%%%%



\problemlabel

\exerciselabel{2}{3.7}\begin{exercise} \label{c4.8.2}
Let $\alpha \not=0$ be a real number and let $A$ be an invertible
matrix.  Show that the inverse of the matrix $\alpha A$ is given by
$\frac{1}{\alpha}A^{-1}$.

\begin{solution}

We can compute
\[ (\alpha A)\left(\frac{1}{\alpha}A^{-1}\right) =
\left(\alpha\frac{1}{\alpha}\right)(AA^{-1}) = I. \]
So the inverse of $\alpha A$ is indeed $\frac{1}{\alpha}A^{-1}$.

\end{solution}
\end{exercise}


%%%%%%%%%%%%%%%%%%%%%%%%%%%%%%%%%%%%%%%%%%%%%%%%%%%%%%%%%%%%%%%%



\problemlabel

\noindent Use row reduction to find the inverse of the given matrix.

\exerciselabel{5}{3.7}\begin{exercise} \label{c4.9.3a}
$\left(\begin{array}{rrr} 1 & 4 & 5\\ 0 & 1 & -1\\ -2 & 0 & -8
\end{array}\right)$.

\begin{solution}
\ans
$A^{-1} = \dps\frac{1}{10}\matthree{-8}{32}{-9}{2}{2}{1}{2}
{-8}{1}$.

\soln Let
\[
M = (A|I_3) = \left(\begin{array}{rrr|rrr} 1 & 4 & 5 & 1 & 0 & 0 \\
0 & 1 & -1 & 0 & 1 & 0 \\
-2 & 0 & -8 & 0 & 0 & 1 \end{array}\right).
\]
Then, row reduce $M$ to obtain the augmented matrix $(I_3|A^{-1})$.

\end{solution}
\end{exercise}


%%%%%%%%%%%%%%%%%%%%%%%%%%%%%%%%%%%%%%%%%%%%%%%%%%%%%%%%%%%%%%%%



\problemlabel

\exerciselabel{10}{3.7}\begin{exercise} \label{c4.9.6}
For which values of $a,b,c$ is the matrix
\[
A =\left(\begin{array}{rrr} 1 & a & b\\ 0 & 1 & c\\ 0 & 0 & 1
\end{array}\right)
\]
invertible?  Find $A\inv$ when it exists.

\begin{solution}

\ans
The matrix $A$ is invertible for any choice of $a$, $b$, and $c$, and
\[
A^{-1} = \left(\begin{array}{rrc} 1 & -a & -b + ac \\ 0 & 1 & -c 
\\ 0 & 0 & 1 \end{array}\right).
\]

\soln Theorem~\ref{invertequiv} states that a matrix is invertible if
it is row equivalent to $I_n$.  By row reducing the augmented matrix
$(A|I_3)$ as follows:
\[
\left(\begin{array}{rrr|rrr} 1 & a & b & 1 & 0 & 0 \\
0 & 1 & c & 0 & 1 & 0 \\ 0 & 0 & 1 & 0 & 0 & 1
\end{array}\right) \rightarrow \left(\begin{array}{rrr|rrc}
1 & 0 & 0 & 1 & -a & -b + ac \\ 0 & 1 & 0 & 0 & 1 & -c \\
0 & 0 & 1 & 0 & 0 & 1 \end{array}\right)
\]
we show that $A$ is invertible for any choice of $a$, $b$, and
$c$, and find a value for $A^{-1}$.

\end{solution}
\end{exercise}


%%%%%%%%%%%%%%%%%%%%%%%%%%%%%%%%%%%%%%%%%%%%%%%%%%%%%%%%%%%%%%%%


\newpage

\problemlabel

\exerciselabel{14}{3.7}\begin{exercise} \label{A3.7.1}
Let $A$ and $B$ be $3\times 3$ invertible matrices so that
\[
A^{-1} = \Matrix{1 & 0 & -1 \\ -1 & -1 &0 \\ 0 & 1 & -1}
\quad\text{and}\quad
B^{-1} = \Matrix{1 & 1 & 1 \\ 1 &1 &0 \\ 1 & 0 & 0}
\]
Without computing $A$ or $B$, determine the following:
\begin{enumeratea}
\item $\rank(A)$
\item The solution to 
\[
Bx=\Matrix{1 \\ 1 \\ 1}
\]
\item $(2BA)^{-1}$
\end{enumeratea} 



\begin{solution}
\begin{enumeratea}
\item $A$ is an invertible $3\times 3$ matrix, so $\rank(A)=3$.
\item The solution is 
\[
x = B^{-1}\Matrix{1 \\ 1 \\ 1}= \Matrix{1 & 1 & 1 \\ 1 &1 &0 \\ 1 & 0 & 0} \Matrix{1 \\ 1 \\ 1}=\Matrix{3 \\ 2 \\ 1}
\]
\item 
\[
(2BA)^{-1}=\frac{1}{2}A^{-1}B^{-1}=\frac{1}{2}
\Matrix{1 & 0 & -1 \\  -1 &-1 &0 \\  0 & 1 & -1}
\Matrix{1 & 1 & 1 \\  1 &1 &0 \\  1 & 0 & 0}
=\frac{1}{2}\Matrix{ 0 & 1 & 1 \\ -2 &-2 &-1 \\ 0 & 1 & 0}
\]
\end{enumeratea}
\end{solution}
\end{exercise}


%%%%%%%%%%%%%%%%%%%%%%%%%%%%%%%%%%%%%%%%%%%%%%%%%%%%%%%%%%%%%%%%



\problemlabel

\exerciselabel{4}{3.8}\begin{exercise} \label{c4.9.5}
Let $A$ be a $2\times 2$ matrix having integer entries.  Find a
condition on the entries of $A$ that guarantees that $A\inv$ has
integer entries.

\begin{solution}

\ans The matrix $A^{-1}$ has integer entries when $|ad - bc| = 1$.

\soln By \eqref{e:formAinv},
\[
A^{-1} = \frac{1}{ad-bc}\mattwo{d}{-b}{-c}{a}.
\]
So, in order for $A^{-1}$ to have integer entries, $\frac{1}{ad-bc}$
must be an integer.  Since $a$, $b$, $c$, and $d$ are integers,
$\frac{1}{ad - bc}$ is an integer only if $|ad - bc| = 1$.


\end{solution}
\end{exercise}


%%%%%%%%%%%%%%%%%%%%%%%%%%%%%%%%%%%%%%%%%%%%%%%%%%%%%%%%%%%%%%%%



\problemlabel

\exerciselabel{15}{3.7}\begin{exercise} \label{A3.7.2}
True or False: Determine whether the following statements are true or false, and explain your answer.
\begin{enumeratea}
\item The only $3\times 2$ matrix $A$ so that $Ax = 0 $ for all $x \in \R^2$ is $A=0$.

\item A system of 5 equations in 3 unknowns with the solution $x_1=0, x_2=-3,x_3=1$ must have infinitely many solutions.

\end{enumeratea}

\begin{solution}
\begin{enumeratea}
\item True:
\[
A=\Matrix{ A e_1 & A e_2}=\Matrix{0 & 0}=\Matrix{ 0 & 0 \\ 0 & 0 \\ 0 & 0}=0
\]
\item False, it may have a unique solution. For example, the system of equations
\[
\Matrix{ 1 & 0 & 0\\ 0 & 1 & 0 \\ 0 & 0 & 1 \\ 1 & 1 & 1 \\ 1 & 0 & 1} = \Matrix{ 0 \\ -3 \\ 1 \\ -2 \\ 1}
\]
has $x_1=0, x_2=-3,x_3=1$ as a unique solution.
\end{enumeratea}
\end{solution}

\end{exercise}


%%%%%%%%%%%%%%%%%%%%%%%%%%%%%%%%%%%%%%%%%%%%%%%%%%%%%%%%%%%%%%%%



\end{document}










