\documentclass{article}

% For preamble materials

\usepackage{pgf,tikz}
\usepackage{mathrsfs}
\usetikzlibrary{shapes,arrows}
\usepackage{framed}
\pgfplotsset{compat=1.16}

\def\fixnote#1{\begin{framed}{\textcolor{red}{Fix note: #1}}\end{framed}}  % Allows insertion of red notes about needed edits
%\def\fixnote#1{}

\def\detail#1{{\textcolor{blue}{Detail: #1}}}   

\graphicspath{
  {./}
  {proofs/}
}

%\pdfOnly{\renewcommand{\answer}[1][[yy]{\fbox{\hspace{1in}\rule[-.3\baselineskip]{0pt}{15pt}}}}


\newcommand{\N}{\mathbb N}
\newcommand{\W}{\mathbb W}
\newcommand{\C}{\mathbb C}
\newcommand{\Z}{\mathbb Z}
\newcommand{\Q}{\mathbb Q}
\newcommand{\R}{\mathbb R}




\title{Math 2568 Homework 12}
\author{\phantom{Dr. Golubitsky}}
\date{Due: Monday, November 25, 2019}

\makeatletter
\newlabel{T:orthocoord}{{9.1.3}{482}}
\newlabel{e:coordorthomat}{{9.1.1}{483}}
\newlabel{T:orthobasis}{{9.2.3}{490}}
\newlabel{T:popdata}{{4}{502}}
\newlabel{S:growthmodels}{{4.2}{185}}
\newlabel{c7.6.1}{{1}{502}}
\newlabel{E:nearestvector}{{9.2.4}{488}}
\newlabel{c7.9.1a}{{3}{509}}
\newlabel{c7.9.1e}{{7}{510}}
\newlabel{lem:orthprop}{{9.4.3}{508}}
\makeatother
\begin{document}

\maketitle


\problemlabel



\exerciselabel{4}{8.3}\begin{exercise} \label{c7.1.4}
Verify that ${\cal V}= \{p_1,p_2,p_3\}$ where
\[
p_1(t)=1+2t, \quad p_2(t)=t+2t^2, \AND p_3(t)=2-t^2,
\]
is a basis for the vector space of polynomials ${\cal P}_2$.
Let $p(t)=t$ and find $[p]_{\cal V}$.

\begin{solution}

\ans If $p(t) = t$, then
$[p]_{\cal V} = (\frac{4}{7}, -\frac{1}{7}, -\frac{2}{7})$.

\soln In order to verify that ${\cal V}$ is a basis for ${\cal P}_2$, first
show that the set $\{1,t,t^2\}$ is a basis for ${\cal P}_2$.  To prove
this, note that any polynomial in ${\cal P}_2$ can be written as
$p = \alpha_1 + \alpha_2t + \alpha_3t^2$, so the set spans ${\cal P}_2$.
Also, $0 = \alpha_1 + \alpha_2t + \alpha_3t^2$ if and only if
$\alpha_1 = \alpha_2 = \alpha_3 = 0$, so the set is linearly
independent.

\para The set $\{1,t,t^2\}$ has dimension 3 and is a basis for
${\cal P}_2$.  Therefore, any linearly independent set of three vectors
in ${\cal P}_2$ will span ${\cal P}_2$.  So we need only show that
${\cal V}$ is a linearly independent set, which we do by solving:
\[
\begin{array}{rcl}
0 & = & \alpha_1p_1(t) + \alpha_2p_2(t) + \alpha_3p_3(t) \\
& = & \alpha_1(1 + 2t) + \alpha_2(t + 2t^2) + \alpha_3(2 - t^2) \\
& = & (\alpha_1 + 2\alpha_3) + (2\alpha_1 + \alpha_2)t +
(2\alpha_2 - \alpha_3). \end{array}
\]
This equation is identically $0$ if
\[
\matthree{1}{2}{0}{2}{0}{1}{0}{2}{1}\vecthree{\alpha_1}{\alpha_2}
{\alpha_3} = 0.
\]
The only solution to this system is $\alpha_1 = \alpha_2 = \alpha_3
 = 0$, so the elements are linearly independent and ${\cal V}$ is
a basis for ${\cal P}_2$.

\para Let $p(t) = t$.  Then find this vector $[p]_{\cal V}$ by solving
$p(t) = \alpha_1p_1(t) + \alpha_2p_2(t) + \alpha_3p_3(t)$. 
That is,
\[ \vecthree{0}{1}{0} = \matthree{1}{0}{2}{2}{1}{0}{0}{2}{-1}
\vecthree{\alpha_1}{\alpha_2}{\alpha_3}. \]
Solve by substitution to obtain $\alpha_1 = \frac{4}{7}$, $\alpha_2
= -\frac{1}{7}$, and $\alpha_3 = -\frac{2}{7}$.

\end{solution}
\end{exercise}


%%%%%%%%%%%%%%%%%%%%%%%%%%%%%%%%%%%%%%%%%%%%%%%%%%%%%%%%%%%%%%%%



\matlabproblemlabel

\exerciselabel{5}{8.3}\begin{computerExercise} \label{c7.1.6}
Let
\[
w_1=(1,0,2), \quad w_2=(2,1,4), \AND w_3=(0,1,-1)
\]
be a basis for $\R^3$.  Find $[v]_{\cal W}$ where $v=(2,1,5)$.

\begin{solution}

\ans $[v]_W = (-2,2,-1)$.

\soln Use \Matlab to row reduce the augmented matrix
$(w_1^t|w_2^t|w_3^t|v)$, obtaining:
\begin{verbatim}
ans = 
     1     0     0    -2
     0     1     0     2
     0     0     1    -1
\end{verbatim}

\end{solution}
\end{computerExercise}


%%%%%%%%%%%%%%%%%%%%%%%%%%%%%%%%%%%%%%%%%%%%%%%%%%%%%%%%%%%%%%%%



\problemlabel

\exerciselabel{1}{9.1}\begin{exercise} \label{c7.4.1}
Find an orthonormal basis for the solutions to the linear equation
\[
2x_1-x_2+x_3=0.
\]

\begin{solution}

\ans The vectors $w_1 = \frac{1}{\sqrt{3}}(1,1,-1)$ and
$w_2 = \frac{1}{\sqrt{2}}(0,1,1)$ form an orthonormal basis
for the solution set.

\soln Find one vector which is a solution to the equation, for
example $(1,1,-1)$.  Then, divide the vector by its length, obtaining
the unit vector $w_1$.  By inspection, find a vector $v_2$ which
satisfies both the given equation and $w_1 \cdot v_2 = 0$.  Then set
$w_2 = \frac{1}{||v_2||}v_2$.

\end{solution}
\end{exercise}


%%%%%%%%%%%%%%%%%%%%%%%%%%%%%%%%%%%%%%%%%%%%%%%%%%%%%%%%%%%%%%%%



\problemlabel

\exerciselabel{2}{9.1}\begin{exercise} \label{c7.4.2}
\begin{itemize}
\item[(a)] Find the coordinates of the vector $v=(1,4)$ in the orthonormal
basis ${\cal V}$
\[
v_1 = \frac{1}{\sqrt{5}}(1,2) \AND v_2 = \frac{1}{\sqrt{5}}(2,-1).
\]
\item[(b)]  Let $A=\mattwo{1}{1}{2}{-3}$. Find $[A]_{\cal V}$.
\end{itemize}

\begin{solution}

(a) By Theorem~\ref{T:orthocoord}:
\[
[v]_{\cal V} = (v \cdot v_1, v \cdot v_2) = 
\frac{1}{\sqrt{5}}(9,-2).
\]

(b) By \eqref{e:coordorthomat}:
\[ [A]_{\cal V} = \mattwo{Av_1 \cdot v_1}{Av_2 \cdot v_1}
{Av_1 \cdot v_2}{Av_2 \cdot v_2} = \mattwo{-1}{3}{2}{-1}. \]

\end{solution}
\end{exercise}


%%%%%%%%%%%%%%%%%%%%%%%%%%%%%%%%%%%%%%%%%%%%%%%%%%%%%%%%%%%%%%%%



\problemlabel

\exerciselabel{1}{9.2}\begin{exercise} \label{c7.5.1}
Find an orthonormal basis of $\R^2$ by applying Gram-Schmidt
orthonormalization to the vectors $w_1=(3,4)$ and $w_2=(1,5)$.

\begin{solution}

\ans The vectors $v_1 = \frac{1}{5}(3,4)$ and $v_2 = \frac{1}{5}(-4,3)$
form an orthonormal basis for $\R^2$.

\soln Find these vectors using Gram-Schmidt orthonormalization
(Theorem~\ref{T:orthobasis}).  More
specifically, calculate $v_1$ and $v_2$ such that:
\[
\begin{array}{rcl}
v_1 & = & \frac{1}{||w_1||}w_1 = \frac{1}{5}(3,4). \\
v_2' & = & w_2 - (w_2 \cdot v_1)v_1 = (1,5) - \frac{23}{25}(3,4)
= \frac{1}{25}(-44,33). \\
v_2 & = & \frac{1}{||v_2'||}v_2' = \frac{5}{11}
\left(\frac{1}{25}(-44,33)\right) = \frac{1}{5}(-4,3).
\end{array}
\]

\end{solution}
\end{exercise}


%%%%%%%%%%%%%%%%%%%%%%%%%%%%%%%%%%%%%%%%%%%%%%%%%%%%%%%%%%%%%%%%



\problemlabel

\exerciselabel{2}{9.2}\begin{exercise} \label{c7.5.2}
Find an orthonormal basis of the plane $W\subset\R^3$ spanned by the
vectors $w_1=(1,2,3)$ and $w_2=(2,5,-1)$ by applying Gram-Schmidt
orthonormalization.

\begin{solution}

\ans The vectors $v_1 = \frac{1}{\sqrt{14}}(1,2,3)$ and
$v_2 = \frac{1}{\sqrt{4746}}(19,52,-41)$ form an orthonormal basis
for $W$.

\soln Use Gram-Schmidt orthonormalization
(Theorem~\ref{T:orthobasis}) to calculate $v_1$ and $v_2$ such that:
\[
\begin{array}{rcl}
v_1 & = & \frac{1}{||w_1||}w_1 = \frac{1}{\sqrt{14}}(1,2,3). \\
v_2' & = & w_2 - (w_2 \cdot v_1)v_1 = (2,5,-1) -
\frac{9}{14}(1,2,3) = \frac{1}{14}(19,52,-41). \\
v_2 & = & \frac{1}{||v_2'||}v_2' = \frac{14}{\sqrt{4746}}
\left(\frac{1}{14}(19,52,-41)\right) = \frac{1}{\sqrt{4746}}
(19,52,-41).
\end{array}
\]

\end{solution}
\end{exercise}


%%%%%%%%%%%%%%%%%%%%%%%%%%%%%%%%%%%%%%%%%%%%%%%%%%%%%%%%%%%%%%%%



\matlabproblemlabel

\exerciselabel{1}{9.3}\begin{computerExercise} \label{c7.6.1}
World population data for each decade of this century (except for 1910)
is given in Table~\ref{T:popdata}.  Assume that population growth is linear
$P=mT+b$ where time $T$ is measured in decades since the year 1900 and $P$ is
measured in billions of people.  This data can be recovered by typing
{\tt e9\_3\_po}.
\begin{itemize}
\item[(a)]  Find $m$ and $b$ to give the best linear fit to this data.
\item[(b)]  Use this linear approximation to the data to make predictions
of the world populations in the year 1910 and 2000.
\item[(c)]  Do you expect the prediction for the year 2000 to be high or low
or on target? Explain why by graphing the data with the best linear fit
superimposed and by using the differential equation population model
discussed in Section~\ref{S:growthmodels}.
\end{itemize}
\begin{table}[htb]
\begin{center}
\begin{tabular}{|c|c||c|c|}
\hline
Year & Population (in millions) & Year & Population (in millions)\\
\hline
1900 & 1625 & 1950 & 2516  \\
1910 & n.a. & 1960 & 3020 \\
1920 & 1813 & 1970 & 3698 \\
1930 & 1987 & 1980 & 4448 \\
1940 & 2213 & 1990 & 5292 \\
\hline
\end{tabular}
\caption{Twentieth Century World Population Data by Decades.}
\label{T:popdata}
\end{center}
\end{table}


\begin{solution}

(a) \ans The best linear fit to the data is obtained with $m \approx
0.4084$ and $b \approx 0.9603$, where $m$ and $b$ are measured in
billions.

\soln Create the matrix $A$ whose columns are $w_1$ and $w_2$.  Then use
\eqref{E:nearestvector} to compute the best values for $m$ and $b$.

(b) In 1910, $P \approx 408.4(1) + 960.3 = 1369$ million people.

\para In 2000, $P \approx 408.4(10) + 960.3 = 5044$ million people.

(c) \ans The prediction for 2000 is likely to be low.

\soln As shown in Figure~\ref{c7.6.1}, a linear approximation does not
fit the data points well.  To understand why, assume that population
change is governed by the differential equation:
\[
\frac{dP}{dT} = rP
\]
where $r$ is constant.  Then $\frac{d^2P}{dT^2} = r^2P > 0$.
Then the population curve is concave up, and a linear approximation
underestimates the population at the endpoints of the curve.

\begin{figure}[htb]
		\centerline{%
		\psfig{file=7-6-1.eps,width=3.0in}}
	\exercap{c7.6.1}
\end{figure}

\end{solution}
\end{computerExercise}


%%%%%%%%%%%%%%%%%%%%%%%%%%%%%%%%%%%%%%%%%%%%%%%%%%%%%%%%%%%%%%%%



\problemlabel

\exerciselabel{2}{9.4}\begin{exercise} \label{c7.7.2}
Let
\[
A=\mattwo{1}{2}{2}{-2}.
\]
Find the eigenvalues and eigenvectors of $A$ and verify that the eigenvectors
are orthogonal.

\begin{solution}

\ans The eigenvalues of $A$ are $\lambda_1 = 2$ and $\lambda_2 = -3$,
with respective eigenvectors $v_1 = (2,1)$ and $v_2 = (1,-2)$.

\soln Indeed, $v_1 \cdot v_2 = (2,1) \cdot (1,-2) = 0$, so the
eigenvectors are orthogonal.

\end{solution}
\end{exercise}


%%%%%%%%%%%%%%%%%%%%%%%%%%%%%%%%%%%%%%%%%%%%%%%%%%%%%%%%%%%%%%%%



\problemlabel

\noindent In Exercises~\ref{c7.9.1a} -- \ref{c7.9.1e} decide whether or not
the given matrix is orthogonal.


\exerciselabel{3}{9.4}\begin{exercise} \label{c7.9.1a}
$\left(\begin{array}{rr} 2 & 0\\ 0 & 1\end{array}\right)$.

\begin{solution}
\ans The matrix is not orthogonal.

\soln By Lemma~\ref{lem:orthprop}, a matrix
$A$ is orthogonal if and only if $A^tA = I_n$.
\[
\mattwo{2}{0}{0}{1}\mattwo{2}{0}{0}{1} = \mattwo{4}{0}{0}{1} \neq I_2.
\]

\end{solution}
\end{exercise}


%%%%%%%%%%%%%%%%%%%%%%%%%%%%%%%%%%%%%%%%%%%%%%%%%%%%%%%%%%%%%%%%



\problemlabel

\exerciselabel{6}{9.4}\begin{exercise} \label{c7.9.1d}
$\left(\begin{array}{rr} \cos(1) & -\sin(1)\\ \sin(1) & \cos(1)
\end{array}\right)$.

\begin{solution}
The matrix is orthogonal, since
\[
\mattwo{\cos(1)}{\sin(1)}{-\sin(1)}{\cos(1)}
\mattwo{\cos(1)}{-\sin(1)}{\sin(1)}{\cos(1)}
= I_2.
\]

\end{solution}
\end{exercise}


%%%%%%%%%%%%%%%%%%%%%%%%%%%%%%%%%%%%%%%%%%%%%%%%%%%%%%%%%%%%%%%%





\end{document}










