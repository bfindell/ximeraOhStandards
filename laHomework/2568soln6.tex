\documentclass{article}

% For preamble materials

\usepackage{pgf,tikz}
\usepackage{mathrsfs}
\usetikzlibrary{shapes,arrows}
\usepackage{framed}
\pgfplotsset{compat=1.16}

\def\fixnote#1{\begin{framed}{\textcolor{red}{Fix note: #1}}\end{framed}}  % Allows insertion of red notes about needed edits
%\def\fixnote#1{}

\def\detail#1{{\textcolor{blue}{Detail: #1}}}   

\graphicspath{
  {./}
  {proofs/}
}

%\pdfOnly{\renewcommand{\answer}[1][[yy]{\fbox{\hspace{1in}\rule[-.3\baselineskip]{0pt}{15pt}}}}


\newcommand{\N}{\mathbb N}
\newcommand{\W}{\mathbb W}
\newcommand{\C}{\mathbb C}
\newcommand{\Z}{\mathbb Z}
\newcommand{\Q}{\mathbb Q}
\newcommand{\R}{\mathbb R}




\newcommand{\MATLAB}{{\bf Matlab. }}

\title{Math 2568 Homework 6}
\author{\phantom{Dr. Golubitsky}}
\date{Due: Monday, October 7, 2019}


\makeatletter
\newlabel{c3.1.ba}{{1}{180}}
\newlabel{c3.1.bd}{{4}{182}}
\newlabel{E:uncoupleda}{{3}{199}}
\newlabel{E:uncoupledc}{{5}{199}}
\newlabel{lin2}{{4.3.2}{194}}
\newlabel{c3.5.5a}{{5}{208}}
\newlabel{c3.5.5d}{{8}{210}}
\newlabel{c6.4.1a}{{2}{229}}
\newlabel{c6.4.1d}{{5}{229}}
\newlabel{c4.10A.1a}{{1}{239}}
\newlabel{c4.10A.1d}{{4}{241}}
\newlabel{c5.1.4a}{{4}{271}}
\newlabel{c5.1.4f}{{10}{272}}
\newlabel{c5.1.5a}{{11}{272}}
\newlabel{c5.1.5e}{{15}{272}}
\newlabel{S:IVP&E}{{4.5}{211}}
\newlabel{e:solnODE}{{4.5.6}{213}}
\makeatother
\begin{document}
\maketitle


\problemlabel

\noindent Determine whether or not each of the given functions $x_1(t)$ and $x_2(t)$ is a solution  to the given differential equation.

\exerciselabel{2}{4.1}\begin{exercise}  \label{c3.1.bb}
ODE:\quad $\frac{\dps dx}{\dps dt} = x + e^t$.

Functions:\quad $x_1(t)=te^t \AND x_2(t)= 2e^t$.

\begin{solution}

\ans The function $x_1(t)$ is a solution to the differential equation;
the function $x_2(t)$ is not a solution.

\soln Compute
\[
\frac{d}{dt}(x_1) = \frac{d}{dt}(te^t) = te^t + e^t, \AND
\frac{dx_1}{dt} = x_1 + e^t = te^t + e^t.
\]
Thus, $x_1(t)$ is a solution to the differential equation.  Then compute
\[
\frac{d}{dt}(x_2) = \frac{d}{dt}(2e^t) = 2e^t, \AND
\frac{dx_2}{dt} = x_2 + e^t = 2e^t + e^t = 3e^t.
\]
Thus, $\frac{d}{dt}(x_2) \neq \frac{dx_2}{dt}$, so $x_2(t)$ is not a
solution to the differential equation.

\end{solution}
\end{exercise}


%%%%%%%%%%%%%%%%%%%%%%%%%%%%%%%%%%%%%%%%%%%%%%%%%%%%%%%%%%%%%%%%



\problemlabel



\exerciselabel{5}{4.1}\begin{exercise} \label{c3.1.1}
Solve the differential equation
\[
\frac{dx}{dt} = 2x,
\]
where $x(0)=1$.  At what time $t_1$ will $x(t_1)=2$?

\begin{solution}

\ans The solution for the given initial value problem is
$\dps x(t) = e^{2t}$.  From this equation, we find that $x(t_1) = 2$
when $t_1 = \frac{1}{2}\ln 2$.

\soln Note that $\frac{dx}{dt} = \lambda x$ implies $x(t) =
x_0e^{\lambda t}$.  In this case, $\frac{dx}{dt} = 2x$, so
$\lambda = 2$, and $x_0 = 1$, so $x(t) = e^{2t}$.  In order to
find $t_1$, substitute into the formula for $x(t)$, obtaining
$e^{2t_1} = 2$ and solve for $t_1$.

\end{solution}
\end{exercise}


%%%%%%%%%%%%%%%%%%%%%%%%%%%%%%%%%%%%%%%%%%%%%%%%%%%%%%%%%%%%%%%%



\problemlabel

\noindent Consider the uncoupled system of differential equations \eqref{lin2}. For each choice of $a$ and $d$, determine whether the origin is a saddle, source, or sink.

\exerciselabel{3}{4.3}\begin{exercise} \label{E:uncoupleda}
$a=1$ and $d=-1$.

\begin{solution}

\ans The origin is a saddle.

\soln This uncoupled system is of the form
\[
\begin{array}{rcl}
\frac{dx}{dt}(t) & = & Ax(t) \\
\frac{dy}{dt}(t) & = & Dy(t)\end{array}
\]
If $AD < 0$, then the origin is a saddle.  If $A < 0$ and
$D < 0$, then the origin is a sink.  If $A > 0$ and $D > 0$, then
the origin is a source.  In this case, $AD = -1 < 0$.

\end{solution}
\end{exercise}


%%%%%%%%%%%%%%%%%%%%%%%%%%%%%%%%%%%%%%%%%%%%%%%%%%%%%%%%%%%%%%%%

 

\problemlabel

\noindent Determine which of the function pairs $(x_1(t),y_1(t))$ and $(x_2(t),y_2(t))$ are solutions to the given system of ordinary differential equations.

\exerciselabel{6}{4.4}\begin{exercise} \label{c3.5.5b}
The ODE is:
\begin{eqnarray*}
\dot{x} & = & 2x - 3y  \\
\dot{y} & = & x - 2y.
\end{eqnarray*}
The pairs of functions are:
\[
(x_1(t),y_1(t)) = e^t(3,1)  \AND (x_2(t),y_2(t)) = (e^{-t},e^{-t}).
\]

\begin{solution}
\ans Both function pairs are solutions to the given system.

\soln To determine whether $(x_1(t),y_1(t)) = (3e^t, e^t)$ is
a solution to the system, compute the left hand sides of the equations:
\[
\frac{dx_1}{dt}(t) = \frac{d}{dt}(3e^t) = 3e^t \AND
\frac{dy_1}{dt}(t) = \frac{d}{dt}(e^t) = e^t.
\]
Then compute the right hand sides of the equations:
\[
2x_1(t) - 3y_1(t) = 2(3e^t) - 3e^t = 3e^t \AND
x_1(t) - 2y_1(t) = 3e^t - 2e^t = e^t.
\]
Since the left hand side of each equation equals the right hand side, the
equations are consistent, and the pair of functions is a solution.

\para Similarly, to determine whether $(x_2(t),y_2(t)) = (e^{-t},e^{-t})$
is a solution to the system, compute the left hand sides of the equations:
\[
\frac{dx_2}{dt}(t) = \frac{d}{dt}(e^{-t}) = -e^{-t} \AND
\frac{dy_2}{dt}(t) = \frac{d}{dt}(e^{-t}) = -e^{-t}.
\]
Then compute the right hand sides of the equations:
\[
2x_2(t) - 3y_2(t) = 2e^{-t} - 3e^{-t} = -e^{-t} \AND
x_2(t) - 2y_2(t) = e^{-t} - 2e^{-t} = -e^{-t}.
\]
Since the left hand side of each equation equals the right hand side, the
equations are consistent, and the pair of functions is a solution.


\end{solution}
\end{exercise}


%%%%%%%%%%%%%%%%%%%%%%%%%%%%%%%%%%%%%%%%%%%%%%%%%%%%%%%%%%%%%%%%



\problemlabel

\exerciselabel{1}{4.5}\begin{exercise} \label{c4.1.5}
Write the system of linear ordinary differential equations
\begin{eqnarray*}
\frac{dx_1}{dt}(t) & = & 4x_1(t) + 5x_2(t) \\
\frac{dx_2}{dt}(t) & = & 2x_1(t) - 3x_2(t)
\end{eqnarray*}
in matrix form.

\begin{solution}
\ans
\arraystart
\[
\left(\begin{array}{r} \dps\frac{dx_1}{dt}(t) \\ 
\dps\frac{dx_2}{dt}(t)\end{array}\right) =
\left(\begin{array}{rr} 4 & 5 \\ 2 & -3\end{array}\right)
\left(\begin{array}{r} x_1(t) \\ x_2(t)\end{array}\right)
\]
\arrayfinish

\end{solution}
\end{exercise}


%%%%%%%%%%%%%%%%%%%%%%%%%%%%%%%%%%%%%%%%%%%%%%%%%%%%%%%%%%%%%%%%




\problemlabel

\exerciselabel{4}{4.5}\begin{exercise} \label{c4.5.2}
Find a solution to
\[
\dot{X}(t)=CX(t)
\]
where
\[
C=\mattwo{1}{-1}{-1}{1}
\]
and
\[
X(0)=\vectwo{2}{1}.
\]
{\bf Hint:} Observe that
\[
\vectwo{1}{1} \AND \vectwo{1}{-1}
\]
are eigenvectors of $C$.

\begin{solution}

\ans 
\[
X(t) = \frac{3}{2}\vectwo{1}{1} + \frac{1}{2}e^{2t}\vectwo{1}{-1}.
\]

\soln Note that if $Cv = \lambda v$, then $X(t) = e^{\lambda t}v$ is a
solution to $\dot{X}(t) = CX(t)$.  Let
\[
v_1 = \vectwo{1}{1} \AND v_2 = \vectwo{1}{-1}.
\]
The eigenvalues corresponding to $v_1$ and $v_2$ are
$\lambda_1 = 0$ and $\lambda_2 = 2$.  This can be verified
by calculating $Cv_1 = 0$ and $Cv_2 = 2v_2$.
So,
\[
X(t) = \vectwo{1}{1} \AND X(t) = e^{2t}\vectwo{1}{-1}
\]
are both solutions to $\dot{X}(t) = CX$.  By the principle of
superposition,
\[
X(t) = \alpha\vectwo{1}{1} + \beta e^{2t}\vectwo{1}{-1}
\]
is also a solution.
Substitute the given the initial condition into the equation to obtain
\[
\vectwo{2}{1} = X(0) = \alpha\vectwo{1}{1} + \beta\vectwo{1}{-1}.
\]
Now, solve the linear system
\[
\begin{array}{rrrrr}
2 & = & \alpha & + & \beta \\
1 & = & \alpha & - & \beta \end{array}
\]
to find that $\alpha = \frac{3}{2}$ and $\beta = \frac{1}{2}$.


\end{solution}
\end{exercise}


%%%%%%%%%%%%%%%%%%%%%%%%%%%%%%%%%%%%%%%%%%%%%%%%%%%%%%%%%%%%%%%%



\problemlabel

\noindent Compute the determinant, trace, and characteristic polynomials for the given  $2\times 2$ matrix.

\exerciselabel{3}{4.6}\begin{exercise} \label{c6.4.1b}
$\mattwo{2}{13}{-1}{5}$.

\begin{solution}
The determinant of the matrix is $23$, the trace is $7$, and
the characteristic polynomial is $p(\lambda)=\lambda^2-7\lambda+23$.

\end{solution}
\end{exercise}


%%%%%%%%%%%%%%%%%%%%%%%%%%%%%%%%%%%%%%%%%%%%%%%%%%%%%%%%%%%%%%%%



\problemlabel

\noindent Find the solution to the system of differential equations $\dot{X} = CX$ satisfying $X(0)=X_0$.

\exerciselabel{3}{4.7}\begin{exercise}  \label{c4.10A.1c}
$C = \mattwo{-3}{2}{-2}{2}$ \AND $X_0 = \vectwo{-1}{3}$.

\begin{solution}
\ans The solution to $\dot{X} = CX$ satisfying this
initial condition is
\[
X(t) = \frac{7}{3}e^t\vectwo{1}{2} - \frac{5}{3}e^{-2t}\vectwo{2}{1}
= \frac{1}{3}\cvectwo{8e^t - 10e^{-2t}}{16e^t - 5e^{-2t}}.
\]

\soln First, find the eigenvalues of $C$, which are the roots of the
characteristic polynomial
\[
p_C(\lambda) = \lambda^2 - \trace(C)\lambda + \det(C) =
\lambda^2 + \lambda - 2 = (\lambda - 1)(\lambda + 2).
\]
So the eigenvalues are: $\lambda_1 = 1$ and $\lambda_2 = -2$.
To find the eigenvector associated to each eigenvalue, solve
the equation $(C - \lambda_jI_2)v_j = 0$ for $j = 1$ and $j = 2$.  Solve
\[
\left(\mattwo{-3}{2}{-2}{2} - \mattwo{1}{0}{0}{1}\right)v_1 =
\mattwo{-4}{2}{-2}{1}v_1 = 0
\]
to obtain $v_1 = (1,2)^t$ and solve
\[
\left(\mattwo{-3}{2}{-2}{2} + \mattwo{2}{0}{0}{2}\right)v_2 =
\mattwo{-1}{2}{-2}{4}v_2 = 0
\]
to obtain $v_2 = (2,1)^t$.  We can then write the general solution
\[
X(t) = \alpha_1e^{\lambda_1 t}v_1 + \alpha_2e^{\lambda_2 t}v_2
= \alpha_1e^t\vectwo{1}{2} + \alpha_2e^{-2t}\vectwo{2}{1}.
\]
From this formula, find $\alpha_1$ and $\alpha_2$ by solving
\[
\vectwo{-1}{3} = X(0) = \alpha_1\vectwo{1}{2} + \alpha_2\vectwo{2}{1} =
\cvectwo{\alpha_1 + 2\alpha_2}{2\alpha_1 + \alpha_2}.
\]
Solving the linear system
\[
\begin{array}{rrrrr}
\alpha_1 & + & 2\alpha_2 & = & -1 \\
2\alpha_1 & + & \alpha_2 & = & 3
\end{array}
\]
we obtain $\alpha_1 = \frac{7}{3}$ and $\alpha_2 = -\frac{5}{3}$
and find the solution to the differential equation.


\end{solution}
\end{exercise}


%%%%%%%%%%%%%%%%%%%%%%%%%%%%%%%%%%%%%%%%%%%%%%%%%%%%%%%%%%%%%%%%




\problemlabel

\noindent You are given a vector space $V$ and a subset $W$.  For each pair, decide whether or not $W$ is a subspace of $V$, and explain why.

\exerciselabel{5}{5.1}\begin{exercise} \label{c5.1.4b}
$V=\R^3$ and $W$ consists of vectors in $\R^3$
     that have a $1$ in their first component.

\begin{solution}
\ans $W$ is not a subspace of $V$.

\soln The subset $W$ is closed neither under addition nor under scalar
multiplication.  For example, let $w_1 = (1,4,2)$ and $w_2 = (1,-1,3)$
be elements of $W$.  Then,
\[
w_1 + w_2 = (1,4,2) + (1,-1,3) = (2,3,5)
\]
which is not an element of $W$.


\end{solution}
\end{exercise}


%%%%%%%%%%%%%%%%%%%%%%%%%%%%%%%%%%%%%%%%%%%%%%%%%%%%%%%%%%%%%%%%



\problemlabel

\noindent You are given a vector space $V$ and a subset $W$.  For each pair, decide whether or not $W$ is a subspace of $V$, and explain why.

\exerciselabel{7}{5.1}\begin{exercise} \label{c5.1.4c}
$V=\R^2$ and $W$ consists of vectors in $\R^2$
     for which the sum of the components is $0$.

\begin{solution}
$W$ is a subspace of $V$, since $W$ is closed under
addition and scalar multiplication.

\end{solution}
\end{exercise}


%%%%%%%%%%%%%%%%%%%%%%%%%%%%%%%%%%%%%%%%%%%%%%%%%%%%%%%%%%%%%%%%



\problemlabel

\noindent Which of the sets $S$ are subspaces?

\exerciselabel{13}{5.1}\begin{exercise} \label{c5.1.5c}
$S = \{(x,y)\in\R^2: (x,y) \mbox{ is on the line through }
(1,1) \mbox{ with slope } 1\}$.

\begin{solution}
The set $S$ is a subspace, since it is closed under
addition and scalar multiplication.

\end{solution}
\end{exercise}


%%%%%%%%%%%%%%%%%%%%%%%%%%%%%%%%%%%%%%%%%%%%%%%%%%%%%%%%%%%%%%%%



\problemlabel

\exerciselabel{20}{5.1}\begin{exercise} \label{c5.1.9}
Recall from equation~\eqref{e:solnODE} of Section~\ref{S:IVP&E}
that solutions to the system of differential equations
\[
\frac{dX}{dt} = \mattwo{-1}{3}{3}{-1} X
\]
are
\[
X(t) = \alpha e^{2t}\vectwo{1}{1} + \beta e^{-4t}\vectwo{1}{-1}.
\]
Use this formula for solutions to show that the set of solutions
to this system of differential equations is a vector subspace of
$(\CCone)^2$.

\begin{solution}

Let $V \subset (\CCone)^2$ be the set of solutions to \eqref{e:solnODE}.
The set is closed under both addition and scalar multiplication and
is a subspace.
To demonstrate, let
\[
x_1(t) = \alpha_1 e^{2t}\vectwo{1}{1} +
\beta_1 e^{-4t}\vectwo{1}{-1} \AND x_2(t) =
\alpha_2 e^{2t}\vectwo{1}{1} + \beta_2 e^{-4t}\vectwo{1}{-1}
\]
be elements of this set.  Adding $x_1$ and $x_2$ yields
\[
\left(\alpha_1 e^{2t}\vectwo{1}{1} +
\beta_1 e^{-4t}\vectwo{1}{-1}\right) +
\left(\alpha_2 e^{2t}\vectwo{1}{1}
+ \beta_2 e^{-4t}\vectwo{1}{-1}\right)
\]
\[
= (\alpha_1 + \alpha_2)e^{2t}\vectwo{1}{1} +
(\beta_1 + \beta_2) e^{-4t}\vectwo{1}{-1} \in V
\]
and multiplying $x_1$ by any real scalar $r$ yields
\[
rx_1 = r\left(\alpha_1 e^{2t}\vectwo{1}{1} +
\beta_1 e^{-4t}\vectwo{1}{-1}\right) = r\alpha_1 e^{2t}\vectwo{1}{1} +
r\beta_1 e^{-4t}\vectwo{1}{-1} \in V.
\]



\end{solution}
\end{exercise}


%%%%%%%%%%%%%%%%%%%%%%%%%%%%%%%%%%%%%%%%%%%%%%%%%%%%%%%%%%%%%%%%


\end{document}










