\documentclass{article}

% For preamble materials

\usepackage{pgf,tikz}
\usepackage{mathrsfs}
\usetikzlibrary{shapes,arrows}
\usepackage{framed}
\pgfplotsset{compat=1.16}

\def\fixnote#1{\begin{framed}{\textcolor{red}{Fix note: #1}}\end{framed}}  % Allows insertion of red notes about needed edits
%\def\fixnote#1{}

\def\detail#1{{\textcolor{blue}{Detail: #1}}}   

\graphicspath{
  {./}
  {proofs/}
}

%\pdfOnly{\renewcommand{\answer}[1][[yy]{\fbox{\hspace{1in}\rule[-.3\baselineskip]{0pt}{15pt}}}}


\newcommand{\N}{\mathbb N}
\newcommand{\W}{\mathbb W}
\newcommand{\C}{\mathbb C}
\newcommand{\Z}{\mathbb Z}
\newcommand{\Q}{\mathbb Q}
\newcommand{\R}{\mathbb R}




\newcommand{\MATLAB}{{\bf Matlab. }}

\title{Math 2568 Homework 7}
\author{\phantom{Dr. Golubitsky}}
\date{Due: Monday, October 14, 2019}

\makeatletter
\newlabel{c3.5.a01}{{1}{205}}
\newlabel{c6.4.1a}{{2}{229}}
\newlabel{c6.4.1d}{{5}{229}}
\newlabel{S:IVP&E}{{4.5}{211}}
\newlabel{e:solnODE}{{4.5.6}{213}}
\newlabel{c5.2.2a}{{5}{280}}
\newlabel{c5.2.2d}{{8}{281}}
\newlabel{basis=span+indep}{{5.5.3}{302}}
\newlabel{L:computerank}{{5.5.4}{303}}
\makeatother
\begin{document}
\maketitle


\matlabproblemlabel



\exerciselabel{1}{4.4}\begin{exercise} \label{c3.5.a01}
Choose the {\sf linear system} in {\pplane} and set $a=0$, $b=1$, and 
$c=-1$.  Then find values $d$ such that except for the origin itself all 
solutions appear to
\begin{itemize}
\item[(a)] spiral into the origin;
\item[(b)] spiral away from the origin;
\item[(c)] form circles around the origin;
\end{itemize}

\begin{solution}

\soln
\begin{enumeratea}
\item All trajectories converge on the origin when $D < 0$;

\item All trajectories move away from the origin when $D > 0$;

\item Trajectories form circles around the origin when $D = 0$.
\end{enumeratea}
\end{solution}
\end{exercise}


%%%%%%%%%%%%%%%%%%%%%%%%%%%%%%%%%%%%%%%%%%%%%%%%%%%%%%%%%%%%%%%%



\problemlabel

\exerciselabel{7}{4.5}\begin{exercise}  \label{c4.9.6A}
Suppose that $A$ is an $n\times n$ matrix with zero as an eigenvalue.
Show that $A$ is not invertible.  {\bf Hint:}  Assume that $A$ is invertible 
and compute $A\inv Av$ where $v$ is an eigenvector of $A$ corresponding to 
the zero eigenvalue.

\begin{solution}
\soln
Suppose that $A$ is an $n\times n$ matrix with zero eigenvalue.  We need to
prove that $A$ is not invertible.  Let $v\in\R^n$ be a nonzero 
eigenvector corresponding to the zero eigenvalue.  Therefore, $Av=0$. 
Suppose that $A\inv$ exists.  Then
\[
v = I_nv = A\inv Av = A\inv 0 = 0,
\]
which is a contradiction.  Therefore, $A$ is not invertible.

\end{solution}
\end{exercise}


%%%%%%%%%%%%%%%%%%%%%%%%%%%%%%%%%%%%%%%%%%%%%%%%%%%%%%%%%%%%%%%%

\newpage

\problemlabel

\noindent Compute the determinant, trace, and characteristic polynomials for the given  $2\times 2$ matrix.

\exerciselabel{4}{4.6}\begin{exercise} \label{c6.4.1c}
$\mattwo{1}{4}{1}{-1}$.

\begin{solution}
\soln
The determinant of the matrix is $-5$, the trace is $0$, and
the characteristic polynomial is $p(\lambda)=\lambda^2-5$.

\end{solution}
\end{exercise}


%%%%%%%%%%%%%%%%%%%%%%%%%%%%%%%%%%%%%%%%%%%%%%%%%%%%%%%%%%%%%%%%



\matlabproblemlabel

\exerciselabel{15}{4.6}\begin{exercise} \label{c7.8.7}
The \Matlab command {\tt eig} computes the eigenvalues
of matrices.  Use {\tt eig} to compute the eigenvalues of 
$A = \mattwoc{2.34}{-1.43}{\pi}{e}$.

\begin{solution}
\ans The eigenvalues of $A$ are $\lambda \approx 2.5291 \pm
2.1111i$.

\soln Enter the matrix $A$ into \Matlab and find its eigenvalues by typing
\begin{verbatim}
A = [2.34 -1.43; pi exp(1)];
eig(A)
\end{verbatim}

\end{solution}
\end{exercise}


%%%%%%%%%%%%%%%%%%%%%%%%%%%%%%%%%%%%%%%%%%%%%%%%%%%%%%%%%%%%%%%%



\problemlabel

\exerciselabel{1}{5.1}\begin{exercise} \label{c5.1.1}
Verify that the set $V_1$ consisting of all scalar multiples of
$(1,-1,-2)$ is a subspace of $\R^3$.

\begin{solution}

The set $V_1 \subset \R^3$ is a subspace.
The set $V_1$ contains all vectors $(a,-a,-2a)$,
where $a \in \R$.  We can show that it is closed under
vector addition, since
\[
a(1,-1,-2) + b(1,-1,-2) = (a + b)(1,-1,-2) \in V_1
\]
where $a$ and $b$ are scalars.  The set $V_1$ is closed under scalar
multiplication since
\[
b(a(1,-1,-2) = (ba)(1,-1,-2) \in V_1.
\]

\end{solution}
\end{exercise}


%%%%%%%%%%%%%%%%%%%%%%%%%%%%%%%%%%%%%%%%%%%%%%%%%%%%%%%%%%%%%%%%

\newpage

\problemlabel

\exerciselabel{17}{5.1}\begin{exercise} \label{c5.1.7a}
For which scalars $a,b,c$ do the solutions to the equation
\[
ax+by = c
\]
form a subspace of $\R^2$?

\begin{solution}

\ans Let $V$ be the subset of solutions $(x,y)$ to $ax + by = c$.
The subset $V$ is a subspace when $c = 0$ and is not a subspace
when $c \neq 0$. 

\soln Let $(x_1,y_1)$ and $(x_2,y_2)$ be elements of $V$.  Then
\[
a(x_1 + x_2) + b(y_1 + y_2) = (ax_1 + by_1) + (ax_2 + by_2) =
c + c = 2c.
\]
Thus $V$ is closed under addition only when $2c = c$, so $c = 0$.
Similarly, for any scalar $r$,
\[
r(ax_1 + by_1) = cr.
\]
So $V$ is closed under scalar multiplication only when $rc = c$ for
any scalar $r$.  Thus, $c = 0$.

\end{solution}
\end{exercise}


%%%%%%%%%%%%%%%%%%%%%%%%%%%%%%%%%%%%%%%%%%%%%%%%%%%%%%%%%%%%%%%%


\newpage

\problemlabel

\exerciselabel{20}{5.1}\begin{exercise} \label{c5.1.9}
Recall from equation~\eqref{e:solnODE} of Section~\ref{S:IVP&E}
that solutions to the system of differential equations
\[
\frac{dX}{dt} = \mattwo{-1}{3}{3}{-1} X
\]
are
\[
X(t) = \alpha e^{2t}\vectwo{1}{1} + \beta e^{-4t}\vectwo{1}{-1}.
\]
Use this formula for solutions to show that the set of solutions
to this system of differential equations is a vector subspace of
$(\CCone)^2$.

\begin{solution}

Let $V \subset (\CCone)^2$ be the set of solutions to \eqref{e:solnODE}.
The set is closed under both addition and scalar multiplication and
is a subspace.
To demonstrate, let
\[
x_1(t) = \alpha_1 e^{2t}\vectwo{1}{1} +
\beta_1 e^{-4t}\vectwo{1}{-1} \AND x_2(t) =
\alpha_2 e^{2t}\vectwo{1}{1} + \beta_2 e^{-4t}\vectwo{1}{-1}
\]
be elements of this set.  Adding $x_1$ and $x_2$ yields
\[
\left(\alpha_1 e^{2t}\vectwo{1}{1} +
\beta_1 e^{-4t}\vectwo{1}{-1}\right) +
\left(\alpha_2 e^{2t}\vectwo{1}{1}
+ \beta_2 e^{-4t}\vectwo{1}{-1}\right)
\]
\[
= (\alpha_1 + \alpha_2)e^{2t}\vectwo{1}{1} +
(\beta_1 + \beta_2) e^{-4t}\vectwo{1}{-1} \in V
\]
and multiplying $x_1$ by any real scalar $r$ yields
\[
rx_1 = r\left(\alpha_1 e^{2t}\vectwo{1}{1} +
\beta_1 e^{-4t}\vectwo{1}{-1}\right) = r\alpha_1 e^{2t}\vectwo{1}{1} +
r\beta_1 e^{-4t}\vectwo{1}{-1} \in V.
\]



\end{solution}
\end{exercise}


%%%%%%%%%%%%%%%%%%%%%%%%%%%%%%%%%%%%%%%%%%%%%%%%%%%%%%%%%%%%%%%%

\newpage

\problemlabel

\noindent Each of the given matrices is in reduced echelon form.  Write solutions of the corresponding homogeneous system of linear equations as a span of vectors.

\exerciselabel{5}{5.2}\begin{exercise} \label{c5.2.2a}
$A = \left(\begin{array}{rrrrr} 1 & 2 & 0 & 1 & 0 \\
	0 & 0 & 1 & 4 & 0 \\ 0 & 0 & 0 & 0 & 1 \end{array}\right)$.

\begin{solution}

\ans The subspace of solutions is spanned by the vectors
\[
(-2,1,0,0,0)^t \AND (-1,0,-4,1,0)^t.
\]

\soln Let $x = (x_1,\dots ,x_5)$ be a solution to $Ax = 0$.  All
solutions to this equation have the form
\[
\left(\begin{array}{r} x_1 \\ x_2 \\ x_3 \\ x_4 \\ x_5
\end{array}\right) = \left(\begin{array}{c} -2x_2 - x_4 \\ x_2 \\
-4x_4 \\ x_4 \\ 0 \end{array}\right) = x_2\left(\begin{array}{r}
-2 \\ 1 \\ 0 \\ 0 \\ 0 \end{array}\right) +
x_4\left(\begin{array}{r} -1 \\ 0 \\ -4 \\ 1 \\ 0
\end{array}\right).
\]

\end{solution}
\end{exercise}


%%%%%%%%%%%%%%%%%%%%%%%%%%%%%%%%%%%%%%%%%%%%%%%%%%%%%%%%%%%%%%%%



\problemlabel

\exerciselabel{22}{5.2}\begin{exercise} \label{c5.2.8b}
Let $V$ be a vector space and let $v,w\in V$ be vectors.  Show that
\[
\Span\{v,w\}=\Span\{v,w,v+3w\}.
\]

\begin{solution}
Every vector $x \in \Span\{v,w\}$ is of the form 
\[
x = av + bw = av + bw + 0(v + 3w) \in \Span\{v,w,v+3w\}.
\]
  Also, every vector $y \in \Span\{v,w,v+3w\}$ is of the form
\[
y = cv + dw + f(v + 3w) = (c + f)v + (d + 3f)w \in \Span\{v,w\}.
\]
Therefore, $\Span\{v,w\} = \Span\{v,w,v+3w\}$.

\end{solution}
\end{exercise}


%%%%%%%%%%%%%%%%%%%%%%%%%%%%%%%%%%%%%%%%%%%%%%%%%%%%%%%%%%%%%%%%

\newpage

\problemlabel

\exerciselabel{2}{5.4}\begin{exercise} \label{c5.4.2}
For which values of $b$ are the vectors $(1,b)$ and $(3,-1)$
linearly independent?

\begin{solution}

\ans The set is linearly independent if $b \neq -\frac{1}{3}$.

\soln Note that a set of two vectors is linearly dependent if one is
a multiple of the other.  So this set is dependent for any values of
$b$ at which
\[
(3,-1) = \alpha(1,b).
\]
When equality holds $\alpha = 3$.  Therefore, $b = -\frac{1}{3}$.  

\end{solution}
\end{exercise}


%%%%%%%%%%%%%%%%%%%%%%%%%%%%%%%%%%%%%%%%%%%%%%%%%%%%%%%%%%%%%%%%



\problemlabel

\exerciselabel{6}{5.4}\begin{exercise} \label{c5.4.5}
Show that the polynomials $p_1(t) = 2+t$, $p_2(t) = 1+t^2$, and
$p_3(t) = t-t^2$ are linearly independent vectors in the vector
space $\CCone$.

\begin{solution}

\ans The polynomials $p_1(t) = 2 + t$, $p_2(t) = 1 + t^2$, and $p_3(t) =
t - t^2$ are linearly independent in $\CCone$.  

\soln We can determine this
by noting that the polynomials are linearly dependent if there exists
a nonzero vector $r = (r_1,r_2,r_3)$ such that $r_1p_1 + r_2p_2 +
r_3p_3 = 0$.  It is convenient to represent each polynomial as a
vector $(a,b,c) = p(t) = a + bt + ct^2$.  Thus, $p_1(t) = (2,1,0)$, 
$p_2(t) = (1,0,1)$, and $p_3(t) = (0,1,-1)$.  Solve the homogeneous
system $Ar = 0$, where $A$ is the matrix whose columns are $p_1$,
$p_2$, and $p_3$, by row reduction.
\[ \matthree{2}{1}{0}{1}{0}{1}{0}{1}{-1} \longrightarrow
\matthree{1}{0}{0}{0}{1}{0}{0}{0}{1}. \]
Therefore, there are no nonzero values of $r$ for which $r_1p_1 + 
r_2p_2 + r_3p_3 = 0$, and the polynomials are linearly independent.


\end{solution}
\end{exercise}


%%%%%%%%%%%%%%%%%%%%%%%%%%%%%%%%%%%%%%%%%%%%%%%%%%%%%%%%%%%%%%%%



\problemlabel

\exerciselabel{1}{5.4}\begin{exercise} \label{c5.4.1}
Let $w$ be a vector in the vector space $V$.  Show that the sets of vectors
$\{w,0\}$ and $\{w,-w\}$ are linearly dependent.

\begin{solution}

To show that the set of vectors $\{w_1,w_2\}$ is linearly dependent,
show that there exist nonzero $a$ and $b$ such that
$aw_1 + bw_2 = 0$.  For the set $\{w,0\}$, if $a = 0$ and $b = 1$,
then $0w + 1(0) = 0$, so the set is linearly dependent.  For the
set $\{w,-w\}$, if $a = 1$ and $b = 1$, then
$w - w = 0$, so the set is linearly dependent.

\end{solution}
\end{exercise}


%%%%%%%%%%%%%%%%%%%%%%%%%%%%%%%%%%%%%%%%%%%%%%%%%%%%%%%%%%%%%%%%



\problemlabel

\exerciselabel{1}{5.5}\begin{exercise} \label{c5.5.1}
Show that ${\cal U}=\{u_1,u_2,u_3\}$ where
\[
u_1=(1,1,0) \quad u_2=(0,1,0) \quad u_3=(-1,0,1)
\]
is a basis for $\R^3$.

\begin{solution}

By Theorem~\ref{basis=span+indep},
${\cal U}$ is a basis for $\R^3$ if the vectors of ${\cal U}$ are
linearly independent and span $\R^3$.  By Lemma~\ref{L:computerank},
the dimension of ${\cal U}$ is equal to the rank of the matrix whose
rows are $u_1$, $u_2$, and $u_3$.  Row reduce this matrix:
\[
\matthree{1}{1}{0}{0}{1}{0}{-1}{0}{1} \longrightarrow
\matthree{1}{0}{0}{0}{1}{0}{0}{0}{1}.
\]
So $\dim({\cal U}) = 3 = \dim(\R^3)$, and we need now only show that
$u_1$, $u_2$, and $u_3$ are linearly independent, which we can do by
row reducing the matrix whose columns are the vectors of ${\cal U}$ as
follows:
\[
\matthree{1}{0}{-1}{1}{1}{0}{0}{0}{1} \longrightarrow
\matthree{1}{0}{0}{0}{1}{0}{0}{0}{1}.
\]
Therefore, there is no nonzero solution to the equation
${\cal U}r = 0$, so the vectors of ${\cal U}$ are linearly independent
and ${\cal U}$ is a basis for $\R^3$.

\end{solution}
\end{exercise}


%%%%%%%%%%%%%%%%%%%%%%%%%%%%%%%%%%%%%%%%%%%%%%%%%%%%%%%%%%%%%%%%


\end{document}










