\documentclass{article}

\usepackage{epsfig}

\graphicspath{
  {./}
  {figures/}
  {../laode}
  {../laode/figures}
}

\usepackage{epstopdf}
\epstopdfsetup{outdir=./}

\usepackage{morewrites}
\makeatletter
\newcommand\subfile[1]{%
\renewcommand{\input}[1]{}%
\begingroup\skip@preamble\otherinput{#1}\endgroup\par\vspace{\topsep}
\let\input\otherinput}
\makeatother

\newcommand{\EXER}{}
\newcommand{\includeexercises}{\EXER\directlua{dofile(kpse.find_file("exercises","lua"))}}

\newenvironment{computerExercise}{\begin{exercise}}{\end{exercise}}

%\newcounter{ccounter}
%\setcounter{ccounter}{1}
%\newcommand{\Chapter}[1]{\setcounter{chapter}{\arabic{ccounter}}\chapter{#1}\addtocounter{ccounter}{1}}

%\newcommand{\section}[1]{\section{#1}\setcounter{thm}{0}\setcounter{equation}{0}}

%\renewcommand{\theequation}{\arabic{chapter}.\arabic{section}.\arabic{equation}}
%\renewcommand{\thefigure}{\arabic{chapter}.\arabic{figure}}
%\renewcommand{\thetable}{\arabic{chapter}.\arabic{table}}

%\newcommand{\Sec}[2]{\section{#1}\markright{\arabic{ccounter}.\arabic{section}.#2}\setcounter{equation}{0}\setcounter{thm}{0}\setcounter{figure}{0}}
  
\newcommand{\Sec}[2]{\section{#1}}

\setcounter{secnumdepth}{2}
%\setcounter{secnumdepth}{1} 

%\newcounter{THM}
%\renewcommand{\theTHM}{\arabic{chapter}.\arabic{section}}

\newcommand{\trademark}{{R\!\!\!\!\!\bigcirc}}
%\newtheorem{exercise}{}

\newcommand{\dfield}{{\sf dfield9}}
\newcommand{\pplane}{{\sf pplane9}}
\newcommand{\PPLANE}{{\sf PPLANE9}}

% BADBAD: \newcommand{\Bbb}{\bf}

\newcommand{\R}{\mbox{$\Bbb{R}$}}
\newcommand{\C}{\mbox{$\Bbb{C}$}}
\newcommand{\Z}{\mbox{$\Bbb{Z}$}}
\newcommand{\N}{\mbox{$\Bbb{N}$}}
\newcommand{\D}{\mbox{{\bf D}}}
\usepackage{amssymb}
%\newcommand{\qed}{\hfill\mbox{\raggedright$\square$} \vspace{1ex}}
%\newcommand{\proof}{\noindent {\bf Proof:} \hspace{0.1in}}

\newcommand{\setmin}{\;\mbox{--}\;}
\newcommand{\Matlab}{{M\small{AT\-LAB}} }
\newcommand{\Matlabp}{{M\small{AT\-LAB}}}
\newcommand{\computer}{\Matlab Instructions}
\newcommand{\half}{\mbox{$\frac{1}{2}$}}
\newcommand{\compose}{\raisebox{.15ex}{\mbox{{\scriptsize$\circ$}}}}
\newcommand{\AND}{\quad\mbox{and}\quad}
\newcommand{\vect}[2]{\left(\begin{array}{c} #1_1 \\ \vdots \\
 #1_{#2}\end{array}\right)}
\newcommand{\mattwo}[4]{\left(\begin{array}{rr} #1 & #2\\ #3
&#4\end{array}\right)}
\newcommand{\mattwoc}[4]{\left(\begin{array}{cc} #1 & #2\\ #3
&#4\end{array}\right)}
\newcommand{\vectwo}[2]{\left(\begin{array}{r} #1 \\ #2\end{array}\right)}
\newcommand{\vectwoc}[2]{\left(\begin{array}{c} #1 \\ #2\end{array}\right)}

\newcommand{\ignore}[1]{}


\newcommand{\inv}{^{-1}}
\newcommand{\CC}{{\cal C}}
\newcommand{\CCone}{\CC^1}
\newcommand{\Span}{{\rm span}}
\newcommand{\rank}{{\rm rank}}
\newcommand{\trace}{{\rm tr}}
\newcommand{\RE}{{\rm Re}}
\newcommand{\IM}{{\rm Im}}
\newcommand{\nulls}{{\rm null\;space}}

\newcommand{\dps}{\displaystyle}
\newcommand{\arraystart}{\renewcommand{\arraystretch}{1.8}}
\newcommand{\arrayfinish}{\renewcommand{\arraystretch}{1.2}}
\newcommand{\Start}[1]{\vspace{0.08in}\noindent {\bf Section~\ref{#1}}}
\newcommand{\exer}[1]{\noindent {\bf \ref{#1}}}
\newcommand{\ans}{\textbf{Answer:} }
\newcommand{\matthree}[9]{\left(\begin{array}{rrr} #1 & #2 & #3 \\ #4 & #5 & #6
\\ #7 & #8 & #9\end{array}\right)}
\newcommand{\cvectwo}[2]{\left(\begin{array}{c} #1 \\ #2\end{array}\right)}
\newcommand{\cmatthree}[9]{\left(\begin{array}{ccc} #1 & #2 & #3 \\ #4 & #5 &
#6 \\ #7 & #8 & #9\end{array}\right)}
\newcommand{\vecthree}[3]{\left(\begin{array}{r} #1 \\ #2 \\
#3\end{array}\right)}
\newcommand{\cvecthree}[3]{\left(\begin{array}{c} #1 \\ #2 \\
#3\end{array}\right)}
\newcommand{\cmattwo}[4]{\left(\begin{array}{cc} #1 & #2\\ #3
&#4\end{array}\right)}

\newcommand{\Matrix}[1]{\ensuremath{\left(\begin{array}{rrrrrrrrrrrrrrrrrr} #1 \end{array}\right)}}

\newcommand{\Matrixc}[1]{\ensuremath{\left(\begin{array}{cccccccccccc} #1 \end{array}\right)}}



\renewcommand{\labelenumi}{\theenumi}
\newenvironment{enumeratea}%
{\begingroup
 \renewcommand{\theenumi}{\alph{enumi}}
 \renewcommand{\labelenumi}{(\theenumi)}
 \begin{enumerate}}
 {\end{enumerate}\endgroup}

\newcounter{help}
\renewcommand{\thehelp}{\thesection.\arabic{equation}}

%\newenvironment{equation*}%
%{\renewcommand\endequation{\eqno (\theequation)* $$}%
%   \begin{equation}}%
%   {\end{equation}\renewcommand\endequation{\eqno \@eqnnum
%$$\global\@ignoretrue}}

%\input{psfig.tex}

\author{Martin Golubitsky and Michael Dellnitz}

%\newenvironment{matlabEquation}%
%{\renewcommand\endequation{\eqno (\theequation*) $$}%
%   \begin{equation}}%
%   {\end{equation}\renewcommand\endequation{\eqno \@eqnnum
% $$\global\@ignoretrue}}

\newcommand{\soln}{\textbf{Solution:} }
\newcommand{\exercap}[1]{\centerline{Figure~\ref{#1}}}
\newcommand{\exercaptwo}[1]{\centerline{Figure~\ref{#1}a\hspace{2.1in}
Figure~\ref{#1}b}}
\newcommand{\exercapthree}[1]{\centerline{Figure~\ref{#1}a\hspace{1.2in}
Figure~\ref{#1}b\hspace{1.2in}Figure~\ref{#1}c}}
\newcommand{\para}{\hspace{0.4in}}

\usepackage{ifluatex}
\ifluatex
\ifcsname displaysolutions\endcsname%
\else
\renewenvironment{solution}{\suppress}{\endsuppress}
\fi
\else
\renewenvironment{solution}{}{}
\fi

%\ifxake
%\newenvironment{matlabEquation}{\begin{equation}}{\end{equation}}
%\else
\newenvironment{matlabEquation}%
{\let\oldtheequation\theequation\renewcommand{\theequation}{\oldtheequation*}\begin{equation}}%
  {\end{equation}\let\theequation\oldtheequation}
%\fi

\makeatother



%\newcommand{\MATLAB}{{\bf Matlab. }}

\title{Math 2568 Homework 8}
\author{\phantom{Dr. Golubitsky}}
\date{Due: Monday, October 21, 2019}

\makeatletter
\newlabel{c5.1.4a}{{4}{271}}
\newlabel{c5.1.4f}{{10}{272}}
\newlabel{c5.2.1a}{{1}{279}}
\newlabel{c5.2.1d}{{4}{280}}
\newlabel{c5.2.2a}{{5}{280}}
\newlabel{c5.2.2d}{{8}{281}}
\newlabel{c5.2.6a}{{16}{284}}
\newlabel{c5.2.6d}{{19}{285}}
\newlabel{c5.3.4a}{{6}{293}}
\newlabel{c5.3.4c}{{8}{293}}
\newlabel{c5.4.8a}{{9}{300}}
\newlabel{c5.4.8c}{{11}{300}}
\newlabel{extendindep}{{5.6.4}{313}}
\newlabel{A5.6.4}{{15}{322}}
\newlabel{A5.6.9}{{20}{322}}
\makeatother
\begin{document}
\maketitle


\problemlabel

\noindent You are given a vector space $V$ and a subset $W$.  For each pair, decide whether or not $W$ is a subspace of $V$, and explain why.

\exerciselabel{8}{5.1}\begin{exercise} \label{c5.1.4g}
$V=\CCone$ and $W$ consists of functions
     $x(t)\in\CCone$ satisfying $\int_{-2}^4x(t)dt =0$.

\begin{solution}
$W$ is a subspace of $V$, since $W$ is closed under
addition and scalar multiplication.

\end{solution}
\end{exercise}


%%%%%%%%%%%%%%%%%%%%%%%%%%%%%%%%%%%%%%%%%%%%%%%%%%%%%%%%%%%%%%%%



\problemlabel



\exerciselabel{16}{5.1}\begin{exercise} \label{c5.1.6}
Let $V$ be a vector space and let $W_1$ and $W_2$ be subspaces.
Show that the intersection $W_1\cap W_2$ is also a subspace of $V$.

\begin{solution}

The subset $W_1 \cap W_2$ is a subspace of $V$.  To show that this
subset is closed under addition and scalar multiplication, 
let $x$ and $y$ be vectors in $W_1 \cap W_2$.  It follows that
$x,y \in W_1$ and $x,y \in W_2$.  Therefore, by the
definition of a subspace, $x + y \in W_1$ and $x + y \in W_2$, so
$x + y \in W_1 \cap W_2$.  Also by definition, $rx \in W_1$ and
$rx \in W_2$, for some scalar $r$, so $rx \in W_1 \cap W_2$.

\end{solution}
\end{exercise}


%%%%%%%%%%%%%%%%%%%%%%%%%%%%%%%%%%%%%%%%%%%%%%%%%%%%%%%%%%%%%%%%



\problemlabel

\noindent A single equation in three variables is given.  For each equation write the subspace of solutions in $\R^3$ as the span of two vectors in $\R^3$.

\exerciselabel{2}{5.2}\begin{exercise} \label{c5.2.1b}
$x - y + 3z = 0$.

\begin{solution}
\ans The subspace of solutions can be spanned by the vectors 
$(1,1,0)^t$ and $(-3,0,1)^t$.

\soln All solutions to $x - y + 3z = 0$ can be written in the form
\[
\vecthree{x}{y}{z} = \cvecthree{y-3z}{y}{z}
= y\vecthree{1}{1}{0} + z\vecthree{-3}{0}{1}.
\]

\end{solution}
\end{exercise}


%%%%%%%%%%%%%%%%%%%%%%%%%%%%%%%%%%%%%%%%%%%%%%%%%%%%%%%%%%%%%%%%



\problemlabel

\noindent Each of the given matrices is in reduced echelon form.  Write solutions of the corresponding homogeneous system of linear equations as a span of vectors.

\exerciselabel{6}{5.2}\begin{exercise} \label{c5.2.2b}
$B = \left(\begin{array}{rrrr} 1 & 3 & 0 & 5 \\
	0 & 0 & 1 & 2 \end{array}\right)$.

\begin{solution}

\ans The subspace of solutions to $Bx = 0$ is spanned by the vectors
\[
(-3,1,0,0)^t \AND (-5,0,-2,1)^t.
\]

\soln Let $x = (x_1,x_2,x_3,x_4)$ be a solution to $Bx = 0$.  All
solutions to this equation have the form
\[
\left(\begin{array}{r} x_1 \\ x_2 \\ x_3 \\ x_4 \end{array}\right)
= \left(\begin{array}{c} -3x_2 - 5x_4 \\ x_2 \\ -2x_4 \\ x_4
\end{array}\right) = x_2\left(\begin{array}{r} -3 \\ 1 \\ 0 \\ 0
\end{array}\right) + x_4\left(\begin{array}{r} -5 \\ 0 \\ -2 \\ 1
\end{array}\right).
\]

\end{solution}
\end{exercise}


%%%%%%%%%%%%%%%%%%%%%%%%%%%%%%%%%%%%%%%%%%%%%%%%%%%%%%%%%%%%%%%%



\problemlabel

\noindent Let $W\subset\CCone$ be the subspace spanned by the two polynomials $x_1(t) = 1$ and $x_2(t)=t^2$.  For the given function $y(t)$ decide whether or not $y(t)$ is an element of $W$.  Furthermore, if $y(t)\in W$, determine whether the set $\{y(t),x_2(t)\}$ is a spanning set for $W$.

\exerciselabel{16}{5.2}\begin{exercise} \label{c5.2.6a}
$y(t) = 1-t^2$,

\begin{solution}

\ans The function $y(t) = 1 - t^2$ is an element of $W$ and the set
$\{y(t),x_2(t)\}$ is a spanning set for $W$.



\soln The space $W$ equals $\Span\{x_1(t),x_2(t)\}$ where $x_1(t)=1$ and 
$x_2(t)=t^2$.  To show that $y(t)$ is an element of $W$, let
$a = 1$ and $b = -1$, and compute
\[
ax_1(t) + bx_2(t) = x_1(t) - x_2(t) = 1 - t^2 = y(t). 
\]
To show that $\{y(t),x_2(t)\}$ is a spanning set for $W$, rewrite every
linear combination of $x_1(t)$ and $x_2(t)$ in terms of $y(t)$ and $x_2(t)$, 
as follows:
\[ 
ax_1(t) + bx_2(t) = a + bt^2 = a(1 - t^2) + (a + b)t^2
= ay(t) + (a + b)x_2(t). 
\]

\end{solution}
\end{exercise}


%%%%%%%%%%%%%%%%%%%%%%%%%%%%%%%%%%%%%%%%%%%%%%%%%%%%%%%%%%%%%%%%



\problemlabel

\exerciselabel{24}{5.2}\begin{exercise} \label{c5.2.10}
Let $Ax=b$ be a system of $m$ linear equations in $n$ unknowns,
and let $r=\rank(A)$ and $s=\rank(A|b)$.  Suppose that this system
has a unique solution.  What can you say about the relative
magnitudes of $m,n,r,s$?

\begin{solution}

\ans The relationship of the constants is $m \geq n = r = s$.

\soln The rank of matrix $A$ cannot be greater than the rank of matrix
$(A|b)$, since $(A|b)$ consists of $A$ plus one column.  The rank of $A$
is the number of pivots in the row reduced matrix.  $(A|b)$ can be row 
reduced through the same operations, and will have either the same number
of pivots as $A$ or, if there is a pivot in the last column, one more
pivot than $A$.  Since the system has a unique solution, it is consistent,
and therefore $(A|b)$ cannot have a pivot in the $(n + 1)^{st}$ column, so
$r = \rank(A) = \rank(A|b) = s$.

\para The set of solutions is parameterized by $n - r$ parameters,
where $n$ is the number of columns of $A$.  Since there is a unique
solution, the set of solutions is parameterized by $0$ parameters,
so $n = r$.

\para The number $m$ of rows of the matrix must be greater than or
equal to $n$ in order for the system to have a unique solution, since
there must be $n$ pivots, and each pivot must be in a separate row.



\end{solution}
\end{exercise}


%%%%%%%%%%%%%%%%%%%%%%%%%%%%%%%%%%%%%%%%%%%%%%%%%%%%%%%%%%%%%%%%



\matlabproblemlabel

\noindent Let $W\subset\R^5$ be the subspace spanned by the vectors \begin{matlabEquation}\label{MATLAB:65}      w_1=(2,0,-1,3,4),\quad w_2=(1,0,0,-1,2),\quad w_3=(0,1,0,0,-1). \end{matlabEquation} Use \Matlab to decide whether the given vectors are elements of $W$.

\exerciselabel{7}{5.3}\begin{computerExercise} \label{c5.3.4b}
$v_2=(-1,12,3,-14,-1)$.

\begin{solution}
\ans Vector $v_2$ is not an element of $W$.

\soln Create the augmented matrix {\tt aug2 = [A v2']}.  Row reducing
{\tt aug2} yields
\begin{verbatim}
ans =
     1     0     0     0
     0     1     0     0
     0     0     1     0
     0     0     0     1
     0     0     0     0
\end{verbatim}
There is a pivot point in the last column, so the linear system
$aw_1 + bw_2 + cw_3 = v_2$ is inconsistent.

\end{solution}
\end{computerExercise}


%%%%%%%%%%%%%%%%%%%%%%%%%%%%%%%%%%%%%%%%%%%%%%%%%%%%%%%%%%%%%%%%



\problemlabel

\exerciselabel{6}{5.4}\begin{exercise} \label{c5.4.5}
Show that the polynomials $p_1(t) = 2+t$, $p_2(t) = 1+t^2$, and
$p_3(t) = t-t^2$ are linearly independent vectors in the vector
space $\CCone$.

\begin{solution}

\ans The polynomials $p_1(t) = 2 + t$, $p_2(t) = 1 + t^2$, and $p_3(t) =
t - t^2$ are linearly independent in $\CCone$.  

\soln We can determine this
by noting that the polynomials are linearly dependent if there exists
a nonzero vector $r = (r_1,r_2,r_3)$ such that $r_1p_1 + r_2p_2 +
r_3p_3 = 0$.  It is convenient to represent each polynomial as a
vector $(a,b,c) = p(t) = a + bt + ct^2$.  Thus, $p_1(t) = (2,1,0)$, 
$p_2(t) = (1,0,1)$, and $p_3(t) = (0,1,-1)$.  Solve the homogeneous
system $Ar = 0$, where $A$ is the matrix whose columns are $p_1$,
$p_2$, and $p_3$, by row reduction.
\[ \matthree{2}{1}{0}{1}{0}{1}{0}{1}{-1} \longrightarrow
\matthree{1}{0}{0}{0}{1}{0}{0}{0}{1}. \]
Therefore, there are no nonzero values of $r$ for which $r_1p_1 + 
r_2p_2 + r_3p_3 = 0$, and the polynomials are linearly independent.


\end{solution}
\end{exercise}


%%%%%%%%%%%%%%%%%%%%%%%%%%%%%%%%%%%%%%%%%%%%%%%%%%%%%%%%%%%%%%%%



\problemlabel

\exerciselabel{8}{5.4}\begin{exercise} \label{c5.4.7}
Suppose that the three vectors $u_1,u_2,u_3\in\R^n$ are linearly
independent.  Show that the set
\[
\{u_1+u_2, u_2+u_3,u_3+u_1\}
\]
is also linearly independent.

\begin{solution}

To show that the vectors $u_1 + u_2$, $u_2 + u_3$ and $u_3 + u_1$
are linearly independent, we assume that there exist scalars $r_1$,
$r_2$, $r_3$ such that
\[ r_1(u_1 + u_2) + r_2(u_2 + u_3) + r_3(u_3 + u_1) = 0. \]
We then prove that $r_1 = r_2 = r_3 = 0$, as follows.
Use distribution to obtain
\[ (r_1 + r_3)u_1 + (r_1 + r_2)u_2 + (r_2 + r_3)u_3 = 0. \]
Since the set $\{u_1,u_2,u_3\}$ is linearly independent,
\[ \begin{array}{rrrrrcl}
r_1 & & & + & r_3 & = & 0 \\
r_1 & + & r_2 & & & = & 0 \\
& & r_2 & + & r_3 & = & 0. \end{array} \]
Solving this system yields $r_1 = r_2 = r_3 = 0$,
so the set $\{u_1 + u_2,u_2 + u_3,u_3 + u_1\}$ is linearly
independent.

\end{solution}
\end{exercise}


%%%%%%%%%%%%%%%%%%%%%%%%%%%%%%%%%%%%%%%%%%%%%%%%%%%%%%%%%%%%%%%%



\matlabproblemlabel

\noindent Determine whether the given sets of vectors are linearly independent or linearly dependent.

\exerciselabel{9}{5.4}\begin{computerExercise} \label{c5.4.8a}
\begin{matlabEquation}\label{MATLAB:67}
v_1 = (2,1,3,4) \quad v_2 = (-4,2,3,1) \quad v_3 = (2,9,21,22)
\end{matlabEquation}

\begin{solution}
\ans The set $\{v_1,v_2,v_3\}$ is linearly dependent.

\soln The set is linearly dependent if there exist scalars $r_1$, $r_2$,
and $r_3$ such that $r_1v_1 + r_2v_2 + r_3v_3 = 0$.  Create a matrix
{\tt A} whose columns are $v_1$, $v_2$ and $v_3$.  Then row reduce
{\tt A} to solve the homogeneous system $Ar = 0$.  Specifically, row
reducing the matrix {\tt A = [v1 v2 v3]} yields
\begin{verbatim}
ans =
     1     0     5
     0     1     2
     0     0     0
     0     0     0
\end{verbatim}
So $-5v_1 - 2v_2 + v_3 = 0$.

\end{solution}
\end{computerExercise}


%%%%%%%%%%%%%%%%%%%%%%%%%%%%%%%%%%%%%%%%%%%%%%%%%%%%%%%%%%%%%%%%

\vspace*{-0.1in}


\matlabproblemlabel

\noindent Determine whether the given sets of vectors are linearly independent or linearly dependent.

\exerciselabel{11}{5.4}\begin{computerExercise} \label{c5.4.8c}
\begin{matlabEquation}\label{MATLAB:69}
x_1 = (3,4,1,2,5) \quad x_2 = (-1,0,3,-2,1)\quad x_3 = (2,4,-3,0,2)
\end{matlabEquation}

\begin{solution}
\ans The set $\{x_1,x_2,x_3\}$ is linearly independent.

\soln The matrix {\tt A} associated to the set $\{x_1,x_2,x_3\}$ row
reduces to
\begin{verbatim}
ans =
     1     0     0
     0     1     0
     0     0     1
     0     0     0
     0     0     0
\end{verbatim}
In this case, there are no nonzero solutions to
$r_1x_1 + r_2x_2 + r_3x_3 = 0$.
 
\end{solution}
\end{computerExercise}


%%%%%%%%%%%%%%%%%%%%%%%%%%%%%%%%%%%%%%%%%%%%%%%%%%%%%%%%%%%%%%%%



\problemlabel

\exerciselabel{6}{5.5}\begin{exercise} \label{c5.5.6}
Let ${\cal P}_3$ be the vector space of polynomials of degree at
most three in one variable $t$.  Let $p(t)=t^3+a_2t^2+a_1t+a_0$ where
$a_0,a_1,a_2\in\R$ are fixed constants.  Show that
\[
\left\{ p, \frac{dp}{dt}, \frac{d^2p}{dt^2}, \frac{d^3p}{dt^3}\right\}
\]
is a basis for ${\cal P}_3$.

\begin{solution}

First, note that the dimension of ${\cal P}^3$ is $4$.  We can show
this by noting that the $4$ polynomials $b_1(t) = t^3$, $b_2(t) =
t^2$, $b_3(t) = t$, and $b_4(t) = 1$ are linearly independent and
span ${\cal P}^3$.  The dimension of a space is equal to the number of
linearly independent vectors in a spanning set for that space.
Therefore, $\{p, \frac{dp}{dt}, \frac{d^2p}{dt^2}, \frac{d^3p}{dt^3}\}$
is a basis if the polynomials are linearly independent.

\para The general polynomial of degree $3$ has the form $q(t) =
a_3t^3 + a_2t^2 + a_1t + a_0$.  We can identify $q(t)$ by the vector 
$(a_3,a_2,a_1,a_0)$.  Thus, the set
\[
\begin{array}{rcl}
\dps p(t) & = & t^3 + a_2t^2 + a_1t + a_0 \\
\dps \frac{dp}{dt}(t) & = & 3t^2 + 2a_2t + a_1 \\
\dps \frac{d^2p}{dt^2}(t) & = & 6t + 2a_2 \\
\dps \frac{d^3p}{dt^3}(t) & = & 6 \end{array}
\]
is identified with the matrix $A$, whose columns are 
$\frac{dp}{dt}$, $\frac{d^2p}{dt^2}$, and $\frac{d^3p}{dt^3}$:
\[ A = \left(\begin{array}{cccc} 1 & 0 & 0 & 0 \\
a_2 & 3 & 0 & 0 \\ a_1 & 2a_2 & 6 & 0 \\ a_0 & a_1 & 2a_2 & 6
\end{array}\right). \]
The matrix is lower triangular and therefore row reduces to $I_4$.  So
this set of polynomials is linearly independent and spans ${\cal P}^3$.

\end{solution}
\end{exercise}


%%%%%%%%%%%%%%%%%%%%%%%%%%%%%%%%%%%%%%%%%%%%%%%%%%%%%%%%%%%%%%%%

\newpage

\vspace*{-0.4in}

\matlabproblemlabel

\exerciselabel{11}{5.6}\begin{computerExercise} \label{c5.6.6}
Find a basis for the subspace of $\R^5$ spanned by
\begin{matlabEquation}\label{MATLAB:63}
\begin{array}{rcl}
u_1 & = & (1,1,0,0,1) \\
u_2 & = & (0,2,0,1,-1)  \\
u_3 & = & (0,-1,1,0,2)   \\
u_4 & = & (1,4,1,2,1)  \\
u_5 & = & (0,0,2,1,3).
\end{array}
\end{matlabEquation}

\begin{solution}

\ans The vectors $(1,0,0,-\frac{1}{2},\frac{3}{2})$, $(0,1,0,\frac{1}{2},
-\frac{1}{2})$, and $(0,0,1,\frac{1}{2},\frac{3}{2})$ form a basis
for the subspace spanned by $u_1, \dots ,u_5$.

\soln Row reduce the matrix {\tt M}, whose
rows are $u_1$, $u_2$, $u_3$, $u_4$ and $u_5$.  According to 
Lemma~\ref{extendindep}, the rows of the
reduced echelon matrix form a basis for $\{u_1,\dots ,u_5\}$.  The
command {\tt rref(M)} yields:
\begin{verbatim}
ans =
    1.0000         0         0   -0.5000    1.5000
         0    1.0000         0    0.5000   -0.5000
         0         0    1.0000    0.5000    1.5000
         0         0         0         0         0
         0         0         0         0         0
\end{verbatim}
\end{solution}
\end{computerExercise}


%%%%%%%%%%%%%%%%%%%%%%%%%%%%%%%%%%%%%%%%%%%%%%%%%%%%%%%%%%%%%%%%

\vspace*{-0.3in}

\problemlabel

In Exercises~\ref{A5.6.4}-\ref{A5.6.9} decide whether the statement is true or false, and explain your answer.

\exerciselabel{15}{5.6}\begin{exercise}  \label{A5.6.4}
Every set of three vectors in $\R^3$ is a basis for $\R^3$.
\begin{solution}
\ans False  

\soln
The vectors could be linearly independent. For example $\{e_1,e_2,e_1+e_2\}$ is not a basis for $\R^3$.
\end{solution}
\end{exercise}


%%%%%%%%%%%%%%%%%%%%%%%%%%%%%%%%%%%%%%%%%%%%%%%%%%%%%%%%%%%%%%%%

\vspace*{-0.2in}

\problemlabel

In Exercises~\ref{A5.6.4}-\ref{A5.6.9} decide whether the statement is true or false, and explain your answer.

\exerciselabel{18}{5.6}\begin{exercise}  \label{A5.6.7}
If $\{v_1,v_2,v_3\}$ is a basis for $\R^3$, the only subspaces of $R^3$ of dimension one are $\text{span}\{v_1\}, \text{span}\{v_2\}$, and $\text{span}\{v_3\}$. 
\begin{solution}
\ans False 
\soln For example, $\{e_1,e_2,e_3\}$ is a basis for $\R^3$ and $\text{span}\{e_1+e_2\}$ does not equal the $x$, $y$, or $z$-axes.
\end{solution}
\end{exercise}


%%%%%%%%%%%%%%%%%%%%%%%%%%%%%%%%%%%%%%%%%%%%%%%%%%%%%%%%%%%%%%%%



\end{document}










