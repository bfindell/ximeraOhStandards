\documentclass{article}

\usepackage{epsfig}

\graphicspath{
  {./}
  {figures/}
  {../laode}
  {../laode/figures}
}

\usepackage{epstopdf}
\epstopdfsetup{outdir=./}

\usepackage{morewrites}
\makeatletter
\newcommand\subfile[1]{%
\renewcommand{\input}[1]{}%
\begingroup\skip@preamble\otherinput{#1}\endgroup\par\vspace{\topsep}
\let\input\otherinput}
\makeatother

\newcommand{\EXER}{}
\newcommand{\includeexercises}{\EXER\directlua{dofile(kpse.find_file("exercises","lua"))}}

\newenvironment{computerExercise}{\begin{exercise}}{\end{exercise}}

%\newcounter{ccounter}
%\setcounter{ccounter}{1}
%\newcommand{\Chapter}[1]{\setcounter{chapter}{\arabic{ccounter}}\chapter{#1}\addtocounter{ccounter}{1}}

%\newcommand{\section}[1]{\section{#1}\setcounter{thm}{0}\setcounter{equation}{0}}

%\renewcommand{\theequation}{\arabic{chapter}.\arabic{section}.\arabic{equation}}
%\renewcommand{\thefigure}{\arabic{chapter}.\arabic{figure}}
%\renewcommand{\thetable}{\arabic{chapter}.\arabic{table}}

%\newcommand{\Sec}[2]{\section{#1}\markright{\arabic{ccounter}.\arabic{section}.#2}\setcounter{equation}{0}\setcounter{thm}{0}\setcounter{figure}{0}}
  
\newcommand{\Sec}[2]{\section{#1}}

\setcounter{secnumdepth}{2}
%\setcounter{secnumdepth}{1} 

%\newcounter{THM}
%\renewcommand{\theTHM}{\arabic{chapter}.\arabic{section}}

\newcommand{\trademark}{{R\!\!\!\!\!\bigcirc}}
%\newtheorem{exercise}{}

\newcommand{\dfield}{{\sf dfield9}}
\newcommand{\pplane}{{\sf pplane9}}
\newcommand{\PPLANE}{{\sf PPLANE9}}

% BADBAD: \newcommand{\Bbb}{\bf}

\newcommand{\R}{\mbox{$\Bbb{R}$}}
\newcommand{\C}{\mbox{$\Bbb{C}$}}
\newcommand{\Z}{\mbox{$\Bbb{Z}$}}
\newcommand{\N}{\mbox{$\Bbb{N}$}}
\newcommand{\D}{\mbox{{\bf D}}}
\usepackage{amssymb}
%\newcommand{\qed}{\hfill\mbox{\raggedright$\square$} \vspace{1ex}}
%\newcommand{\proof}{\noindent {\bf Proof:} \hspace{0.1in}}

\newcommand{\setmin}{\;\mbox{--}\;}
\newcommand{\Matlab}{{M\small{AT\-LAB}} }
\newcommand{\Matlabp}{{M\small{AT\-LAB}}}
\newcommand{\computer}{\Matlab Instructions}
\newcommand{\half}{\mbox{$\frac{1}{2}$}}
\newcommand{\compose}{\raisebox{.15ex}{\mbox{{\scriptsize$\circ$}}}}
\newcommand{\AND}{\quad\mbox{and}\quad}
\newcommand{\vect}[2]{\left(\begin{array}{c} #1_1 \\ \vdots \\
 #1_{#2}\end{array}\right)}
\newcommand{\mattwo}[4]{\left(\begin{array}{rr} #1 & #2\\ #3
&#4\end{array}\right)}
\newcommand{\mattwoc}[4]{\left(\begin{array}{cc} #1 & #2\\ #3
&#4\end{array}\right)}
\newcommand{\vectwo}[2]{\left(\begin{array}{r} #1 \\ #2\end{array}\right)}
\newcommand{\vectwoc}[2]{\left(\begin{array}{c} #1 \\ #2\end{array}\right)}

\newcommand{\ignore}[1]{}


\newcommand{\inv}{^{-1}}
\newcommand{\CC}{{\cal C}}
\newcommand{\CCone}{\CC^1}
\newcommand{\Span}{{\rm span}}
\newcommand{\rank}{{\rm rank}}
\newcommand{\trace}{{\rm tr}}
\newcommand{\RE}{{\rm Re}}
\newcommand{\IM}{{\rm Im}}
\newcommand{\nulls}{{\rm null\;space}}

\newcommand{\dps}{\displaystyle}
\newcommand{\arraystart}{\renewcommand{\arraystretch}{1.8}}
\newcommand{\arrayfinish}{\renewcommand{\arraystretch}{1.2}}
\newcommand{\Start}[1]{\vspace{0.08in}\noindent {\bf Section~\ref{#1}}}
\newcommand{\exer}[1]{\noindent {\bf \ref{#1}}}
\newcommand{\ans}{\textbf{Answer:} }
\newcommand{\matthree}[9]{\left(\begin{array}{rrr} #1 & #2 & #3 \\ #4 & #5 & #6
\\ #7 & #8 & #9\end{array}\right)}
\newcommand{\cvectwo}[2]{\left(\begin{array}{c} #1 \\ #2\end{array}\right)}
\newcommand{\cmatthree}[9]{\left(\begin{array}{ccc} #1 & #2 & #3 \\ #4 & #5 &
#6 \\ #7 & #8 & #9\end{array}\right)}
\newcommand{\vecthree}[3]{\left(\begin{array}{r} #1 \\ #2 \\
#3\end{array}\right)}
\newcommand{\cvecthree}[3]{\left(\begin{array}{c} #1 \\ #2 \\
#3\end{array}\right)}
\newcommand{\cmattwo}[4]{\left(\begin{array}{cc} #1 & #2\\ #3
&#4\end{array}\right)}

\newcommand{\Matrix}[1]{\ensuremath{\left(\begin{array}{rrrrrrrrrrrrrrrrrr} #1 \end{array}\right)}}

\newcommand{\Matrixc}[1]{\ensuremath{\left(\begin{array}{cccccccccccc} #1 \end{array}\right)}}



\renewcommand{\labelenumi}{\theenumi}
\newenvironment{enumeratea}%
{\begingroup
 \renewcommand{\theenumi}{\alph{enumi}}
 \renewcommand{\labelenumi}{(\theenumi)}
 \begin{enumerate}}
 {\end{enumerate}\endgroup}

\newcounter{help}
\renewcommand{\thehelp}{\thesection.\arabic{equation}}

%\newenvironment{equation*}%
%{\renewcommand\endequation{\eqno (\theequation)* $$}%
%   \begin{equation}}%
%   {\end{equation}\renewcommand\endequation{\eqno \@eqnnum
%$$\global\@ignoretrue}}

%\input{psfig.tex}

\author{Martin Golubitsky and Michael Dellnitz}

%\newenvironment{matlabEquation}%
%{\renewcommand\endequation{\eqno (\theequation*) $$}%
%   \begin{equation}}%
%   {\end{equation}\renewcommand\endequation{\eqno \@eqnnum
% $$\global\@ignoretrue}}

\newcommand{\soln}{\textbf{Solution:} }
\newcommand{\exercap}[1]{\centerline{Figure~\ref{#1}}}
\newcommand{\exercaptwo}[1]{\centerline{Figure~\ref{#1}a\hspace{2.1in}
Figure~\ref{#1}b}}
\newcommand{\exercapthree}[1]{\centerline{Figure~\ref{#1}a\hspace{1.2in}
Figure~\ref{#1}b\hspace{1.2in}Figure~\ref{#1}c}}
\newcommand{\para}{\hspace{0.4in}}

\usepackage{ifluatex}
\ifluatex
\ifcsname displaysolutions\endcsname%
\else
\renewenvironment{solution}{\suppress}{\endsuppress}
\fi
\else
\renewenvironment{solution}{}{}
\fi

%\ifxake
%\newenvironment{matlabEquation}{\begin{equation}}{\end{equation}}
%\else
\newenvironment{matlabEquation}%
{\let\oldtheequation\theequation\renewcommand{\theequation}{\oldtheequation*}\begin{equation}}%
  {\end{equation}\let\theequation\oldtheequation}
%\fi

\makeatother



%\newcommand{\MATLAB}{{\bf Matlab. }}

\title{Math 2568 Homework 9}
\author{\phantom{Dr. Golubitsky}}
\date{Due: Monday, October 28, 2019}

\makeatletter
\newlabel{c6.1.06a}{{5}{327}}
\newlabel{c6.1.06d}{{8}{328}}
\newlabel{Ex.1.06}{{6.1.8}{327}}
\newlabel{c6.1.2a}{{13}{329}}
\newlabel{c6.1.2b}{{14}{329}}
\newlabel{E:c6.1.2}{{6.1.11}{329}}
\newlabel{c6.1.3a}{{15}{329}}
\newlabel{c6.1.3c}{{17}{330}}
\newlabel{c6.6.2a}{{4}{342}}
\newlabel{c6.6.2d}{{7}{344}}
\newlabel{E:CC1}{{6.2.3}{333}}
\newlabel{E:CC2}{{6.2.4}{334}}
\newlabel{e:exp1eva}{{6.2.16}{339}}
\newlabel{P:invprod}{{3.7.3}{154}}
\newlabel{c6.5.3a}{{3}{351}}
\newlabel{c6.5.3b}{{4}{351}}
\newlabel{a6.3.1_C}{{6.3.5}{353}}
\newlabel{a6.3.1_B}{{6.3.6}{354}}
\newlabel{E:sssa}{{10}{361}}
\newlabel{E:sssd}{{13}{362}}
\newlabel{A6.4.1}{{16}{364}}
\newlabel{A6.4.3}{{18}{364}}
\newlabel{A6.4-a}{{6.4.5}{364}}
\newlabel{E:RD2}{{6.2.2}{333}}
\makeatother
\begin{document}
\maketitle


\problemlabel

\noindent Consider the system of differential equations \begin{equation} \label{Ex.1.06} \begin{array}{rcr} \frac{dx}{dt}  & = & x-y \\ \frac{dy}{dt}  & = & -x+y. \end{array} \end{equation}

\exerciselabel{5}{6.1}\begin{exercise} \label{c6.1.06a}
The eigenvalues of the coefficient matrix of \eqref{Ex.1.06} are $0$ and $2$.
Find the associated eigenvectors.

\begin{solution}

\ans The eigenvector associated to $\lambda_1 = 0$ is $v_1 = (1,1)^t$,
and the eigenvector associated to $\lambda_2 = 2$ is $v_2 = (1,-1)^t$.

\soln Solve the systems
\[
\mattwo{1}{-1}{-1}{1}v_1 = 0 \AND \mattwo{1}{-1}{-1}{0}v_2 = 2v_2.
\]

\end{solution}
\end{exercise}


%%%%%%%%%%%%%%%%%%%%%%%%%%%%%%%%%%%%%%%%%%%%%%%%%%%%%%%%%%%%%%%%



\problemlabel

\noindent Consider the system of differential equations \begin{equation} \label{Ex.1.06} \begin{array}{rcr} \frac{dx}{dt}  & = & x-y \\ \frac{dy}{dt}  & = & -x+y. \end{array} \end{equation}

\exerciselabel{7}{6.1}\begin{exercise} \label{c6.1.06c}
Find the solution to \eqref{Ex.1.06} satisfying initial conditions
$X(0)=(2,6)^t$.

\begin{solution}
The solution with initial condition $X(0) = (2,6)^t$ is 
\[
X(t) = 4\vectwo{1}{1} - 2e^{2t}\vectwo{1}{-1}.
\]


\end{solution}
\end{exercise}


%%%%%%%%%%%%%%%%%%%%%%%%%%%%%%%%%%%%%%%%%%%%%%%%%%%%%%%%%%%%%%%%



\problemlabel

\noindent Consider the system of differential equations \begin{equation}  \label{E:c6.1.2} \begin{array}{rcl} \frac{dx}{dt} & = & -2x+7y \\ \frac{dy}{dt} & = &  5y, \end{array} \end{equation}

\exerciselabel{13}{6.1}\begin{exercise} \label{c6.1.2a}
Find a solution to \eqref{E:c6.1.2}
satisfying the initial condition $(x(0),y(0)) = (1,0)$.

\begin{solution}

\ans If $(x(0),y(0)) = (1,0) = X_2(0)$, then
\[
(x(t),y(t)) = e^{-2t}(1,0).
\]

\soln The general solution to the system is
\[
X(t) = r_1e^{5t}(1,1) + r_2e^{-2t}(1,0).
\]
To obtain this solution, first rewrite the system of differential
equations as
\[
\frac{dX}{dt} = CX = \mattwo{-2}{7}{5}{0}\vectwo{x}{y}.
\]
By inspection of $C$, $(1,1)^t$ and $(1,0)^t$ are eigenvectors with
eigenvalues $5$ and $-2$ respectively.  Therefore:
\[
X_1(t) = e^{5t}\vectwo{1}{1} \AND X_2(t) = e^{-2t}\vectwo{1}{0}
\]
are solutions to the differential equation.
The initial values $X_1(0) = (1,1)$ and $X_2(0) = (1,0)$ are linearly
independent, so the general solution is valid.
To find $r_1$ and $r_2$, evaluate
\[
X(0) = (x(0),y(0)) = r_1(1,1) + r_2(1,0) = (r_1 + r_2,r_1).
\]

\end{solution}
\end{exercise}


%%%%%%%%%%%%%%%%%%%%%%%%%%%%%%%%%%%%%%%%%%%%%%%%%%%%%%%%%%%%%%%%



\problemlabel

\noindent Consider the matrix \[ C = \left(\begin{array}{rrr} -1 & -10 & -6\\  0 & 4  & 3 \\  0  & -14  & -9 	\end{array}\right). \]

\exerciselabel{15}{6.1}\begin{exercise} \label{c6.1.3a}
Verify that
\[
v_1 = \left(\begin{array}{r} 1 \\ 0\\ 0\end{array}\right) \qquad
v_2 = \left(\begin{array}{r} 2 \\ -1\\ 2\end{array}\right) \quad \AND \quad
v_3 = \left(\begin{array}{r} 6 \\ -3\\ 7\end{array}\right)
\]
are eigenvectors of $C$ and find the associated eigenvalues.

\begin{solution}

\ans The vector $v_1 = (1,0,0)^t$ is an eigenvector with associated
eigenvalue $\lambda_1 = -1$.  The vector $v_2 = (2,-1,2)^t$ is an
eigenvector with associated eigenvalue $\lambda_2 = -2$.  The vector
$v_3 = (6,-3,7)^t$ is an eigenvector with associated eigenvalue
$\lambda_3 = -3$.

\soln To verify, compute
\[
\begin{array}{l}
\matthree{-1}{-10}{-6}{0}{4}{3}{0}{-14}{-9}\vecthree{1}{0}{0} =
\vecthree{-1}{0}{0} = -1\vecthree{1}{0}{0}. \\
\matthree{-1}{-10}{-6}{0}{4}{3}{0}{-14}{-9}\vecthree{2}{-1}{2} =
\vecthree{-4}{2}{-4} = -2\vecthree{2}{-1}{2}. \\
\matthree{-1}{-10}{-6}{0}{4}{3}{0}{-14}{-9}\vecthree{6}{-3}{7} =
\vecthree{-18}{9}{-21} = -3\vecthree{6}{-3}{7}.
\end{array}
\]

\end{solution}
\end{exercise}


%%%%%%%%%%%%%%%%%%%%%%%%%%%%%%%%%%%%%%%%%%%%%%%%%%%%%%%%%%%%%%%%



\problemlabel

Compute the general solution for the given system of differential equations.

\exerciselabel{4}{6.2}\begin{exercise}  \label{c6.6.2a}
$\dps\frac{dX}{dt} = \mattwo{-1}{-4}{2}{3} X$.

\begin{solution}

\ans The general solution to the differential equation is
\[
X(t) =
\alpha_1\cvectwo{2e^t\cos(2t)}{e^t(\sin(2t) - \cos(2t))} +
\alpha_2\cvectwo{2e^t\sin(2t)}{-e^t(\sin(2t) + \cos(2t))}.
\]

\soln First, find the eigenvalues of $C$, which are the roots of the
characteristic polynomial
\[
p_C(\lambda) = \lambda^2 - 2\lambda + 5.
\]
The eigenvalues are $\lambda_1 = 1 + 2i$ and $\lambda_2 = 1 - 2i$.  Then,
find the eigenvector associated to $\lambda_1$ by solving the equation
\[
(C - \lambda_1I_2)v_1 =
\left(\mattwo{-1}{-4}{2}{3} - \cmattwo{1 + 2i}{0}{0}{1 + 2i}\right)v_1
= \cmattwo{-2 - 2i}{-4}{2}{2 - 2i}v_1 = 0.
\]
Solve this equation to find that
\[
v_1 = \cvectwo{2}{-1 - i} = \vectwo{2}{-1} + i\vectwo{0}{-1}
\]
is an eigenvector of $C$.  Since the eigenvalues of $C$ are complex, we
can find the general solution using \eqref{E:CC1} and \eqref{E:CC2}.  In this
case, since $\lambda_1 = 1 + 2i$ is an eigenvalue, let $\sigma = 1$ and
let $\tau = 2$.  Then $v_1 = v + iw$, where $v = (2,-1)^t$ and
$w = (0,-1)^t$.  By \eqref{E:CC1},
\[
X_1(t) = e^{\sigma t}(\cos(\tau t)v - \sin(\tau t)w) \AND
X_2(t) = e^{\sigma t}(\sin(\tau t)v + \cos(\tau t)w)
\]
are solutions to the differential equation.  In this case,
\[
\begin{array}{rcl}
X_1(t) & = & e^t\left(\cos(2t)\vectwo{2}{-1} -
\sin(2t)\vectwo{0}{-1}\right)
= e^t\cvectwo{2\cos(2t)}{\sin(2t) - \cos(2t)}. \\
X_2(t) & = & e^t\left(\sin(2t)\vectwo{2}{-1} +
\cos(2t)\vectwo{0}{-1}\right)
= e^t\cvectwo{2\sin(2t)}{-\sin(2t) - \cos(2t)}.
\end{array}
\]
The general solution consists of all linear combinations
$X(t) = \alpha_1X_1(t) + \alpha_2X_2(t)$.


\end{solution}
\end{exercise}


%%%%%%%%%%%%%%%%%%%%%%%%%%%%%%%%%%%%%%%%%%%%%%%%%%%%%%%%%%%%%%%%



\problemlabel

Compute the general solution for the given system of differential equations.

\exerciselabel{7}{6.2}\begin{exercise}  \label{c6.6.2d}
$\dps\frac{dX}{dt} = \mattwo{-4}{4}{-1}{0} X$.

\begin{solution}
\ans The general solution to the differential equation is
\[
X(t) = \alpha e^{-2t}\cvectwo{2}{1} + \beta e^{-2t}\cvectwo{2t + 1}{t + 1}.
\]

\soln First, find the eigenvalues of $C$, which are the roots of the
characteristic polynomial
\[
p_C(\lambda) = \lambda^2 + 4\lambda + 4 = (\lambda + 2)^2.
\]
Thus, $C$ has a double eigenvalue at $\lambda_1 = -2$.  Since $C$ is not
a multiple of $I_2$, $C$ has only one linearly independent eigenvector.
Find this eigenvector by solving the equation
\[
(C - \lambda_1I_2)v_1 = \left(\mattwo{-4}{4}{-1}{0} + \mattwo{2}{0}{0}{2}
\right)v_1 = \mattwo{-2}{4}{-1}{2}v_1 = 0,
\]
obtaining $v_1 = (2,1)^t$.  Find the generalized eigenvector $w_1$ by
solving the equation $(C - \lambda_1 I_2)w_1 = v_1$, that is,
\[
\mattwo{-2}{4}{-1}{2}w_1 = \vectwo{2}{1}.
\]
So $w_1 = (1,1)^t$ is the generalized eigenvector.
Now, by \eqref{e:exp1eva}, we know that the
general solution to $\dot{X} = CX$ when $C$ has equal eigenvalues and only
one independent eigenvector is
\[
X(t) = e^{\lambda_1 t}(\alpha v_1 + \beta(w_1 + tv_1)).
\]
In this case,
\[
X(t) = e^{-2t}\left(\alpha \vectwo{2}{1} + \beta\left(\vectwo{1}{1} +
t\vectwo{2}{1}\right)\right).
\]






\end{solution}
\end{exercise}


%%%%%%%%%%%%%%%%%%%%%%%%%%%%%%%%%%%%%%%%%%%%%%%%%%%%%%%%%%%%%%%%



\problemlabel



\exerciselabel{1}{6.3}\begin{exercise} \label{c6.5.1}
Suppose that the matrices $A$ and $B$ are similar and the matrices
$B$ and $C$ are similar.  Show that $A$ and $C$ are also similar
matrices.

\begin{solution}

Since $A$ and $B$ are similar and $B$ and $C$ are similar,
$A = P^{-1}BP$ for some matrix $P$, and $B = Q^{-1}BQ$
for some matrix $Q$.  Therefore,
\[ A = P^{-1}BP = P^{-1}Q^{-1}CQP. \]
By Proposition~\ref{P:invprod}, $(QP)^{-1} = P^{-1}Q^{-1}$, so
\[ A = (QP)^{-1}C(QP) \]
thus, $A$ and $C$ are similar.

\end{solution}
\end{exercise}


%%%%%%%%%%%%%%%%%%%%%%%%%%%%%%%%%%%%%%%%%%%%%%%%%%%%%%%%%%%%%%%%



\problemlabel

\noindent Determine whether or not the given matrices are similar, and why.

\exerciselabel{3}{6.3}\begin{exercise} \label{c6.5.3a}
$A = \mattwo{1}{2}{3}{4} \AND B = \mattwo{2}{-2}{-3}{8}$.

\begin{solution}

\ans Matrices $A$ and $B$ are not similar.

\soln When two matrices are similar, the traces are equal.  In this case,
$\trace(A) = 5$ and $\trace(B) = 10$, so the matrices are not similar.

\end{solution}
\end{exercise}


%%%%%%%%%%%%%%%%%%%%%%%%%%%%%%%%%%%%%%%%%%%%%%%%%%%%%%%%%%%%%%%%



\problemlabel

\exerciselabel{5}{6.3}\begin{exercise} \label{c6.5.4}
Let $B=P\inv AP$ so that $A$ and $B$ are similar matrices.  Suppose
that $v$ is an eigenvector of $B$ with eigenvalue $\lambda$.  Show
that $Pv$ is an eigenvector of $A$ with eigenvalue $\lambda$.

\begin{solution}

Since, $A$ and $B$ are similar matrices, if $Bv = \lambda v$, then
\[ A(Pv) = PP^{-1}APv = PBv = \lambda (Pv). \]
Thus, $Pv$ is an eigenvector of $A$ with eigenvalue $\lambda$.

\end{solution}
\end{exercise}


%%%%%%%%%%%%%%%%%%%%%%%%%%%%%%%%%%%%%%%%%%%%%%%%%%%%%%%%%%%%%%%%



\problemlabel

\exerciselabel{10}{6.3}\begin{exercise} \label{a6.3.1}
Use {\pplane} to verify that the nonzero solutions to the system
\[
\frac{dX}{dt} = CX
\]
where
\begin{equation} \label{a6.3.1_C}
C = \Matrix{0 & -1\\ 1 & 0}
\end{equation}
are circles around the origin.  Let 
\[
P = \Matrix{2 & 1\\ 3 & 4}
\]
and  let 
\[
B = P^{-1}CP =  \Matrix{-2.8 & -3.4\\ 2.6 & 2.8}
\]
Describe the solutions to the system
\begin{equation} \label{a6.3.1_B}
\frac{dX}{dt} = BX.
\end{equation}
What is the relationship between solutions of \eqref{a6.3.1_C} to solutions of \eqref{a6.3.1_B}?

\begin{solution}
\soln 

The phase plane 

\begin{figure}[htb]
                       \centerline{%
%                       \psfig{file=exfigure/F_6_3_a.eps,width=2.0in}
%                       \psfig{file=exfigure/F_6_3_b.eps,width=2.0in}}
                       \psfig{file=F_6_3_a.eps,width=2.0in}
                       \psfig{file=F_6_3_b.eps,width=2.0in}}
                \exercaptwo{a6.3.1}
\end{figure}

\end{solution}
\end{exercise}


%%%%%%%%%%%%%%%%%%%%%%%%%%%%%%%%%%%%%%%%%%%%%%%%%%%%%%%%%%%%%%%%%
%
%
%
%\matlabproblemlabel
%
%\noindent Use {\pplane} to determine whether the origin is a saddle, sink, or source in $\dot{X}=CX$ for the given matrix $C$.
%
%\exerciselabel{11}{6.4}\begin{computerExercise} \label{E:sssb}
%$C=\mattwoc{-10}{-2.7}{4.32}{1.6}$.
%
%\begin{solution}
%\ans The origin is a saddle.
%
%\soln Enter the system into {\pplane}.  Then compute trajectories with
%different initial conditions and note that some trajectories approach the
%origin in forward time, while some approach the origin in backward time.
%
%\end{solution}
%\end{computerExercise}
%
%
%%%%%%%%%%%%%%%%%%%%%%%%%%%%%%%%%%%%%%%%%%%%%%%%%%%%%%%%%%%%%%%%%
%
%
%
%\problemlabel
%
%In Exercises~\ref{A6.4.1}-\ref{A6.4.3} use the given data (the eigenvectors  $v_1,v_2\in\R^2$ and associated eigenvalues $\lambda_1,\lambda_2\in\C$ of the $2\times 2$ matrix $C$, and initial condition $X_0\in\R^2$) to    \begin{enumeratea} \item Find the general solution of the system of differential equations  \begin{equation} \label{A6.4-a} \frac{dX}{dt} = CX.  \end{equation} \item Sketch the trajectory in phase space of \eqref{A6.4-a} with initial condition $X_0$. %\item Sketch the time series of the first coordinate of \eqref{A6.4-a} with initial condition $X_0$. \end{enumeratea}
%
%\exerciselabel{16}{6.4}\begin{exercise} \label{A6.4.1}
%\[
%v_1 = \Matrix{ 1 \\ -1} \quad  v_2 = \Matrix{ 1 \\ 1 }\quad  \lambda_1 = -1\quad \lambda_2 = 3\quad  X_0 = \Matrix{2 \\ 0 }
%\]
%
%\begin{solution}
%\soln Since the eigenvalues are real, the general solution is given by \eqref{E:RD2} and is:
%\[
%X(t) = \alpha_1 e^{-t} \Matrix{ 1 \\ -1} + \alpha_2 e^{3t}\Matrix{ 1 \\ 1 }
%\]
%%The solution to the initial value problem is
%%\[X(0) = \Matrix{2 \\ 0 }=  \alpha_1\Matrix{ 1 \\ -1} + \alpha_2 \Matrix{ 1 \\ 1 }\]
%%That is,  $\alpha_1 = 1 = \alpha_2$.
% 
%\begin{figure}[htb]
%           \centerline{%
%           \psfig{file=exfigure/A6_4_1a.eps,width=2.0in}}
%	 %  \psfig{file=exfigure/A6_4_1b.eps,width=2.0in}}
%           \exercap{A6.4.1}
%\end{figure}
%
%\end{solution}
%\end{exercise}
%
%
%%%%%%%%%%%%%%%%%%%%%%%%%%%%%%%%%%%%%%%%%%%%%%%%%%%%%%%%%%%%%%%%%


\end{document}










