\documentclass{article}

% For preamble materials

\usepackage{pgf,tikz}
\usepackage{mathrsfs}
\usetikzlibrary{shapes,arrows}
\usepackage{framed}
\pgfplotsset{compat=1.16}

\def\fixnote#1{\begin{framed}{\textcolor{red}{Fix note: #1}}\end{framed}}  % Allows insertion of red notes about needed edits
%\def\fixnote#1{}

\def\detail#1{{\textcolor{blue}{Detail: #1}}}   

\graphicspath{
  {./}
  {proofs/}
}

%\pdfOnly{\renewcommand{\answer}[1][[yy]{\fbox{\hspace{1in}\rule[-.3\baselineskip]{0pt}{15pt}}}}


\newcommand{\N}{\mathbb N}
\newcommand{\W}{\mathbb W}
\newcommand{\C}{\mathbb C}
\newcommand{\Z}{\mathbb Z}
\newcommand{\Q}{\mathbb Q}
\newcommand{\R}{\mathbb R}




\title{Math 2568 Homework 2}
\author{\phantom{Dr. Golubitsky}}
\date{Due: Wednesday, September 4, 2019}

\makeatletter
\newlabel{c1.4.1a}{{5}{13}}
\newlabel{c1.4.1c}{{8}{13}}
\newlabel{c2.3.6a}{{1}{34}}
\newlabel{c2.3.6c}{{3}{34}}
\newlabel{c2.3.7a}{{4}{34}}
\newlabel{c2.3.7c}{{6}{34}}
\newlabel{e:refexamp6}{{2.3.14}{34}}
\newlabel{c2.3.10a}{{10}{35}}
\newlabel{c2.3.10b}{{11}{35}}
\newlabel{c2.3.11a}{{12}{35}}
\newlabel{c2.3.11d}{{15}{35}}
\newlabel{c2.3.1a}{{22}{36}}
\newlabel{c2.3.1c}{{24}{36}}
\newlabel{number}{{2.4.6}{40}}
\newlabel{c2.4.3a}{{5}{42}}
\newlabel{c2.4.3d}{{8}{42}}
\newlabel{c2.4.4a}{{9}{42}}
\newlabel{c2.4.4c}{{11}{43}}
\newlabel{c2.5.2a}{{4}{48}}
\newlabel{c2.5.2c}{{6}{48}}
\makeatother
\begin{document}
\maketitle


\problemlabel

\noindent Determine whether the given pair of vectors is perpendicular.

\exerciselabel{8}{1.4}\begin{exercise} \label{c1.4.1c}
  $x=(2,1,4,5)$ and $y=(1,-4,3,-2)$.
    \begin{multipleChoice}
    \choice[correct]{The vectors are perpendicular.}
    \choice{The vectors are not perpendicular.}    
  \end{multipleChoice}
  \begin{hint}
    Compute: $(2,1,4,5) \cdot (1,-4,3,-2) = 0$.
  \end{hint}

\begin{solution}
\ans The vectors are perpendicular.

\soln Compute: $(2,1,4,5) \cdot (1,-4,3,-2) = 0$.

\end{solution}
\end{exercise}

\problemlabel



\exerciselabel{7}{2.1}\begin{exercise} \label{c2.1.11}
\begin{itemize}
\item[(a)] Find a quadratic polynomial $p(x) = ax^2 + bx + c$
  satisfying $p(0) = 1$, $p(1) = 5$, and $p(-1) = -5$.
  \begin{prompt}
    The quadratic $p(x) = \answer{-x^2 + 5x + 1}$ satisfies these conditions.
  \end{prompt}
  \begin{hint}
    Since $p(x) = ax^2 + bx + c$ for any quadratic equation, we find
this solution by evaluating $p(0) = 1$, $p(1) = 5$, and $p(-1) = -5$,
which yields the system of equations
\[
\begin{array}{lrrrrrrrr}
p(0) & = & & & & & c & = & 1 \\
p(1) & = & a & + & b & + & c & = & 5 \\
p(-1) & = & a & - & b & + & c & = & -5\end{array}
\]
We solve this system to obtain $(a,b,c) = (-1,5,1)$, then substitute
these coefficients into the general quadratic.
  \end{hint}
\end{itemize}

\begin{solution}

(a) \ans The quadratic $p(x) = -x^2 + 5x + 1$ satisfies these conditions.

\soln Since $p(x) = ax^2 + bx + c$ for any quadratic equation, we find
this solution by evaluating $p(0) = 1$, $p(1) = 5$, and $p(-1) = -5$,
which yields the system of equations
\[
\begin{array}{lrrrrrrrr}
p(0) & = & & & & & c & = & 1 \\
p(1) & = & a & + & b & + & c & = & 5 \\
p(-1) & = & a & - & b & + & c & = & -5\end{array}
\]
We solve this system to obtain $(a,b,c) = (-1,5,1)$, then substitute
these coefficients into the general quadratic.



\end{solution}
\end{exercise}

\problemlabel

\exerciselabel{5}{2.2}\begin{exercise} \label{c2.2.9}
\begin{itemize}
\item[(a)] Find a vector $u$ normal to the plane $2x+2y+z=3$.
  \begin{prompt}
    \begin{validator}[(ux==uy) && (uy==2*uz) && (uy != 0)]
      \(
        u = \left( \answer[format=integer,id=ux]{2}, \answer[format=integer,id=uy]{2}, \answer[format=integer,id=uz]{1} \right)
      \)
    \end{validator}
  \end{prompt}
\item[(b)] Find a vector $v$ normal to the plane $x+y+2z=4$.
  \begin{prompt}
    \begin{validator}[(vx==vy) && (2*vy==vz) && (vy != 0)]
      \(
        v = \left( \answer[format=integer,id=vx]{1}, \answer[format=integer,id=vy]{1}, \answer[format=integer,id=vz]{2} \right)
      \)
    \end{validator}
  \end{prompt}
\item[(c)] Find the cosine of the angle $\theta$ between the vectors $u$ and $v$.
  \begin{prompt}
    \[
      \cos \theta = \answer{\frac{2}{\sqrt{6}}}
    \]
  \end{prompt}
\end{itemize}

\begin{solution}

(a) $u = (2,2,1)$, since we know that the normal vector to the plane
$ax + by + cz = d$ is $(a,b,c)$.

(b) $v = (1,1,2)$.

(c) $\cos\theta = \frac{u \cdot v}{||u||\;||v||} = \frac{2}{\sqrt{6}}$.
In \Matlabp, type {\tt acos(2/sqrt(6))*180/pi} to obtain $\theta =
35.2644^\circ$.

\end{solution}
\end{exercise}

\problemlabel

\noindent Determine whether the given matrix is in reduced echelon form.

\exerciselabel{1}{2.3}\begin{exercise} \label{c2.3.6a}
$\left(\begin{array}{rrrr}
1 & -1 &  0 &   1   \\
0 &  1 &  0 &  -6    \\
         0 &  0 &  1 &   0   \end{array}\right)$.
     \begin{multipleChoice}
       \choice{The matrix is in reduced echelon form.}         
       \choice[correct]{The matrix is not in reduced echelon form.}         
     \end{multipleChoice}

\begin{solution}
The matrix is not in reduced echelon form.


\end{solution}
\end{exercise}

\problemlabel

\noindent We list the reduced echelon form of an augmented matrix of a system of linear equations.  Which columns in these augmented matrices contain pivots?  Describe all solutions to these systems of equations in the form of \eqref{e:refexamp6}.

\exerciselabel{4}{2.3}     \begin{exercise}
       The solutions of the system are:
\[
\left(\begin{array}{r} x_1 \\ x_2 \\ x_3\end{array} \right)
= \left(\begin{array}{c} \answer{-4}x_2 \\ x_2 \\ \answer{5}\end{array} \right)
\]

\begin{solution}
The $1^{st}$ and $3^{rd}$ columns of the matrix contain
pivots.  The solutions of the system are:

\[
\left(\begin{array}{r} x_1 \\ x_2 \\ x_3\end{array} \right)
= \left(\begin{array}{c} -4x_2 \\ x_2 \\ 5\end{array} \right)
\]

\end{solution}
     \end{exercise}

\problemlabel

\exerciselabel{9}{2.3}\begin{exercise} \label{c2.3.9}
Use row reduction and back substitution to solve the following
system of two equations in three unknowns:
\[
\begin{array}{rcrcrcrc}
 x_1 & - & x_2 & + & x_3 & = & 1 \\
2x_1 & + & x_2 & - & x_3 & = & -1
\end{array}
\]
\begin{hint}
   Row reduce the augmented matrix of the system:
\[
\left(\begin{array}{rrr|r} 1 & -1 & 1 & 1 \\ 2 & 1 & -1 & -1\end{array}\right)
\longrightarrow
\left(\begin{array}{rrr|r} 1 & 0 & 0 & 0 \\ 0 & 1 & -1 & -1\end{array}\right).
\]
\end{hint}
\begin{prompt}
  The solution to this system is
\[
\left(\begin{array}{c} x_1 \\ x_2 \\ x_3\end{array}\right) =
\left(\begin{array}{c} \answer{0} \\ x_3 - \answer{1} \\ x_3\end{array}\right),
\]
where $x_3$ is any real number.
\end{prompt}
Is $(1,2,2)$ a solution to this system?
\begin{multipleChoice}
  \choice{Yes, $(1,2,2)$ is a solution.}
  \choice[correct]{No, $(1,2,2)$ is not a solution.}
\end{multipleChoice}
If not, is there a solution for which $x_3=2$?
\begin{multipleChoice}
  \choice[correct]{Yes, there is a solution for which $x_3 = 2$.}
  \choice{No, there is no solution for which $x_3 = 2$.}  
\end{multipleChoice}
\begin{hint}
  There is a solution for which $x_3 = 2$, namely $(0,1,2)$.
\end{hint}

\begin{solution}

\ans The solution to this system is
\[
\left(\begin{array}{c} x_1 \\ x_2 \\ x_3\end{array}\right) =
\left(\begin{array}{c} 0 \\ x_3 - 1 \\ x_3\end{array}\right),
\]
where $x_3$ is any real number.

\soln Row reduce the augmented matrix of the system:
\[
\left(\begin{array}{rrr|r} 1 & -1 & 1 & 1 \\ 2 & 1 & -1 & -1\end{array}\right)
\longrightarrow
\left(\begin{array}{rrr|r} 1 & 0 & 0 & 0 \\ 0 & 1 & -1 & -1\end{array}\right).
\]
Although $(1,2,2)$ is not a solution to this system, there is a solution
for which $x_3 = 2$, namely $(0,1,2)$.

\end{solution}
\end{exercise}

\problemlabel

\noindent Determine the augmented matrix and all solutions for each system of linear equations

\exerciselabel{11}{2.3}\begin{exercise} \label{c2.3.10b}
$\begin{array}{rcl}
2x-y+z+w & = & 1\\
   x+2y-z+w & = & 7 \end{array}$.
  %BADBAD

\begin{solution}
The augmented matrix for this system is
\[
\left(\begin{array}{rrrr|r} 2 & -1 & 1 & 1 & 1 \\ 1 & 2 & -1 & 1 & 7
\end{array}\right)
\]
which can be row reduced to
\[
\left(\begin{array}{rrrr|r} 1 & 0 & \frac{1}{5} & \frac{3}{5} &
\frac{9}{5} \\ 0 & 1 & -\frac{3}{5} & \frac{1}{5} & \frac{13}{5}
\end{array}\right).
\]
The solution set is therefore
\[
\left(\begin{array}{c} x_1 \\ x_2 \\ x_3 \\ x_4\end{array}\right) =
\left(\begin{array}{c} \frac{9}{5} - \frac{1}{5}x_3 - \frac{3}{5}x_4
\\ \frac{13}{5} + \frac{3}{5}x_3 - \frac{1}{5}x_4 \\ x_3 \\ x_4
\end{array}\right).
\]

\end{solution}
\end{exercise}

\problemlabel

\noindent Consider the augmented matrices representing systems of linear equations, and decide \begin{itemize} \item[(a)] if there are zero, one or infinitely many solutions, and \item[(b)] if solutions are not unique, how many variables can be assigned arbitrary values. \end{itemize}

\exerciselabel{14}{2.3}\begin{exercise} \label{c2.3.11c}
$\left(\begin{array}{ccc|c}  1 & 0 & 2 & 1\\ 0 & 5 & 0 & 2 \\ 0 & 0 & 4 & 3
       \end{array}\right)$.
     \begin{multipleChoice}
       \choice{There are no solutions.}
       \choice[correct]{There is exactly one solution.}
       \choice{There are infinitely many solutions.}
     \end{multipleChoice}
     \begin{hint}
       The row-reduced form of the matrix is:
\[
\left(\begin{array}{rrr|r} 1 & 0 & 2 & 1 \\ 0 & 1 & 0 & \frac{2}{5}
\\ 0 & 0 & 1 & \frac{3}{4}\end{array}\right).
\]
     \end{hint}

\begin{solution}

\ans The system has a unique solution.

\soln The row-reduced form of the matrix is:
\[
\left(\begin{array}{rrr|r} 1 & 0 & 2 & 1 \\ 0 & 1 & 0 & \frac{2}{5}
\\ 0 & 0 & 1 & \frac{3}{4}\end{array}\right).
\]

\end{solution}
\end{exercise}

\matlabproblemlabel

\noindent Use elementary row operations and \Matlab to put each of the given matrices into row echelon form.  Suppose that the matrix is the augmented matrix for a system of linear equations.  Is the system consistent or inconsistent?

\exerciselabel{24}{2.3}\begin{computerExercise} \label{c2.3.1c}
\[
\left(\begin{array}{rrrr}
 -2 & 1 &  9 & 1\\
  3 & 3 & -4 & 2\\
  1 & 4 &  5 & 5
\end{array}\right).
\]

\begin{solution}
The row-reduced matrix is:
\begin{verbatim}
A =
    1.0000   -0.5000   -4.5000   -0.5000
         0    1.0000    2.1111    0.7778
         0         0         0    2.0000
\end{verbatim}
This matrix represents an inconsistent linear system.


\end{solution}
\end{computerExercise}

\problemlabel

\exerciselabel{3}{2.4}\begin{exercise} \label{c2.4.2}
The augmented matrix of a consistent system of five equations in seven
unknowns has rank equal to three.  How many parameters are needed to
specify all solutions?
\begin{prompt}
  There are $\answer{4}$ parameters needed to specify all solutions.
\end{prompt}
\begin{hint}
  According to Theorem \ref{number}, $n - \ell$
parameters are needed to parameterize the set of all solutions of a
linear system, where $n$ is the number of unknowns, and $\ell$ is the
rank of the reduced echelon matrix.  In this case, $n = 7$ and $\ell =
3$.
\end{hint}

\begin{solution}

\ans Four parameters are needed to specify all solutions.

\soln According to Theorem \ref{number}, $n - \ell$
parameters are needed to parameterize the set of all solutions of a
linear system, where $n$ is the number of unknowns, and $\ell$ is the
rank of the reduced echelon matrix.  In this case, $n = 7$ and $\ell =
3$.

\end{solution}
\end{exercise}

\matlabproblemlabel

\noindent Use {\tt rref} on the given augmented matrices to determine whether the associated system of linear equations is consistent or inconsistent.  If the equations are consistent, then determine how many parameters are needed to enumerate all solutions.

\exerciselabel{5}{2.4}\begin{computerExercise} \label{c2.4.3a}
\begin{matlabEquation}\label{MATLAB:17}
A = \left(\begin{array}{rrrrr|r}
2 & 1 & 3 & -2 & 4 & 1 \\
5 & 12 & -1 & 3 & 5 & 1 \\
-4  &  -21 &    11  &  -12  &    2  &    1  \\
23  &  59  &  -8   & 17  &  21  &   4
\end{array}\right) \quad
\end{matlabEquation}

\begin{solution}

\ans Matrix $A$ is consistent and requires 3 parameters to enumerate
all solutions.

\soln
\begin{verbatim}
rref(A) = 
    1.0000         0    1.9474   -1.4211    2.2632    0.5789
         0    1.0000   -0.8947    0.8421   -0.5263   -0.1579
         0         0         0         0         0         0
         0         0         0         0         0         0
\end{verbatim}

\end{solution}
\end{computerExercise}

\matlabproblemlabel

\noindent Compute the rank of the given matrix.

\exerciselabel{10}{2.4}\begin{computerExercise} \label{c2.4.4b}
$\left(\begin{array}{rrrr} 2 & 1 & 0 & 1\\
	-1 & 3 & 2 & 4\\ 5 & -1 & 2 & -2\end{array}\right)$.

\begin{solution}
The rank of the matrix is $3$.

\end{solution}
\end{computerExercise}

\matlabproblemlabel

\noindent Use \Matlab to solve the given system of linear equations to four significant decimal places.

\exerciselabel{5}{2.5}\begin{computerExercise} \label{c2.5.2b}
\[
\begin{array}{rcrcr}
(4-i)x_1 & + & (2+3i)x_2 & = &  -i \\
   i x_1 & - &     4 x_2 & = & 2.2
\end{array}.
\]

\begin{solution}

Enter the left-hand side of each system as matrix {\tt A} 
and the right-hand side as vector {\tt b}:

\begin{verbatim}
A =
   4.0000 - 1.0000i    2.0000 + 3.0000i
        0 + 1.0000i   -4.0000          

b =                              A\b =
        0 - 1.0000i                 0.3006+ 0.2462i
   2.2000                          -0.6116+ 0.0751i
\end{verbatim}

\end{solution}
\end{computerExercise}

\end{document}
