\documentclass{article}

\usepackage{epsfig}

\graphicspath{
  {./}
  {figures/}
  {../laode}
  {../laode/figures}
}

\usepackage{epstopdf}
\epstopdfsetup{outdir=./}

\usepackage{morewrites}
\makeatletter
\newcommand\subfile[1]{%
\renewcommand{\input}[1]{}%
\begingroup\skip@preamble\otherinput{#1}\endgroup\par\vspace{\topsep}
\let\input\otherinput}
\makeatother

\newcommand{\EXER}{}
\newcommand{\includeexercises}{\EXER\directlua{dofile(kpse.find_file("exercises","lua"))}}

\newenvironment{computerExercise}{\begin{exercise}}{\end{exercise}}

%\newcounter{ccounter}
%\setcounter{ccounter}{1}
%\newcommand{\Chapter}[1]{\setcounter{chapter}{\arabic{ccounter}}\chapter{#1}\addtocounter{ccounter}{1}}

%\newcommand{\section}[1]{\section{#1}\setcounter{thm}{0}\setcounter{equation}{0}}

%\renewcommand{\theequation}{\arabic{chapter}.\arabic{section}.\arabic{equation}}
%\renewcommand{\thefigure}{\arabic{chapter}.\arabic{figure}}
%\renewcommand{\thetable}{\arabic{chapter}.\arabic{table}}

%\newcommand{\Sec}[2]{\section{#1}\markright{\arabic{ccounter}.\arabic{section}.#2}\setcounter{equation}{0}\setcounter{thm}{0}\setcounter{figure}{0}}
  
\newcommand{\Sec}[2]{\section{#1}}

\setcounter{secnumdepth}{2}
%\setcounter{secnumdepth}{1} 

%\newcounter{THM}
%\renewcommand{\theTHM}{\arabic{chapter}.\arabic{section}}

\newcommand{\trademark}{{R\!\!\!\!\!\bigcirc}}
%\newtheorem{exercise}{}

\newcommand{\dfield}{{\sf dfield9}}
\newcommand{\pplane}{{\sf pplane9}}
\newcommand{\PPLANE}{{\sf PPLANE9}}

% BADBAD: \newcommand{\Bbb}{\bf}

\newcommand{\R}{\mbox{$\Bbb{R}$}}
\newcommand{\C}{\mbox{$\Bbb{C}$}}
\newcommand{\Z}{\mbox{$\Bbb{Z}$}}
\newcommand{\N}{\mbox{$\Bbb{N}$}}
\newcommand{\D}{\mbox{{\bf D}}}
\usepackage{amssymb}
%\newcommand{\qed}{\hfill\mbox{\raggedright$\square$} \vspace{1ex}}
%\newcommand{\proof}{\noindent {\bf Proof:} \hspace{0.1in}}

\newcommand{\setmin}{\;\mbox{--}\;}
\newcommand{\Matlab}{{M\small{AT\-LAB}} }
\newcommand{\Matlabp}{{M\small{AT\-LAB}}}
\newcommand{\computer}{\Matlab Instructions}
\newcommand{\half}{\mbox{$\frac{1}{2}$}}
\newcommand{\compose}{\raisebox{.15ex}{\mbox{{\scriptsize$\circ$}}}}
\newcommand{\AND}{\quad\mbox{and}\quad}
\newcommand{\vect}[2]{\left(\begin{array}{c} #1_1 \\ \vdots \\
 #1_{#2}\end{array}\right)}
\newcommand{\mattwo}[4]{\left(\begin{array}{rr} #1 & #2\\ #3
&#4\end{array}\right)}
\newcommand{\mattwoc}[4]{\left(\begin{array}{cc} #1 & #2\\ #3
&#4\end{array}\right)}
\newcommand{\vectwo}[2]{\left(\begin{array}{r} #1 \\ #2\end{array}\right)}
\newcommand{\vectwoc}[2]{\left(\begin{array}{c} #1 \\ #2\end{array}\right)}

\newcommand{\ignore}[1]{}


\newcommand{\inv}{^{-1}}
\newcommand{\CC}{{\cal C}}
\newcommand{\CCone}{\CC^1}
\newcommand{\Span}{{\rm span}}
\newcommand{\rank}{{\rm rank}}
\newcommand{\trace}{{\rm tr}}
\newcommand{\RE}{{\rm Re}}
\newcommand{\IM}{{\rm Im}}
\newcommand{\nulls}{{\rm null\;space}}

\newcommand{\dps}{\displaystyle}
\newcommand{\arraystart}{\renewcommand{\arraystretch}{1.8}}
\newcommand{\arrayfinish}{\renewcommand{\arraystretch}{1.2}}
\newcommand{\Start}[1]{\vspace{0.08in}\noindent {\bf Section~\ref{#1}}}
\newcommand{\exer}[1]{\noindent {\bf \ref{#1}}}
\newcommand{\ans}{\textbf{Answer:} }
\newcommand{\matthree}[9]{\left(\begin{array}{rrr} #1 & #2 & #3 \\ #4 & #5 & #6
\\ #7 & #8 & #9\end{array}\right)}
\newcommand{\cvectwo}[2]{\left(\begin{array}{c} #1 \\ #2\end{array}\right)}
\newcommand{\cmatthree}[9]{\left(\begin{array}{ccc} #1 & #2 & #3 \\ #4 & #5 &
#6 \\ #7 & #8 & #9\end{array}\right)}
\newcommand{\vecthree}[3]{\left(\begin{array}{r} #1 \\ #2 \\
#3\end{array}\right)}
\newcommand{\cvecthree}[3]{\left(\begin{array}{c} #1 \\ #2 \\
#3\end{array}\right)}
\newcommand{\cmattwo}[4]{\left(\begin{array}{cc} #1 & #2\\ #3
&#4\end{array}\right)}

\newcommand{\Matrix}[1]{\ensuremath{\left(\begin{array}{rrrrrrrrrrrrrrrrrr} #1 \end{array}\right)}}

\newcommand{\Matrixc}[1]{\ensuremath{\left(\begin{array}{cccccccccccc} #1 \end{array}\right)}}



\renewcommand{\labelenumi}{\theenumi}
\newenvironment{enumeratea}%
{\begingroup
 \renewcommand{\theenumi}{\alph{enumi}}
 \renewcommand{\labelenumi}{(\theenumi)}
 \begin{enumerate}}
 {\end{enumerate}\endgroup}

\newcounter{help}
\renewcommand{\thehelp}{\thesection.\arabic{equation}}

%\newenvironment{equation*}%
%{\renewcommand\endequation{\eqno (\theequation)* $$}%
%   \begin{equation}}%
%   {\end{equation}\renewcommand\endequation{\eqno \@eqnnum
%$$\global\@ignoretrue}}

%\input{psfig.tex}

\author{Martin Golubitsky and Michael Dellnitz}

%\newenvironment{matlabEquation}%
%{\renewcommand\endequation{\eqno (\theequation*) $$}%
%   \begin{equation}}%
%   {\end{equation}\renewcommand\endequation{\eqno \@eqnnum
% $$\global\@ignoretrue}}

\newcommand{\soln}{\textbf{Solution:} }
\newcommand{\exercap}[1]{\centerline{Figure~\ref{#1}}}
\newcommand{\exercaptwo}[1]{\centerline{Figure~\ref{#1}a\hspace{2.1in}
Figure~\ref{#1}b}}
\newcommand{\exercapthree}[1]{\centerline{Figure~\ref{#1}a\hspace{1.2in}
Figure~\ref{#1}b\hspace{1.2in}Figure~\ref{#1}c}}
\newcommand{\para}{\hspace{0.4in}}

\usepackage{ifluatex}
\ifluatex
\ifcsname displaysolutions\endcsname%
\else
\renewenvironment{solution}{\suppress}{\endsuppress}
\fi
\else
\renewenvironment{solution}{}{}
\fi

%\ifxake
%\newenvironment{matlabEquation}{\begin{equation}}{\end{equation}}
%\else
\newenvironment{matlabEquation}%
{\let\oldtheequation\theequation\renewcommand{\theequation}{\oldtheequation*}\begin{equation}}%
  {\end{equation}\let\theequation\oldtheequation}
%\fi

\makeatother



\title{Math 2568 Homework 10}
\author{\phantom{Dr. Golubitsky}}
\date{Due: Friday, November 8, 2019}

\makeatletter
\newlabel{c10.1.1a}{{1}{411}}
\newlabel{c10.1.1c}{{3}{412}}
\newlabel{e:inductdet}{{7.1.9}{407}}
\newlabel{D:determinants}{{7.1.1}{400}}
\newlabel{P:ERO}{{7.1.4}{402}}
\newlabel{c10.2.1a}{{1}{423}}
\newlabel{c10.2.1b}{{2}{423}}
\newlabel{c10.2.9a}{{13}{427}}
\newlabel{c10.2.9b}{{14}{428}}
\newlabel{T:inveig}{{7.2.7}{420}}
\newlabel{T:tracen}{{7.2.9}{421}}
\makeatother
\begin{document}

\maketitle


\problemlabel

\noindent  In Exercises~\ref{c10.1.1a} -- \ref{c10.1.1c} compute the 
determinants of the given matrix.


\exerciselabel{1}{7.1}\begin{exercise} \label{c10.1.1a}
$A = \left(\begin{array}{rrr} -2 & 1 & 0 \\ 4 & 5& 0 \\ 1 & 0 & 2
\end{array} \right)$.

\begin{solution}

\ans The determinant of the matrix is $-28$.

\soln Expand along the third column, obtaining:
\[
\det\matthree{-2}{1}{0}{4}{5}{0}{1}{0}{2} = 2\det\mattwo{-2}{1}{4}{5}
= 2(-14) = -28.
\]

\end{solution}
\end{exercise} 


%%%%%%%%%%%%%%%%%%%%%%%%%%%%%%%%%%%%%%%%%%%%%%%%%%%%%%%%%%%%%%%%



\problemlabel

\exerciselabel{2}{7.1}\begin{exercise} \label{c10.1.1b}
$B = \left(\begin{array}{rrrr} 1 & 0 & 2 & 3 \\ -1 & -2 & 3 & 2
\\ 4 & -2 & 0 & 3 \\ 1 & 2 & 0 & -3 \end{array} \right)$.

\begin{solution}

\ans The determinant of the matrix is $-110$.

\soln First row reduce:
\[
\det\left(\begin{array}{rrrr}
1 & 0 & 2 & 3 \\ 
-1 & -2 & 3 & 2 \\
4 & -2 & 0 & 3 \\
1 & 2 & 0 & -3 \end{array}\right) =
\det\left(\begin{array}{rrrr}
1 & 0 & 2 & 3 \\ 
0 & -2 & 5 & 5 \\
0 & -2 & -8 & -9 \\
0 & 2 & -2 & -6 \end{array}\right).
\]
Then, use formula \eqref{e:inductdet}:
\[ \begin{array}{rcl}
\det\matthree{-2}{5}{5}{-2}{-8}{-9}{2}{-2}{-6} & = &
\begin{array}{l}
(-2)(-8)(-6) + 5(-9)2 + 5(-2)(-2) - 5(-8)2 \\
- 5(-2)(-6) - (-2)(-9)(-2) \end{array} \\
& = & -96 - 90 + 20 + 80 - 60 + 36 \\
& = & -110.
\end{array}
\]

\end{solution}
\end{exercise}


%%%%%%%%%%%%%%%%%%%%%%%%%%%%%%%%%%%%%%%%%%%%%%%%%%%%%%%%%%%%%%%%



\problemlabel



\exerciselabel{5}{7.1}\begin{exercise} \label{c10.1.3}
Show that the determinants of similar $n\times n$ matrices are
equal. 

\begin{solution}

Two $n \times n$ matrices $B$ and $C$ are similar if there exists
an $n \times n$ matrix $P$ such that $B = P^{-1}CP$.  Therefore, by
Definition~\ref{D:determinants}(c),
\[
\det(B) = \det(P^{-1}CP)
= \det(P^{-1})\det(C)\det(P)
= \det(P)^{-1}\det(P)\det(C)
= \det(C).
\]

\end{solution}
\end{exercise}


%%%%%%%%%%%%%%%%%%%%%%%%%%%%%%%%%%%%%%%%%%%%%%%%%%%%%%%%%%%%%%%%



\problemlabel

\exerciselabel{15}{7.1}\begin{exercise}  \label{c10.1.c8} 
Suppose that two $n\times p$ matrices $A$ and $B$ are row
equivalent. \index{row!equivalent} Show that there is an invertible
$n\times n$ matrix $P$ such that $B = PA$.

\begin{solution}

By Proposition~\ref{P:ERO}, every
elementary row operation on $A$ can be represented by an invertible $n
\times n$ matrix $R$.  That is, the matrix $RA$ is row equivalent to
$A$.  If $A$ and $B$ are row equivalent, then there exist matrices
$R_1,\ldots,R_k$ such that $B = R_k\cdots R_1A$.  The product of invertible $n
\times n$ matrices is an invertible $n \times n$ matrix.  Thus $P =
R_k\cdots R_1$ is an invertible $n \times n$ matrix such that $B =
PA$.

\end{solution}
\end{exercise}


%%%%%%%%%%%%%%%%%%%%%%%%%%%%%%%%%%%%%%%%%%%%%%%%%%%%%%%%%%%%%%%%



\problemlabel

\noindent In Exercises~\ref{c10.2.1a} -- \ref{c10.2.1b} determine the 
characteristic polynomial and the eigenvalues of the given matrices.


\exerciselabel{1}{7.2}\begin{exercise} \label{c10.2.1a}
$A = \left(\begin{array}{rrr} -9 & -2 & -10 \\ 3 & 2 & 3 \\
8 & 2 & 9 \end{array}\right)$. 

\begin{solution}

\ans The characteristic polynomial of $A$ is $p_A(\lambda) =
-\lambda^3 + 2\lambda^2 + \lambda - 2$, and the eigenvalues are
$\lambda_1 = 1$, $\lambda_2 = -1$, and $\lambda_3 = 2$.

\soln Compute:
\[
\begin{array}{rcl}
p_A(\lambda) & = & \det(A - \lambda I_3) \\
& = & \cmatthree{-9 - \lambda}{-2}{-10}{3}{2 - \lambda}{3}
{8}{2}{9 - \lambda} \\
& = & (-9 - \lambda)\det\cmattwo{2 - \lambda}{3}{2}{9 - \lambda}
- 3\det\cmattwo{-2}{-10}{2}{9 - \lambda} + \\
& & 8\det\cmattwo{-2}{-10}{2 - \lambda}{3} \\
& = & (-9 - \lambda)(\lambda^2 - 11\lambda + 12)
- 3(2 + 2\lambda) + 8(14 - 10\lambda) \\
& = & -\lambda^3 + 2\lambda^2 + \lambda - 2 \\
& = & (\lambda - 1)(\lambda + 1)(\lambda - 2). \end{array}
\]
The eigenvalues of $A$ are the roots of the characteristic polynomial.

\end{solution}
\end{exercise}


%%%%%%%%%%%%%%%%%%%%%%%%%%%%%%%%%%%%%%%%%%%%%%%%%%%%%%%%%%%%%%%%



\matlabproblemlabel

\noindent In Exercises~\ref{c10.2.9a} -- \ref{c10.2.9b}, use \Matlab to 
compute (a) the eigenvalues, traces, and characteristic polynomials of 
the given matrix.  (b) Use the results from part (a) to confirm 
Theorems~\ref{T:inveig} and \ref{T:tracen}.


\exerciselabel{13}{7.2}\begin{computerExercise} \label{c10.2.9a}
\begin{matlabEquation}\label{find-eigenvalues}
A=\left( \begin{array}{rrrrr}
      -12 & -19 &  -3 &  14 &   0\\
      -12 &  10 &  14 & -19 &   8\\
        4 &  -2 &   1 &   7 &  -3\\
       -9 &  17 & -12 &  -5 &  -8\\
      -12 &  -1 &   7 &  13 & -12
\end{array} \right).
\end{matlabEquation}

\begin{solution}

(a) By calculation in \Matlab using the {\tt eig}, {\tt trace}, and
{\tt poly} commands, the eigenvalues of $A$ are 
\[
\lambda = -0.5861 \pm 20.2517, \quad
\lambda = -12.9416, \quad
\lambda = -9.1033, \AND
\lambda = 5.2171.
\]
The trace of $A$ is $-18$.  The characteristic polynomial of $A$ is
\[
p_A = \lambda^5 + 18\lambda^4 + 433\lambda^3 + 6296\lambda^2 +
429\lambda - 252292.
\]
Note that in order to obtain an accurate value for the characteristic
polynomial, it may be necessary to use the {\tt format} command.

(b) Theorem~\ref{T:inveig} states that the eigenvalues of $A^{-1}$ are
the inverses of the eigenvalues of $A$.  In \Matlab, compute
\begin{verbatim}
eig(inv(A)) =
  -0.1098    
  -0.0773    
  -0.0014 + 0.0493i
  -0.0014 - 0.0493i
   0.1917
\end{verbatim}
Then, compute the inverse of each eigenvalue of $A$ to find that if
$\lambda$ is an eigenvalue of $A$, then $\lambda^{-1}$ is indeed an
eigenvalue of $A^{-1}$. 

\end{solution}
\end{computerExercise}


%%%%%%%%%%%%%%%%%%%%%%%%%%%%%%%%%%%%%%%%%%%%%%%%%%%%%%%%%%%%%%%%


\end{document}










