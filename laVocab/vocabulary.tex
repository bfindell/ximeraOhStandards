% Section 2.2 Anatomy of Figures

\documentclass[nooutcomes]{ximera}
%\documentclass[space,handout,nooutcomes]{ximera}

\usepackage{epsfig}

\graphicspath{
  {./}
  {figures/}
  {../laode}
  {../laode/figures}
}

\usepackage{epstopdf}
\epstopdfsetup{outdir=./}

\usepackage{morewrites}
\makeatletter
\newcommand\subfile[1]{%
\renewcommand{\input}[1]{}%
\begingroup\skip@preamble\otherinput{#1}\endgroup\par\vspace{\topsep}
\let\input\otherinput}
\makeatother

\newcommand{\EXER}{}
\newcommand{\includeexercises}{\EXER\directlua{dofile(kpse.find_file("exercises","lua"))}}

\newenvironment{computerExercise}{\begin{exercise}}{\end{exercise}}

%\newcounter{ccounter}
%\setcounter{ccounter}{1}
%\newcommand{\Chapter}[1]{\setcounter{chapter}{\arabic{ccounter}}\chapter{#1}\addtocounter{ccounter}{1}}

%\newcommand{\section}[1]{\section{#1}\setcounter{thm}{0}\setcounter{equation}{0}}

%\renewcommand{\theequation}{\arabic{chapter}.\arabic{section}.\arabic{equation}}
%\renewcommand{\thefigure}{\arabic{chapter}.\arabic{figure}}
%\renewcommand{\thetable}{\arabic{chapter}.\arabic{table}}

%\newcommand{\Sec}[2]{\section{#1}\markright{\arabic{ccounter}.\arabic{section}.#2}\setcounter{equation}{0}\setcounter{thm}{0}\setcounter{figure}{0}}
  
\newcommand{\Sec}[2]{\section{#1}}

\setcounter{secnumdepth}{2}
%\setcounter{secnumdepth}{1} 

%\newcounter{THM}
%\renewcommand{\theTHM}{\arabic{chapter}.\arabic{section}}

\newcommand{\trademark}{{R\!\!\!\!\!\bigcirc}}
%\newtheorem{exercise}{}

\newcommand{\dfield}{{\sf dfield9}}
\newcommand{\pplane}{{\sf pplane9}}
\newcommand{\PPLANE}{{\sf PPLANE9}}

% BADBAD: \newcommand{\Bbb}{\bf}

\newcommand{\R}{\mbox{$\Bbb{R}$}}
\newcommand{\C}{\mbox{$\Bbb{C}$}}
\newcommand{\Z}{\mbox{$\Bbb{Z}$}}
\newcommand{\N}{\mbox{$\Bbb{N}$}}
\newcommand{\D}{\mbox{{\bf D}}}
\usepackage{amssymb}
%\newcommand{\qed}{\hfill\mbox{\raggedright$\square$} \vspace{1ex}}
%\newcommand{\proof}{\noindent {\bf Proof:} \hspace{0.1in}}

\newcommand{\setmin}{\;\mbox{--}\;}
\newcommand{\Matlab}{{M\small{AT\-LAB}} }
\newcommand{\Matlabp}{{M\small{AT\-LAB}}}
\newcommand{\computer}{\Matlab Instructions}
\newcommand{\half}{\mbox{$\frac{1}{2}$}}
\newcommand{\compose}{\raisebox{.15ex}{\mbox{{\scriptsize$\circ$}}}}
\newcommand{\AND}{\quad\mbox{and}\quad}
\newcommand{\vect}[2]{\left(\begin{array}{c} #1_1 \\ \vdots \\
 #1_{#2}\end{array}\right)}
\newcommand{\mattwo}[4]{\left(\begin{array}{rr} #1 & #2\\ #3
&#4\end{array}\right)}
\newcommand{\mattwoc}[4]{\left(\begin{array}{cc} #1 & #2\\ #3
&#4\end{array}\right)}
\newcommand{\vectwo}[2]{\left(\begin{array}{r} #1 \\ #2\end{array}\right)}
\newcommand{\vectwoc}[2]{\left(\begin{array}{c} #1 \\ #2\end{array}\right)}

\newcommand{\ignore}[1]{}


\newcommand{\inv}{^{-1}}
\newcommand{\CC}{{\cal C}}
\newcommand{\CCone}{\CC^1}
\newcommand{\Span}{{\rm span}}
\newcommand{\rank}{{\rm rank}}
\newcommand{\trace}{{\rm tr}}
\newcommand{\RE}{{\rm Re}}
\newcommand{\IM}{{\rm Im}}
\newcommand{\nulls}{{\rm null\;space}}

\newcommand{\dps}{\displaystyle}
\newcommand{\arraystart}{\renewcommand{\arraystretch}{1.8}}
\newcommand{\arrayfinish}{\renewcommand{\arraystretch}{1.2}}
\newcommand{\Start}[1]{\vspace{0.08in}\noindent {\bf Section~\ref{#1}}}
\newcommand{\exer}[1]{\noindent {\bf \ref{#1}}}
\newcommand{\ans}{\textbf{Answer:} }
\newcommand{\matthree}[9]{\left(\begin{array}{rrr} #1 & #2 & #3 \\ #4 & #5 & #6
\\ #7 & #8 & #9\end{array}\right)}
\newcommand{\cvectwo}[2]{\left(\begin{array}{c} #1 \\ #2\end{array}\right)}
\newcommand{\cmatthree}[9]{\left(\begin{array}{ccc} #1 & #2 & #3 \\ #4 & #5 &
#6 \\ #7 & #8 & #9\end{array}\right)}
\newcommand{\vecthree}[3]{\left(\begin{array}{r} #1 \\ #2 \\
#3\end{array}\right)}
\newcommand{\cvecthree}[3]{\left(\begin{array}{c} #1 \\ #2 \\
#3\end{array}\right)}
\newcommand{\cmattwo}[4]{\left(\begin{array}{cc} #1 & #2\\ #3
&#4\end{array}\right)}

\newcommand{\Matrix}[1]{\ensuremath{\left(\begin{array}{rrrrrrrrrrrrrrrrrr} #1 \end{array}\right)}}

\newcommand{\Matrixc}[1]{\ensuremath{\left(\begin{array}{cccccccccccc} #1 \end{array}\right)}}



\renewcommand{\labelenumi}{\theenumi}
\newenvironment{enumeratea}%
{\begingroup
 \renewcommand{\theenumi}{\alph{enumi}}
 \renewcommand{\labelenumi}{(\theenumi)}
 \begin{enumerate}}
 {\end{enumerate}\endgroup}

\newcounter{help}
\renewcommand{\thehelp}{\thesection.\arabic{equation}}

%\newenvironment{equation*}%
%{\renewcommand\endequation{\eqno (\theequation)* $$}%
%   \begin{equation}}%
%   {\end{equation}\renewcommand\endequation{\eqno \@eqnnum
%$$\global\@ignoretrue}}

%\input{psfig.tex}

\author{Martin Golubitsky and Michael Dellnitz}

%\newenvironment{matlabEquation}%
%{\renewcommand\endequation{\eqno (\theequation*) $$}%
%   \begin{equation}}%
%   {\end{equation}\renewcommand\endequation{\eqno \@eqnnum
% $$\global\@ignoretrue}}

\newcommand{\soln}{\textbf{Solution:} }
\newcommand{\exercap}[1]{\centerline{Figure~\ref{#1}}}
\newcommand{\exercaptwo}[1]{\centerline{Figure~\ref{#1}a\hspace{2.1in}
Figure~\ref{#1}b}}
\newcommand{\exercapthree}[1]{\centerline{Figure~\ref{#1}a\hspace{1.2in}
Figure~\ref{#1}b\hspace{1.2in}Figure~\ref{#1}c}}
\newcommand{\para}{\hspace{0.4in}}

\usepackage{ifluatex}
\ifluatex
\ifcsname displaysolutions\endcsname%
\else
\renewenvironment{solution}{\suppress}{\endsuppress}
\fi
\else
\renewenvironment{solution}{}{}
\fi

%\ifxake
%\newenvironment{matlabEquation}{\begin{equation}}{\end{equation}}
%\else
\newenvironment{matlabEquation}%
{\let\oldtheequation\theequation\renewcommand{\theequation}{\oldtheequation*}\begin{equation}}%
  {\end{equation}\let\theequation\oldtheequation}
%\fi

\makeatother



%\newcommand{\N}{\mathbb N}
%\newcommand{\W}{\mathbb W}
%\newcommand{\C}{\mathbb C}
%\newcommand{\Z}{\mathbb Z}
%\newcommand{\Q}{\mathbb Q}
\renewcommand{\R}{\mathbb R}

\title{Vocabulary Review}
\author{Martin Golubitsky}
\begin{document}
\begin{abstract}
Short-answer, multiple-choice, and select-all questions about key vocabulary. 
\end{abstract}
\maketitle

%Useful questions: 
%
%What is regular quadrilateral? 
% Based on definition of ? 
%Write the Pythagorean theorem. 
%Measure angles. 
%Angle sum in a triangle. 
%Triangulate a figure 

% Based on definition 2.3.1. 
\begin{question}
A linear system of equations that has no solutions is said to be $\answer[format=string]{inconsistent}$.  With one or more solutions, the system is $\answer[format=string]{consistent}$. 
\end{question} 

% Based on definition 2.4.1. 
\begin{question}
Suppose that for matrix $E$, two conditions hold:
\begin{enumerate}
\item The first nonzero entry in each row of $E$ is equal to $1$. This leading entry $1$ is called a 
$\answer[format=string]{pivot}$.
\item The leading entry in the $(i + 1)^{st}$ row of $E$ occurs in a column to the right of the column where the leading entry in the $i^{th}$ row occurs.
\end{enumerate}
Then the matrix $E$ is (row) $\answer[format=string]{echelon form}$.  (Hint: two words.)

Note: A consequence of this definition is that all rows that are identically zero occur at the \wordChoice{\choice{top}\choice[correct]{bottom}} of the matrix.
\end{question} 

% Based on definition 2.4.2. 
\begin{question}
If an $m \times n$ matrix can be transformed into another by a sequence of elementary row operations, the two matrices are said to be $\answer[format=string]{row equivalent}$.  (Hint: two words.)
\end{question} 

% Based on definition 2.4.4. 
\begin{question}
Suppose that for matrix $E$, two conditions hold:
\begin{enumerate}
\item $E$ is in echelon form, and
\item in every column of $E$ having a pivot, every entry in that column other than the pivot is $0$.
\end{enumerate}
Then $E$ is said to be in $\answer[format=string]{reduced echelon form}$.  (Hint: three words.) 
\end{question} 

% Based on definition 2.4.10. 
\begin{question}
Let $A$ be an $m \times n$ matrix that is row equivalent to a reduced echelon form matrix $E$. 
The number of nonzero rows in $E$ is called the $\answer[format=string]{rank}$ of $A$.
% denoted $\textrm{rank}(A)$
\end{question} 

% Based on definition 3.3.1. 
\begin{question}
A mapping $L: \mathbb{R}^n \to \mathbb{R}^m$ is $\answer[format=string]{linear}$ if 
\begin{enumerate}
\item $L(x+y)=\answer[format=string]{L(x)+L(y)}$ for all $x,y\in \mathbb{R}^n$. 
\item $L(cx)=\answer[format=string]{cL(x)}$ for all $x\in \mathbb{R}^n$ and all scalars $c\in \mathbb{R}$.
\end{enumerate}
\end{question} 

% Based on definition 3.3.2. 
\begin{question}
Let $j$ be an integer between $1$ and $n$. The $n$-vector $e_j$ is the vector that has a $\answer{1}$ in the $j^{th}$ entry and a $\answer{0}$ in every other entry. 
\end{question} 

% Based on definition 3.7.1. 
\begin{question}
Given an $n \times n$ matrix $A$, if there is an $n \times n$ matrix $B$ such that 
$AB=I_n$ and $BA=I_n$, then $A$ is said to be $\answer[format=string]{invertible}$, and the matrix $B$ is called the $\answer[format=string]{inverse}$ of $A$. If there is no such matrix $B$, the $A$ is said to be noninvertible, or $\answer[format=string]{singular}$. 
\end{question} 

% Based on definition 3.8.1. 
\begin{question}
If $A = \begin{bmatrix} a & b \\ c & d\end{bmatrix}$, then the determinant of $A$, $det(A) = \answer{ad - bc}$. 
\end{question} 

%% Based on definition 4.4.1. 
%\begin{question}
%An invariant line for a linear system of differential equations is called an eigendirection. 
%Hence, for some real number $\lambda$. 
%$Cv = \lambda v$ % (4.5.8) 
%\end{question} 
%
%% Based on definition 4.5.1. 
%\begin{question}
%A nonzero vector $v$ satisfying (4.5.8) is called an eigenvector of the matrix $C$, and the number $\lambda$ is an eigenvalue of the matrix $C$.
%\end{question} 
%
%% Based on definition 4.6.1. 
%\begin{question}
%The characteristic polynomial of the matrix $A$ is $pA(\lambda) = det(A - \lambda I_2)$. 
%\end{question}
%
%% Based on definition 4.6.2.
%\begin{question}
%For $2 \times 2$ matrices we now generalize our first definition of eigenvalues, definition 4.5.1, to include complex eigenvalues, as follows:
%
%An eigenvalue of $A$ is a root of the characteristic polynomial $pA$.
%\end{question} 
%


\end{document}


% Other item types
%
%
%\begin{question}  
%When three (or more) points all lie on the same line, we say they are \dots
%\begin{multipleChoice}  
%\choice{coplanar.}  
%\choice[correct]{collinear.}  
%\choice{conjoined.}
%\choice{concurrent.}  
%\choice{none of these.}
%\end{multipleChoice}  
%\end{question}
%


%\begin{question}  
%The \textbf{circumcenter} of a triangle is \dots [select all]
%  \begin{selectAll}  
%    \choice{the point of concurrency of the medians.}  
%    \choice{the point of concurrency of the angle bisectors.}  
%    \choice[correct]{the point of concurrency of the perpendicular bisectors.}  
%    \choice{the point of concurrency of the altitudes.}  
%    \choice{the balance point for the triangle.}
%    \choice{the center in the triangle.}
%    \choice{the center of the incircle.}
%    \choice[correct]{the center of the circumcircle.}
%    \choice{equidistant from the sides of the triangle.}
%    \choice[correct]{equidistant from the vertices of the triangle.}    
%  \end{selectAll}  
%\end{question}

%  \begin{explanation}\hfil
%    \begin{itemize}
%    \item Curve $A$ is defined by an
%      \wordChoice{\choice{even}\choice[correct]{odd}} degree
%      polynomial with a \wordChoice{\choice[correct]{positive}\choice{negative}}
%      leading term.
%    \item Curve $B$ is defined by an
%      \wordChoice{\choice{even}\choice[correct]{odd}} degree
%      polynomial with a
%      \wordChoice{\choice{positive}\choice[correct]{negative}} leading
%      term.
%    \item Curve $C$ is defined by an
%      \wordChoice{\choice[correct]{even}\choice{odd}} degree
%      polynomial with a \wordChoice{\choice[correct]{positive}\choice{negative}}
%      leading term.
%    \item Curve $D$ is defined by an
%      \wordChoice{\choice[correct]{even}\choice{odd}} degree
%      polynomial with a \wordChoice{\choice{positive}\choice[correct]{negative}}
%      leading term.
%    \end{itemize}
%  \end{explanation}
