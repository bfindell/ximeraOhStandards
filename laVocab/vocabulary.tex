\documentclass[nooutcomes]{ximera}
%\documentclass[space,handout,nooutcomes]{ximera}

% For preamble materials

\usepackage{pgf,tikz}
\usepackage{mathrsfs}
\usetikzlibrary{shapes,arrows}
\usepackage{framed}
\pgfplotsset{compat=1.16}

\def\fixnote#1{\begin{framed}{\textcolor{red}{Fix note: #1}}\end{framed}}  % Allows insertion of red notes about needed edits
%\def\fixnote#1{}

\def\detail#1{{\textcolor{blue}{Detail: #1}}}   

\graphicspath{
  {./}
  {proofs/}
}

%\pdfOnly{\renewcommand{\answer}[1][[yy]{\fbox{\hspace{1in}\rule[-.3\baselineskip]{0pt}{15pt}}}}


\newcommand{\N}{\mathbb N}
\newcommand{\W}{\mathbb W}
\newcommand{\C}{\mathbb C}
\newcommand{\Z}{\mathbb Z}
\newcommand{\Q}{\mathbb Q}
\newcommand{\R}{\mathbb R}




%\newcommand{\N}{\mathbb N}
%\newcommand{\W}{\mathbb W}
%\newcommand{\C}{\mathbb C}
%\newcommand{\Z}{\mathbb Z}
%\newcommand{\Q}{\mathbb Q}
\renewcommand{\R}{\mathbb R}

\title{Vocabulary Review}
\author{Martin Golubitsky and Brad Findell}
\begin{document}
\begin{abstract}
Questions about key vocabulary for linear algebra.
\end{abstract}
\maketitle


% Based on definition 2.3.1. 
\begin{question}
A linear system of equations that has no solutions is said to be $\answer[format=string]{inconsistent}$.  With one or more solutions, the system is $\answer[format=string]{consistent}$. 
\end{question} 

% Based on definition 2.4.1. 
\begin{question}
Suppose that for matrix $E$, two conditions hold:
\begin{enumerate}
\item The first nonzero entry in each row of $E$ is equal to $1$. This leading entry $1$ is called a 
$\answer[format=string]{pivot}$.
\item The leading entry in the $(i + 1)^{st}$ row of $E$ occurs in a column to the right of the column where the leading entry in the $i^{th}$ row occurs.
\end{enumerate}
Then the matrix $E$ is in (row) $\answer[format=string]{echelon form}$.  (Hint: two words.)

Note: A consequence of this definition is that all rows that are identically zero occur at the \wordChoice{\choice{top}\choice[correct]{bottom}} of the matrix.
\end{question} 

% Based on definition 2.4.2. 
\begin{question}
If an $m \times n$ matrix can be transformed into another by a sequence of elementary row operations, the two matrices are said to be $\answer[format=string]{row equivalent}$.  (Hint: two words.)
\end{question} 

% Based on definition 2.4.4. 
\begin{question}
Suppose that for matrix $E$, two conditions hold:
\begin{enumerate}
\item $E$ is in echelon form, and
\item in every column of $E$ having a pivot, every entry in that column other than the pivot is $0$.
\end{enumerate}
Then $E$ is said to be in $\answer[format=string]{reduced echelon form}$.  (Hint: three words.) 
\end{question} 

% Based on definition 2.4.10. 
\begin{question}
Let $A$ be an $m \times n$ matrix that is row equivalent to a reduced echelon form matrix $E$. 
The number of nonzero rows in $E$ is called the $\answer[format=string]{rank}$ of $A$.
% denoted $\textrm{rank}(A)$
\end{question} 

% Based on definition 3.3.1. 
\begin{question}
A mapping $L: \mathbb{R}^n \to \mathbb{R}^m$ is $\answer[format=string]{linear}$ if 
\begin{enumerate}
\item $L(x+y)=\answer[format=string]{L(x)+L(y)}$ for all $x,y\in \mathbb{R}^n$. 
\item $L(cx)=\answer[format=string]{cL(x)}$ for all $x\in \mathbb{R}^n$ and all scalars $c\in \mathbb{R}$.
\end{enumerate}
\end{question} 

% Based on definition 3.3.2. 
\begin{question}
Let $j$ be an integer between $1$ and $n$. The $n$-vector $e_j$ is the vector that has a $\answer{1}$ in the $j^{th}$ entry and a $\answer{0}$ in every other entry. 
\end{question} 

% Based on definition 3.7.1. 
\begin{question}
Given an $n \times n$ matrix $A$, if there is an $n \times n$ matrix $B$ such that 
$AB=I_n$ and $BA=I_n$, then $A$ is said to be $\answer[format=string]{invertible}$, and the matrix $B$ is called the $\answer[format=string]{inverse}$ of $A$. If there is no such matrix $B$, the $A$ is said to be noninvertible, or $\answer[format=string]{singular}$. 
\end{question} 

% Based on definition 3.8.1. 
\begin{question}
If $A = \begin{bmatrix} a & b \\ c & d\end{bmatrix}$, then the determinant of $A$, $det(A) = \answer{ad - bc}$. 
\end{question} 

%% Based on definition 4.4.1. 
%\begin{question}
%An invariant line for a linear system of differential equations is called an eigendirection. 
%Hence, for some real number $\lambda$. 
%$Cv = \lambda v$ % (4.5.8) 
%\end{question} 
%
%% Based on definition 4.5.1. 
%\begin{question}
%A nonzero vector $v$ satisfying (4.5.8) is called an eigenvector of the matrix $C$, and the number $\lambda$ is an eigenvalue of the matrix $C$.
%\end{question} 
%
%% Based on definition 4.6.1. 
%\begin{question}
%The characteristic polynomial of the matrix $A$ is $pA(\lambda) = det(A - \lambda I_2)$. 
%\end{question}
%
%% Based on definition 4.6.2.
%\begin{question}
%For $2 \times 2$ matrices we now generalize our first definition of eigenvalues, definition 4.5.1, to include complex eigenvalues, as follows:
%
%An eigenvalue of $A$ is a root of the characteristic polynomial $pA$.
%\end{question} 
%


\end{document}


% Other item types
%
%
%\begin{question}  
%When three (or more) points all lie on the same line, we say they are \dots
%\begin{multipleChoice}  
%\choice{coplanar.}  
%\choice[correct]{collinear.}  
%\choice{conjoined.}
%\choice{concurrent.}  
%\choice{none of these.}
%\end{multipleChoice}  
%\end{question}
%


%\begin{question}  
%The \textbf{circumcenter} of a triangle is \dots [select all]
%  \begin{selectAll}  
%    \choice{the point of concurrency of the medians.}  
%    \choice{the point of concurrency of the angle bisectors.}  
%    \choice[correct]{the point of concurrency of the perpendicular bisectors.}  
%    \choice{the point of concurrency of the altitudes.}  
%    \choice{the balance point for the triangle.}
%    \choice{the center in the triangle.}
%    \choice{the center of the incircle.}
%    \choice[correct]{the center of the circumcircle.}
%    \choice{equidistant from the sides of the triangle.}
%    \choice[correct]{equidistant from the vertices of the triangle.}    
%  \end{selectAll}  
%\end{question}

%  \begin{explanation}\hfil
%    \begin{itemize}
%    \item Curve $A$ is defined by an
%      \wordChoice{\choice{even}\choice[correct]{odd}} degree
%      polynomial with a \wordChoice{\choice[correct]{positive}\choice{negative}}
%      leading term.
%    \item Curve $B$ is defined by an
%      \wordChoice{\choice{even}\choice[correct]{odd}} degree
%      polynomial with a
%      \wordChoice{\choice{positive}\choice[correct]{negative}} leading
%      term.
%    \item Curve $C$ is defined by an
%      \wordChoice{\choice[correct]{even}\choice{odd}} degree
%      polynomial with a \wordChoice{\choice[correct]{positive}\choice{negative}}
%      leading term.
%    \item Curve $D$ is defined by an
%      \wordChoice{\choice[correct]{even}\choice{odd}} degree
%      polynomial with a \wordChoice{\choice{positive}\choice[correct]{negative}}
%      leading term.
%    \end{itemize}
%  \end{explanation}
