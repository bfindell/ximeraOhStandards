\documentclass[nooutcomes]{ximera}
%\documentclass[space,handout,nooutcomes]{ximera}

% For preamble materials

\usepackage{pgf,tikz}
\usepackage{mathrsfs}
\usetikzlibrary{shapes,arrows}
\usepackage{framed}
\pgfplotsset{compat=1.16}

\def\fixnote#1{\begin{framed}{\textcolor{red}{Fix note: #1}}\end{framed}}  % Allows insertion of red notes about needed edits
%\def\fixnote#1{}

\def\detail#1{{\textcolor{blue}{Detail: #1}}}   

\graphicspath{
  {./}
  {proofs/}
}

%\pdfOnly{\renewcommand{\answer}[1][[yy]{\fbox{\hspace{1in}\rule[-.3\baselineskip]{0pt}{15pt}}}}


\newcommand{\N}{\mathbb N}
\newcommand{\W}{\mathbb W}
\newcommand{\C}{\mathbb C}
\newcommand{\Z}{\mathbb Z}
\newcommand{\Q}{\mathbb Q}
\newcommand{\R}{\mathbb R}




\title{Inscribed Angles}
\author{Brad Findell}
\begin{document}
\begin{abstract}
Inscribed angles exploration. 
\end{abstract}
\maketitle


\begin{definition}
In a circle, a \textbf{central angle} has the center of the circle as its vertex.  
An \textbf{inscribed angle} has a point on the circle as its vertex. 
An \textbf{arc} of a circle has both a measure and a length.  \textbf{Arc measure} indicates an amount of turning (in degrees).  An \textbf{arc length} is a distance.  
\end{definition}

\begin{center}  
\geogebra{kcq9bpbd}{800}{460}  
\end{center}

\begin{problem}

\begin{enumerate}
\item Keeping points $A$ and $C$ fixed, when point $B$ moves, $m\angle ABC$ \wordChoice{\choice{increases}\choice[correct]{stays the same}\choice{decreases}\choice{varies widely}}.  

\item The arc measure is \wordChoice{\choice[correct]{equal to}\choice{half}\choice{double}\choice{unrelated to}} the measure of the corresponding central angle. 
\item The measure of an inscribed angle is \wordChoice{\choice{equal to}\choice[correct]{half}\choice{double}\choice{unrelated to}} the measure of the corresponding central angle. 
\item The measure of an inscribed angle is \wordChoice{\choice{equal to}\choice[correct]{half}\choice{double}\choice{unrelated to}} the measure of the corresponding arc. 
\end{enumerate}
\end{problem}


Maybe some questions about visually estimating angle measures or arc measures.  


\end{document}
